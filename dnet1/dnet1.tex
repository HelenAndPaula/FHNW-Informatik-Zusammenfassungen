\documentclass[11pt,a4paper]{article}
\usepackage[utf8]{inputenc}
\usepackage{amsmath}
\usepackage{amsfonts}
\usepackage{amssymb}
\author{platzh1rsch}
\title{Datennetze 1}

\begin{document}

\section{Datennetze}
\subsection{Das Netz als eine Plattform}
Das Telefonnetz wurde für Sprachübermittlung (analog) konstruiert. In den 60er- und 0er-Jahren wurde es immer mehr digitalisiert. Parallel dazu enstanden in den 70er- und 80er-Jahren Computernetze. \linebreak
In den 90er-Jahren hat sich unter den Datennetzen das Internet für die breite Öffentlichkeit als Datennetz durchgesetzt.
\subsubsection{Was ist eigentlich "Internet"?}
Internet ist ein Netz das viele verschiedenartige lokale Netze miteinander verbindet. Es gibt standardisierte Kommunikationsregeln (TCP/IP).\linebreak
Datennetze werden in folgende 4 Elemente unterteilt:\linebreak
\begin{itemize}
\item Regeln
\item Nachrichten
\item Kommunikationskanal
\item Netzgeräte
\end{itemize}
\subsection{Dedizierte / Konvergierte Netze}
\subsubsection{Dedizierte Netze}
Früher baute man dedizierte (d.h. spezifische) Netze für eine Anwendung. Diese waren i.d.R. nicht miteinander kompatibel.
\subsubsection{Konvergierte Netze}
Mit der Zeit wurde es möglich verschiedene Dienste über ein Universalnetz zu transportieren. Diese Universalnetze nennt man konvergierte Netze.

\subsection{Architektur des Internets}
\subsubsection{Netzwerkarchitektur}
\begin{itemize}
\item Fehlertoleranz
\item Skalierbarkeit
\end{itemize}
\subsubsection{Vermittlung}
\begin{itemize}
\item Leitungsvermittlung
    \begin{itemize}
    \item konstant. und hohe Qualität
    \item schlechte Ausnutzung der Bandbreite
    \item limitierte Teilnehmerzahl
    \item Kosten
    \item limitierte Bandbreite
    \item Sicherheit
    \end{itemize}
\item Multiplexierung (TDM - Time Division Multiplex)
\item Paketvermittlung
    \begin{itemize}
    \item statistisch Multiplexen -> bessere Ausnutzung des Kanals
    \item günstiger
    \item mehr Overhead
    \item Verzögerung ist ein Problem -> kein QoS (Quality of Service)
    \item Sicherheit muss separat gewährleistet werden
    \end{itemize}
\end{itemize}

\section{Schichtenmodell}
Jedes Paket hat folgende Informationen:
\begin{itemize}
\item Absender
\item Ziel
\item zugehörige Anwendung
\item Nummer des Blocks
\end{itemize}
OSI - Open System Interconnection\linebreak
ISO - Standard Behörde (USA)\linebreak
MTU - Maximum Transmission Unit\linebreak
PDU - Protocol Data Unit (Header + Payload)\linebreak
\subsection{Protokolle}
Regelsammlung wie 2 Gegenüber kommunizieren sollen. Im Internet arbeiten verschiedene Protokolle zusammen und bilden so einen Protokoll-Stack

\subsection{Bekannte Anwendungsprotokolle}
\begin{tabular}{l c}
HTTP & 80 \\
FTP & 80, 81 \\
SSH & 22 \\
Telnet & 23 \\
DNS & 53 \\
\end{tabular}



\section{PacketTracer}
- Switches arbeiten mit MAC-Adressen nicht mit IPs
\section{Cisco}
Cisco Networking Academy: 
http://www.cisco.com/web/learning/netacad/index.html
CCNA
\end{document}