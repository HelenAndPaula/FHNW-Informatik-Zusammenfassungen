\documentclass[11pt,a4paper]{article}
\usepackage[utf8]{inputenc}
\usepackage{amsmath}
\usepackage{amsfonts}
\usepackage{amssymb}
\author{platzh1rsch}
\title{Diskrete Mathematik 1}
\begin{document}



\section{Mengen und Relationen}
\subsection{Naive Mengenlehre}
- Georg Cantor 1845 -1918 \\
Menge: "Sammlung" von Objekten\\
Diese Objekte heissen Elemente.\\
Notation: X $ /in $ M $\rightarrow$ X ist Element von M\\
Eine Menge ist durch ihre Elemente eindeutig bestimmt.\\
\\
Bsp: M = {1,2,3}, M = N $\rightarrow$ N = \{3,1,2\}\\
\\
Beschreibung von Mengen\\
1. Durch Aufzählung: M = \{ 1,2,3 \}\\
2. Durch Prädikate:  M = ${x | P(x)}$  "Menge aller x, die das Prädikat P erfüllen"
3. grafische Darstellung (Venn-Diagramme)\\
Bsp. $a \in A, d \in B, c \in A, c \in B$\\

\subsubsection{Notation}
$\forall x \in G: $   "Für alle x aus der Menge G ..."\\
$\exists x \in G: $   "Es existiert ein Element x in der Menge G ..."\\\\
Beispiele:\\
1. G := N = \{0,1,2,3...\}\\
   A := \{1,2\}\\
   B := \{3,4\}\\
$   A B = \emptyset$\\

\subsubsection{Satz 1}
1. G Grundmenge \\
2. A, B, C Teilmengen von G \\

\subsection{weitere Mengen-Konstruktionen}
\subsubsection{Potenzmenge}
\textbf{Definition:} $ P(M) := \{ x | x M \} $ Potenzmenge von M\\
Die Menge aller Teilmengen von M\\
\textbf{Beispiele}\\
a) $ M:=\{1\} \rightarrow P(M) = \{\emptyset,\{1\}\}$\\
b) $ M:= \{1,2,3\} \rightarrow P(M) = \{\emptyset, \{1\},\{2\},\{3\},\{1,2\},\{2,3\},\{1,3\},\{1,2,3\}\}$\\
c) $ M:= \emptyset \rightarrow P(M) = \{\emptyset \}$

\subsubsection{das kartesische Produkt}
Seien A, B Mengen, $a \in A$, $b \in B$\\
\textbf{Definition:} Das Symbol (a,b) heisst das geordnete Paar von a und b.\\
\textbf{Bemerkung:} (a,b) = (c,d) $\rightarrow$ a=c und b=d\\\\

\textbf{Definition:} Seien A,B Mengen\\
$AxB := \{ (x,y) | x \in A, y \in B \}$ heisst das kartesische Produkt von A und B.\\\\

\textbf{Beispiel:} \\
a) $\{1,2,3\} x \{4,5\}$ // i.a. $AxB \neq BxA$\\
= \{ (1,4),(1,5),(2,4),(2,5),(3,4),(3,5) \} \\
b) \{1,2\}x\{1,2\} = \{(1,1),(1,2),(2,1),(2,2)\}\\
c) A = \{a,b\}\\
$Ax\emptyset$ = \{(a,$\emptyset$),(b,$\emptyset$)\}\\

\subsubsection{Partitionen}
Gegeben eine Menge M\\
\textbf{Definition:} Eine Partition von M ist eine Menge $\pi$\\
$\pi := \{ A_i | i \in I \}$ \\
(I = Indexmenge) mit \\
1.) $A_i \neq \emptyset$\\
2.) $A_i \subset$ M \\
3.) $A_i \cap A_J$ = $\emptyset$\\
4.) $\cup A_i$ = M = $ A_1 \cup A_2 \cup A_3 ... $\\\\

\textbf{Beispiel:}\\
a) M:= N* = \{1,2,3,..\} \\
$A_1 := $\{1\},$ A_2 := $\{2\},$ A_3 := $\{$x \in N* | x \geq 3$\} \\
$\pi$ = \{$A_1,A_2,A_3$\} ist eine Partition von M.\\
b) M := RxR \\
$A_a$ = \{(x,y) | x=a, y $\in$ R \}\\

$\pi$ = \{ $A_a $ | $a \in R $\}

\section{Relationen}
Durch Relationen werden Beziehungen zwischen Objekten ausgedrückt.\\
Eine Relation ist stets eine Teilmenge des kartesischen Produktes.
Seien $M_1,...,M_n$ Mengen\\\\

\textbf{Definition}\\
Eine Teilmenge $ R \subset M_1 x M_2 x ... x M_n$\\
heisst eine n-stellige Relation auf $M_1, M_2,...,M_n$\\\\

\textbf{Beispiel 1:}\\
M = { Einwohner von Brugg }\\
$R_1 \subset MxMxM // M^3 $\\
(a,b,c) $ \in R_1 : <==> $ "a ist Vater von c", "b ist Mutter von c"\\\\

\textbf{Beispiel 2:}\\
$R_2 \subset R^2 = RyR$\\
$R_2 = \{(x,y) | x^2+y^2=1\} \subset RxR$\\\\

\textbf{Beispiel 3:}\\
$R_3 \subset R^2 = RyR$\\
$R_2 = \{(x,y) | y=e^x\}$\\\\

\textbf{Beispiel 4:}\\
Sei A eine beliebige Menge.\\
$R_4 := \{(B,C) | B \subset C \subset C \subset A\} P(A)x(PaA)$\\

\subsection{Beschränkung auf binäre Relationen: R $\subset M_1xM_2$}
\textbf{Notation:} $ xRy : <==> (x,y) \in R \subset M_1 x M_2$ \\
\subsection{Darstellung von binären Relationen auf endlichen Mengen}
Sei {R} $\subset M^2$ = MxM // Relation "auf" der Menge M \\
1) Matrizen\\
M:= \{$m_1,m_2,m_3$\} Wir nummerieren die Elemente
$A_R$ := 3x3 Matrix, $a_ij$ = \{0, falls ($m_i$,$,_j$) $\not\in R$, 1, sonst \}\\
2) (gerichtete Graphen) \\
M:= \{$a_y,a_2,a_3,a_4$\}\\
$R\subset M^2$ : R = \{($a_1,a_4$),($a_4,a_3$),($a_2,a_3$)\}\\
$G_R$ Punkte = Elemente der Menge M

\subsection{Spezielle Eigenschaften von Relationen}
\textbf{Definition}\\
1) $R \subset M^2$ reflexiv: $<==> \forall x \in M : (x,x) \in R$ - alle Loops\\
2) $R \subset M^2$ irreflexiv: $<==> \forall x \in M : (x,x) \not\in R$ - keine Loops\\
3) $R \subset M^2$ symmetrisch: $<==> \forall x,y \in M : (x,x) \in R \rightarrow (y,x) \in R$ - nur Doppelpfeile\\
4) $R \subset M^2$ antisymmetrisch: $<==> \forall x,y \in M : (x,y) \in \wedge (y,x) \in R \rightarrow x=y $ - keine Doppelpfeile\\
5) $R \subset M^2$ transitiv: $<==> \forall x,y,z \in M : (x,y),(y,z) \in R \rightarrow (x,z)\in R$ - Abkürzungen \\

\textbf{Beispiel} M := {1,2,3,4}\\
1) $R_1 = \{ (1,1),(2,2),(1,2),(3,3),(4,4)\} \subset M^2$\\
- reflexiv, antisymmetrisch, transitiv\\
2) $R_2 = \{ (1,2),(2,1),(2,3),(3,2),(1,1)\} \subset M^2$\\
- nur symmetrisch\\
3) $R_3 = \{ (1,1),(1,2),(1,3),(1,4),(2,1),(2,2),(2,3),(2,4),(4,4)\} \subset M^2$\\
- transitiv \\
4) $R_4 = \{ (1,2),(2,3),(2,4),(3,3)\} \subset M^2$\\
- keine speziellen Eigenschaften\\
5) $R_5 = \emptyset $\\
- alles ausser reflexiv

\subsection{Äquivalenzrelationen}
\textbf{Definition}\\
R $\subset$ M x M heisst Äquivalenzrelation \\
1) R ist reflexiv\\
2) R ist symmetrisch\\
3) R ist transitiv\\\\
\textbf{Beispiel}
M := \{1,2,3,4,5\}\\
R = \{(1,1)(2,2),(3,3),(4,4),(5,5),(1,2),(2,1),(1,3),(3,1),(2,3),(3,2),(4,5),(5,4)\}\\
Sei R eine Äquivalenzrelation auf M und a $\in$ M.\\
$\textbf{Definition}$\\
$\lbrack a \rbrack := \{x \in M | (x,a) \in R \}$\\
$\lbrack a \rbrack $ist die Äquivalenzklasse von a bzgl. R (in M.).\\

\textbf{Beispiel}\\
$\lbrack 1 \rbrack = \{1,2,3\} \subset $ M\\
$\lbrack 2 \rbrack = \{2,1,3\} $\\
$\lbrack 3 \rbrack = \{3,2,1\} $\\
$\lbrack 4 \rbrack = \{4,5\} $\\
$\lbrack 5 \rbrack = \{5,4\} $\\


\textbf{Satz 2} \\
Voraussetzung: R ist Äquivalenzrelation auf M\\
Behauptung: $\Pi := \{\lbrack a \rbrack | a \in M\}$ ist eine Partition von M.\\
Beweis: 1) $\forall a \in M : a \in \lbrack a \rbrack , weil R reflexiv : \forall x \in M : (x,x) \in R$\\
d.h. $\lbrack a \rbrack \not$= $\emptyset$ und U $\lbrack$a$\rbrack$ = M (x), a $\in$ M\\
$M = U{a} \subset a \in \lbrack a \rbrack \rightarrow {a} \subset \lbrack a \rbrack $\\
2) $\lbrack a \rbrack \cap \lbrack b \rbrack \not$= $\emptyset \rightarrow \lbrack a \rbrack = \lbrack b \rbrack$ \\
Sei $\lbrack a \rbrack \cap \lbrack \not$=$\emptyset \rightarrow \exists c \in M : c \in \lbrack a \rbrack \cap \lbrack b \rbrack, d.h. c \in \lbrack a \rbrack und c \in \lbrack b \rbrack $\\
$\rightarrow (c,a) \in R und (c,b) \in R$ \\
R symmetrisch: (c,a)$\in$ R $\rightarrow (a,c) \in$ R \\
R transitiv: (a,c) $\in$ R und (c,b) $\in$ R $\rightarrow$ (a,b) $\in$ R\\

Wir zeigen: $\lbrack a \rbrack \subset \lbrack b \rbrack (und \lbrack b \rbrack \subset \lbrack a \rbrack$\\
z.Z. d $\in \lbrack a \rbrack \rightarrow d \in \lbrack b \rbrack$ \\

d $\in \lbrack a \rbrack , d.h. (d,a) \in R, und (a,b) \in R (R ist transitiv)$\\
(d,b) $\in R,d.h. d \in \lbrack b \rbrack$

\subsection{Relationsoperationen}
Sei M eine Menge.\\
Wir betrachten binäre Relationen auf M: R $\subset$ MxM.\\
Seien $R_{1}, R_{2} \subset$ MxM, $R_{1}, R_{2}$ sind Mengen.\\
Wir können die Operationen $\cap$, $\cup$ auf $R_{1}, R_{2}$ anwenden:\\
\\
\textbf{Beispiel:}\\
Sei M = \{1,2,3\}\\
$R_{1}$: (1,1),(1,2),(2,2),(2,1),(2,3),(3,2)\\
$R_{2}$: (1,2),(2,1),(3,3)\\
$R_{1} \cap R_{2}$: (1,2),(2,1)\\
\\
Nun betrachten wir Operationen, die ausnutzen, dass R eine Teilmenge eines kart. Produktes ist:\\
Seien R,S Relationen auf M, d.h. R,S $\subset$ MxM\\
\textbf{Definition:}\\
$R^{T}$:= \{(x,y) | (y,x) $\in$ R \} heisst die transponierte Relation zu R.\\
\textbf{Bemerkung:} Falls $A_{R}$ die Matrix der Relation R ist, dann ist$ A_{R}^{T}$ die Matrix von $R^{T}$.\\
\textbf{Definition:}\\
S $\circ$ R := \{(x,z) | $\exists y \in$ M:(x,y)$\in$ R und (y,z | $\in$ S\}\\
heisst das Produkt von R und S\\
\\
\textbf{Beispiel}\\
1) M = \{1,2,3\}\\
R = \{(1,1),(1,2),(1,3),(2,3)\}\\
%
% TODO: Graph & Matrix
%
$R^{T}$ = \{(1,2),(2,1),(3,1),(3,2)\}\\
\\
%
% TODO: Graph & Matrix
%
2) M = \{1,2,3,4\}\\
R = \{(1,3),(1,4)\}\\
S = \{(3,3),(3,2),(3,4),(4,2)\}\\
\\
%
% TODO: Graph & Matrix für R und S
%
S $\circ$ R: \{(1,2),(1,2),(1,4),(1,3)\}\\
%
% TODO: Graph & Matrix für R°S
%
\\
\textbf{Definition:} I := \{(x,x) | x $\in$M\}\\
\textbf{Bemerkung:} $A_{I}$= 
\[ \left(\begin{array}{ccc}
1 & .. & 0 \\
0 & 1 & 0\\
0 & 0 & 1\\
\end{array}\right).\]
\\
\textbf{Satz 3}\\
Voraussetzung: Q,R,S $\in$ P(MxM)\\
Behauptung:\\
1) (Q$\circ$S)$\circ$R = Q$\circ$(S$\circ$R)\\
2) I$\circ$R = R$\circ$I=R

\subsection{Funktionen}


\end{document}