\documentclass[landscape,twocolumn,a4paper]{article}

\setlength{\oddsidemargin}{-.6in}	 % default=0in
\setlength{\textwidth}{11in}	 % default=9in
\setlength{\columnsep}{0.25in}	 % default=10pt
\setlength{\columnseprule}{1pt}	 % default=0pt (no line)
\setlength{\textheight}{7.5in}	 % default=5.15in
\setlength{\topmargin}{-1.0in}	 % default=0.20in
\setlength{\headsep}{0.25in}	 % default=0.35in
\setlength{\parskip}{1.2ex}
\setlength{\parindent}{0mm}
\usepackage{titlesec}
\titlespacing*{\section}{0pt}{5pt}{-5pt}
\titlespacing*{\subsection}{0pt}{0pt}{-5pt}
\titlespacing*{\subsubsection}{0pt}{0pt}{-5pt}


%Math
\usepackage{amsmath}
\usepackage{amsfonts}
\usepackage{amssymb}
\usepackage{amsthm}
\usepackage{ulem}
\usepackage{stmaryrd} %f\UTF{00FC}r Blitz!
\usepackage{tikz}

\usepackage{multirow}
\usepackage{multicol}


%PageStyle
\usepackage[ngerman]{babel} % deutsche Silbentrennung
\usepackage[utf8]{inputenc} 
\usepackage{fancyhdr, graphicx}
\usepackage[scaled=0.8]{helvet}
%\usepackage{enumitem}
\usepackage{parskip}
%\usepackage[a4paper,top=2cm]{geometry}
%\setlength{\textwidth}{17cm}
%\setlength{\oddsidemargin}{-0.5cm}


%My Commands
\newcommand{\RN}{\mathbb{R}} %Real Number
\newcommand{\NN}{\mathbb{N}} %Natural Number
\newcommand{\QN}{\mathbb{Q}} %Rational Number
\newcommand{\ZN}{\mathbb{Z}} %ganze Zahlen
\newcommand{\CN}{\mathbb{C}}
\newcommand{\PN}{\mathbb{P}} %Primzahle Bitches
\newcommand{\Teilt}{\mid} %|
\newcommand{\Teiltn}{\nmid} %kein teiler
\newcommand{\Potp}{\mathcal{P}} %Potenzmenge
\newcommand{\Pota}{\mathcal{A}}
\newcommand{\Potr}{\mathcal{R}}
\newcommand{\Potn}{\mathcal{N}}
\newcommand{\Bold}[1]{\textbf{#1}} %Boldface
\newcommand{\Kursiv}[1]{\textit{#1}} %Italic
\newcommand{\T}[1]{\text{#1}} %Textmode
\newcommand{\Nicht}[1]{\T{\sout{$ #1 $}}} %Streicht Shit durch
\newcommand{\lra}{\leftrightarrow} %Arrows
\newcommand{\ra}{\rightarrow}
\newcommand{\la}{\leftarrow}
\newcommand{\lral}{\longleftrightarrow}
\newcommand{\ral}{\longrightarrow}
\newcommand{\lal}{\longleftarrow}
\newcommand{\Lra}{\Leftrightarrow}
\newcommand{\Ra}{\Rightarrow}
\newcommand{\La}{\Leftarrow}
\newcommand{\Lral}{\Longleftrightarrow}
\newcommand{\Ral}{\Longrightarrow}
\newcommand{\Lal}{\Longleftarrow}
\newcommand{\Vektor}[1]{\vec{#1}}
\newcommand{\Brace}[1]{\left( #1 \right)} %()
\newcommand{\Bracel}[1]{\left\lbrace #1 \right.} %(
\newcommand{\Bracer}[1]{\right. #1 \right\rbrace} %)
\newcommand{\Brack}[1]{\left\lbrace #1 \right\rbrace} %{}
\newcommand{\Brackl}[1]{\left\lbrace #1 \right.} %{
\newcommand{\Brackr}[1]{\right. #1 \right\rbrace} %}
\newcommand{\Result}[1]{\underline{\underline{#1}}} %Doppelt unterstrichen
\newcommand{\Abs}[1]{\left| #1 \right|} %Absolutbetrag
\newcommand{\Norm}[1]{\Abs{\Abs{ #1 }}} %Norm
\newcommand{\Arrays}[1]{\left(\begin{array}{c}#1\end{array}\right)} %Array mit einer Kolonne ()
\newcommand{\Array}[2]{\left(\begin{array}{#1}#2\end{array}\right)} %Array mit n Kolonnen ()
\newcommand{\Bracka}[2]{\left\lbrace\begin{array}{#1}#2\end{array}\right\rbrace} %Array mit {}
\newcommand{\Brackal}[2]{\left\lbrace\begin{array}{#1} #2 \end{array}\right.} %Array mit {
\newcommand{\Brackar}[2]{\left.\begin{array}{#1} #2 \end{array}\right\rbrace} %Array mit }
\newcommand{\Sumone}[2]{\sum_{#2=1}^{#1}} %Summe von 1
\newcommand{\Sumz}[2]{\sum_{#2=0}^{#1}} %Summe von 0
\newcommand{\Sum}[2]{\sum_{#2}^{#1}} %Allgemeine Summe
\newcommand{\Oneover}[1]{\frac{1}{#1}} %1 über igendwas
\newcommand{\Tablewt}[3]{\begin{table*}[h]\caption{#1} \begin{tabular}{#2}{#3}\end{tabular}\end{table*}} %Table mit Titel
\newcommand{\Oben}[2]{\overset{#1}{#2}} %etwas über etwas anderem
\newcommand{\Unten}[2]{\underset{#1}{#2}} %etwas unter etwas anderem
\newcommand{\Bildcap}[2]{\begin{figure}[htb]\centering\includegraphics[width=0.2\textwidth]{#1} \caption{#2}\end{figure}} %Bild mit beschriftung
\newcommand{\Bildjpeg}[1]{\includegraphics[width=0.2\textwidth]{#1.jpeg}} %Bilder jpeg!!
\newcommand{\Bildjpg}[1]{\includegraphics[width=0.2\textwidth]{#1.jpg}} %Bilder jpg!!
\newcommand{\Bild}[1]{\includegraphics[width=0.4\textwidth]{#1}} %Bilder jpg!!
%Beispiel für lstlisting \lstinputlisting[label=Aufgabe 4a,caption=Aufgabe 4a]{4a.java}




%Config
\renewcommand{\headrulewidth}{0pt}
\setlength{\headheight}{15.2pt}

%Metadata
\title{
	\vspace{5cm}
	WST Cheetsheet
}
\author{ Jan Fässler}
\date{5. Semester (HS 2013)}


% hier beginnt das Dokument
\begin{document}

\subsubsection*{Urnen}
3 Urnen a 20 Kugeln: $U_1$: 4r 16w / $U_2$: 10r 10w / $U_3$: 20r 0w \\
1 Rote wird gezogen, mit welcher wahrscheinlichkeit ist es $U_1$:
\begin{equation*}
\frac{\frac{1}{3} \cdot \frac{4}{20}}{\frac{1}{3} \cdot \frac{4}{20} + \frac{1}{3} \cdot \frac{10}{20} + \frac{1}{3} \cdot 1}
\end{equation*}	
2 Rote werden nacheinander gezogen, mit welcher wahrscheinlichkeit ist es $U_1$:
\begin{equation*}
\frac{\frac{1}{3} \cdot \frac{4}{20} \cdot \frac{4}{20}}{\frac{1}{3} \cdot \frac{4}{20} \cdot \frac{4}{20} + \frac{1}{3} \cdot \frac{10}{20} \cdot \frac{10}{20} + \frac{1}{3} \cdot 1 \cdot 1}
\end{equation*}		

\subsubsection*{Stoffgebiete - Prüfung}
10 Stoffgebiete aus 20 gelernt, 4 werden ausgewählt. - Wie hoch ist die Wahrscheinlichkeit, dass x ausgewählt werden:
\begin{tabular}{c | c | c} 
	0 & 1 & 2\\
	\hline 
	&& \\
	$\frac{\binom{10}{4}}{\binom{20}{4}}$ & $\frac{\binom{10}{1} \cdot \binom{10}{3}}{\binom{20}{4}}$ & $\frac{\binom{10}{2} \cdot \binom{10}{2}}{\binom{20}{4}}$
\end{tabular}

\subsubsection*{Torten}
Konditor stellt x Torten her. Herstellung kostet 8 CHF, der Verkaufspreis beträgt 20 CHF. Wahrscheinlichkeiten sind folgende:
\begin{tabular}{l | c c c c c}
	x & 0 & 1 & 2 & 3 & 4 \\
	\hline
	P(x) & 0.05 & 0.25 & 0.4 & 0.2 & 0.1
\end{tabular} \\
\begin{align*}
	x =& \text{ produzierte Torten} \\
	E(x_0) =&  &0 \\
	E(x_1) =& -8*0.05+0.95*12 =& 11 \\
	E(x_2) =& -16*0.05+0.25*4+0.7*24 =& 17 \\
	E(x_3) =& -24*0.05+0.25*-4+0.4*16+0.3*36 =& 15
\end{align*}

\subsubsection*{Kaputte Artikel}
\begin{tabular}{c c c}
	Hersteller & Marktanteil & defekt \\
	\hline
	A & 50\% & 5\% \\
	B & 30\% & 10\% \\
	C & 20\% & 15\%
\end{tabular} \\
Mit welcher Wahrscheinlichkeit hat es in einem 6er-Pack genau ein defektes Stück? \\
Zuerst mischen, dann abfüllen: \\
$p = 0.5 \cdot 0.05 + 0.3 \cdot 0.1 + 0.2 \cdot 0.15 = 0.085 \rightarrow binopdf(1,6,0.085)$\\
Zuerst abfüllen, dann mischen: \\
$binopdf(1,6,0.05) \cdot 0.5 + binopdf(1,6,0.1) \cdot 0.3 + binopdf(1,6,0.15) \cdot 0.2$

\subsubsection*{Kindergeburt}
Im Jahr 1983 wurden in der Stadt Zürich 2994 Kinder geboren. Davon waren 1562 Knaben. Geben Sie das Konfidenzintervall für die Wahrscheinlichkeit einer Knabengeburt an für $Q=95\%$.
\begin{equation*}
	Q=95\% \hspace{1cm} \alpha = \frac{1 + 0.95}{2} \Rightarrow z_\alpha=norminv(\alpha) = 1.96
\end{equation*}
\begin{equation*}
	\left[ \frac{1562}{2994} - \frac{1.96}{2994} \cdot \sqrt{\frac{1562 \cdot (2994 - 1562)}{2994}} , \frac{1562}{2994} + \frac{1.96}{2994} \cdot \sqrt{\frac{1562 \cdot (2994 - 1562)}{2994}} \right]
\end{equation*}

\subsubsection*{Werkstücken}
Von 1000 Werkstücken einer Sendung erwiesen sich 30 als defekt. Bestimmen Sie das Konfidenzintervall für den Anteil der defekten Stücke in der Gesamtproduktion. Arbeiten Sie mir Q = 90\%. Sie können die approximative Variante verwenden.
\begin{equation*}
	Q=90\% \hspace{1cm} \alpha = \frac{1 + 0.9}{2} \Rightarrow z_\alpha=norminv(\alpha) = 1.6449
\end{equation*}
\begin{equation*}
	\left[ \frac{30}{1000} - \frac{1.6449}{1000} \cdot \sqrt{\frac{30 \cdot (1000 - 30)}{1000}} , \frac{30}{1000} + \frac{1.6449}{1000} \cdot \sqrt{\frac{30 \cdot (1000 - 30)}{1000}} \right]
\end{equation*}

\subsubsection*{Lose}
Eine Losverkäuferin behauptet, dass in ihrem Lostopf 3000 Lose mit den Nummern 1, 2, 3, \dots enthalten sind. Ich möchte dies überprüfen. Dabei kaufe ich 5 Lose und entscheide mich dagegen, falls alle Lose eine Zahl kleiner gleich 1600 tragen. (Ich bin der erste Kunde.) 

Berechnen Sie den Fehler erster Art: ($H_0$ trifft zu wird aber verworfen)
\begin{equation*}
	P(F_1) = \frac{1600}{3000} \cdot \frac{1599}{2999} \cdot \frac{1598}{2998} \cdot \frac{1597}{2997} \cdot \frac{1596}{2996}=0.043
\end{equation*}

Berechnen Sie den Fehler zweiter Art, wenn tatsächlich 2000 Lose im Topf sind. ($H_0$ trifft nicht zu wird aber angenommen)
\begin{equation*}
	P(F_2) = 1- \frac{1600}{2000} \cdot \frac{1599}{1999} \cdot \frac{1598}{1998} \cdot \frac{1597}{1997} \cdot \frac{1596}{1996}=0.6727
\end{equation*}

Berechnen Sie den Fehler zweiter Art, wenn tatsächlich 1000 Lose im Topf sind: 

Da 1000 Lose im Topf und $H_0=3000$ Lose kann $H_0$ nicht eintreten $\rightarrow 0$

\subsubsection*{Verwerfungsbereich berechnen}
\begin{equation*}
	H_0: P(X)=0.5 \hspace{0.5cm} H_1: P(X) \neq 0.5 \hspace{0.5cm} n=3000 \hspace{0.5cm} k=1578 \hspace{0.5cm} \alpha = 0.01 \text{(Irrtum)}
\end{equation*}
\begin{equation*}
	X \sim Bin(3000,0.5) \hspace{1cm} \sigma = \sqrt{n \cdot p \cdot (1 -p)} \hspace{1cm} norminv(\frac{\alpha}{2}, 0, \sigma)=-70.541 \dots
\end{equation*}
Verwerfungsbereich:
\begin{equation*}
	[0, 1500-70,541 \dots,] \cup [1500 + 70.541 \dots, 3000]
\end{equation*}

\subsubsection*{Internet}
Ein Internetprovider möchte im Fichtelgebirge eine Werbekampagne durchführen, da er vermutet, dass dort höchstens 40\% der Haushalte mit langsamem Internetzugang wissen, dass ein schnellerer Zugang möglich ist. Um diese Vermutung zu testen, werden 50 Haushalte mit langsamem Internetzugang zufällig ausgewählt und befragt. Der Provider möchte möglichst vermeiden, dass die Werbekampagne auf Grund des Testergebnisses irrtümlich unterlassen wird.

Geben Sie die hierfür geeignete Nullhypothese an und bestimmen Sie die zugehörige Entscheidungsregel auf einem Signifikanzniveau von 5\%:

Nullhypothese $H_0 : p \leq 0.4$ Alternativhypothese $H_1 : p > 0.4$. \\
$\Rightarrow binoinv(0.95,50,0.4) = 26$ \\
Verwerfungsbereich: $[27 \dots 50]$ 

Beschreiben Sie den Fehler 2. Art in Worten und berechnen Sie ihn für den Fall, dass 50\% der Haushalte über die Internetzugangsmöglichkeiten bereits gut informiert sind:

Obwohl die Haushalte gut informiert sind, wird die Kampagne durchgeführt.
$\Rightarrow binocdf (26, 50, 0.5) = 66.4\%$.

\subsubsection*{Winzigweich}
Die Software-Firma Winzigweich vermutet, dass mindestens 60\% der Bewerber um eine freie Stelle eine Eignungsprüfung einem herkömmlichen Bewerbungsgespräch vorziehen würden.

Kann diese Vermutung (Nullhypothese) auf dem Signifikanzniveau von 5\% abgelehnt werden, wenn bei einer Befragung von 200 zufällig ausgewählten Bewerbern nur 109 eine Eignungsprüfung bevorzugen? \\
$\Rightarrow binoinv(0.05,200,0.6) = 109$

Beschreiben Sie den Fehler 1. Art in Worten: \\
Obwohl mindestens 60\% der Befragten eine Eignungsprüfung vorziehen, wird auf Grund des Tests vermutet, dass dem nicht so ist.
 
\end{document} 