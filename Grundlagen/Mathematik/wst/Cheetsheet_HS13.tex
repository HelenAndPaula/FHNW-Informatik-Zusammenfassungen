\documentclass[landscape,twocolumn,a4paper]{article}

\setlength{\oddsidemargin}{-.6in}	 % default=0in
\setlength{\textwidth}{11in}	 % default=9in
\setlength{\columnsep}{0.25in}	 % default=10pt
\setlength{\columnseprule}{1pt}	 % default=0pt (no line)
\setlength{\textheight}{7.5in}	 % default=5.15in
\setlength{\topmargin}{-1.0in}	 % default=0.20in
\setlength{\headsep}{0.25in}	 % default=0.35in
\setlength{\parskip}{1.2ex}
\setlength{\parindent}{0mm}
\usepackage{titlesec}
\titlespacing*{\section}{0pt}{5pt}{-5pt}
\titlespacing*{\subsection}{0pt}{0pt}{-5pt}
\titlespacing*{\subsubsection}{0pt}{0pt}{-5pt}


%Math
\usepackage{amsmath}
\usepackage{amsfonts}
\usepackage{amssymb}
\usepackage{amsthm}
\usepackage{ulem}
\usepackage{stmaryrd} %f\UTF{00FC}r Blitz!
\usepackage{tikz}

\usepackage{multirow}
\usepackage{multicol}


%PageStyle
\usepackage[ngerman]{babel} % deutsche Silbentrennung
\usepackage[utf8]{inputenc} 
\usepackage{fancyhdr, graphicx}
\usepackage[scaled=0.8]{helvet}
%\usepackage{enumitem}
\usepackage{parskip}
%\usepackage[a4paper,top=2cm]{geometry}
%\setlength{\textwidth}{17cm}
%\setlength{\oddsidemargin}{-0.5cm}


%My Commands
\newcommand{\RN}{\mathbb{R}} %Real Number
\newcommand{\NN}{\mathbb{N}} %Natural Number
\newcommand{\QN}{\mathbb{Q}} %Rational Number
\newcommand{\ZN}{\mathbb{Z}} %ganze Zahlen
\newcommand{\CN}{\mathbb{C}}
\newcommand{\PN}{\mathbb{P}} %Primzahle Bitches
\newcommand{\Teilt}{\mid} %|
\newcommand{\Teiltn}{\nmid} %kein teiler
\newcommand{\Potp}{\mathcal{P}} %Potenzmenge
\newcommand{\Pota}{\mathcal{A}}
\newcommand{\Potr}{\mathcal{R}}
\newcommand{\Potn}{\mathcal{N}}
\newcommand{\Bold}[1]{\textbf{#1}} %Boldface
\newcommand{\Kursiv}[1]{\textit{#1}} %Italic
\newcommand{\T}[1]{\text{#1}} %Textmode
\newcommand{\Nicht}[1]{\T{\sout{$ #1 $}}} %Streicht Shit durch
\newcommand{\lra}{\leftrightarrow} %Arrows
\newcommand{\ra}{\rightarrow}
\newcommand{\la}{\leftarrow}
\newcommand{\lral}{\longleftrightarrow}
\newcommand{\ral}{\longrightarrow}
\newcommand{\lal}{\longleftarrow}
\newcommand{\Lra}{\Leftrightarrow}
\newcommand{\Ra}{\Rightarrow}
\newcommand{\La}{\Leftarrow}
\newcommand{\Lral}{\Longleftrightarrow}
\newcommand{\Ral}{\Longrightarrow}
\newcommand{\Lal}{\Longleftarrow}
\newcommand{\Vektor}[1]{\vec{#1}}
\newcommand{\Brace}[1]{\left( #1 \right)} %()
\newcommand{\Bracel}[1]{\left\lbrace #1 \right.} %(
\newcommand{\Bracer}[1]{\right. #1 \right\rbrace} %)
\newcommand{\Brack}[1]{\left\lbrace #1 \right\rbrace} %{}
\newcommand{\Brackl}[1]{\left\lbrace #1 \right.} %{
\newcommand{\Brackr}[1]{\right. #1 \right\rbrace} %}
\newcommand{\Result}[1]{\underline{\underline{#1}}} %Doppelt unterstrichen
\newcommand{\Abs}[1]{\left| #1 \right|} %Absolutbetrag
\newcommand{\Norm}[1]{\Abs{\Abs{ #1 }}} %Norm
\newcommand{\Arrays}[1]{\left(\begin{array}{c}#1\end{array}\right)} %Array mit einer Kolonne ()
\newcommand{\Array}[2]{\left(\begin{array}{#1}#2\end{array}\right)} %Array mit n Kolonnen ()
\newcommand{\Bracka}[2]{\left\lbrace\begin{array}{#1}#2\end{array}\right\rbrace} %Array mit {}
\newcommand{\Brackal}[2]{\left\lbrace\begin{array}{#1} #2 \end{array}\right.} %Array mit {
\newcommand{\Brackar}[2]{\left.\begin{array}{#1} #2 \end{array}\right\rbrace} %Array mit }
\newcommand{\Sumone}[2]{\sum_{#2=1}^{#1}} %Summe von 1
\newcommand{\Sumz}[2]{\sum_{#2=0}^{#1}} %Summe von 0
\newcommand{\Sum}[2]{\sum_{#2}^{#1}} %Allgemeine Summe
\newcommand{\Oneover}[1]{\frac{1}{#1}} %1 über igendwas
\newcommand{\Tablewt}[3]{\begin{table*}[h]\caption{#1} \begin{tabular}{#2}{#3}\end{tabular}\end{table*}} %Table mit Titel
\newcommand{\Oben}[2]{\overset{#1}{#2}} %etwas über etwas anderem
\newcommand{\Unten}[2]{\underset{#1}{#2}} %etwas unter etwas anderem
\newcommand{\Bildcap}[2]{\begin{figure}[htb]\centering\includegraphics[width=0.2\textwidth]{#1} \caption{#2}\end{figure}} %Bild mit beschriftung
\newcommand{\Bildjpeg}[1]{\includegraphics[width=0.2\textwidth]{#1.jpeg}} %Bilder jpeg!!
\newcommand{\Bildjpg}[1]{\includegraphics[width=0.2\textwidth]{#1.jpg}} %Bilder jpg!!
\newcommand{\Bild}[1]{\includegraphics[width=0.4\textwidth]{#1}} %Bilder jpg!!
%Beispiel für lstlisting \lstinputlisting[label=Aufgabe 4a,caption=Aufgabe 4a]{4a.java}




%Config
\renewcommand{\headrulewidth}{0pt}
\setlength{\headheight}{15.2pt}

%Metadata
\title{
	\vspace{5cm}
	WST Cheetsheet
}
\author{ Jan Fässler}
\date{5. Semester (HS 2013)}


% hier beginnt das Dokument
\begin{document}
\subsubsection*{Laplace-Experiment}
Ein Zufallsexperiment mit $n$ verschiedenen möglichen Ergebnissen, die alle dieselbe Wahrscheinlichkeit, alse $\frac{1}{n}$ haben. Die Wahrscheinlichkeit eines Ereignisses $E \subseteq\Omega$ wird für diesen Fall folgendermassen definiert:
\begin{equation}
  P(E)=\frac{|E|}{|\Omega|}=\frac{\text{Anzahl günstige Ergebnisse}}{\text{Anzahl aller Ergebnisse}}
\end{equation}
Dieses $P$ heisst auch \textbf{Gleichverteilung}. 

\subsubsection*{Produkte}
Wenn es bei einem mehrstufigen Auswahlprozess für das 1. Objekt $n_1$ Möglichkeiten, für das 2. Objekt $n_2$ Möglichkeiten, \dots , und für das $k$-te Objekt $n_k$ Möglichkeiten gibt, dann gibt es für den gesamten Auswahlprozess $n_1*n_2* \dots * n_k$ Möglichkeiten.
\begin{equation}
  |A_1 \times A_2 \times \dots \times A_k| = |A_1| * |A_2| * \dots * |A_k|
\end{equation}

\subsubsection*{Summen}
Wenn es $n_1$ Objekte mit Eigenschaft 1, $n_2$ Objekte mit Eigenschaft 2, \dots, $n_k$ Objekte mit Eigenschaft $k$ gibt, und kein Objekt zwei der Eigenschaften gleichzeitig besitzt, dann gibt es insgesamt $n_1+n_2+$ \dots $+n_k$ Objekte die eine der Eigenschaft besitzen.
\begin{equation}
  |A_1 \cup A_2 \cup \dots \cup A_k| = |A_1| + |A_2| + \dots + |A_k|
\end{equation}

\subsubsection*{Fakultät}
Wir ziehen $n$ mal ohne Zurücklegen mit Beachtung der Reihenfolge aus einer Urne mit $n$ Kugeln. Dan gibt es $n*(n-1)*$ \dots $*2*1$ Möglichkeiten.
\begin{equation}
	n! := n*(n-1)* \dots *2*1 \sim \sqrt{2*\pi*n}*(\frac{n}{e})^n
\end{equation}

\subsubsection*{Binominalkoeffizient}
$\binom{n}{k}$ gibt die Anzahl aller $k$-elementigen Teilmengen einer $n$-elementigen Menge an. \\
\begin{equation}
	 \binom{n}{k} := \frac{n!}{(n-k)!*k!}
\end{equation}

\subsubsection*{Urnenmodel}
\begin{center}
	\begin{tabular}{l|l|l}
		 					&	zur\"{u}cklegen			&	nicht zur\"{u}cklegen\\
		 \hline
		 geordnet		&	$n^k$							&	$n!$ oder $\frac{n!}{(n-k)!}$\\
		 \hline
		 ungeordnet	&	$\binom{k+n-1}{k}$	&	$\binom{n}{k}=\frac{n!}{k!(n-k)!}$
	\end{tabular}
\end{center}

\subsubsection*{Wahrscheinlichkeit}
Eien Wahrscheinlichkeit $P: 2^{\Omega} \rightarrow [0,1]$ erfüllt:
\begin{align}
	 P(\Omega) &= 1 \\
	 \forall E \subseteq \Omega : P(E^c) &=1-O(E)\\
	 P(0) &= 0
\end{align}
\begin{align}
	 P(E_1 \cup E_2 \cup E_3 \cup  \dots ) = P(E_1) + P(E_2) + P(E_3) + \dots \\	 
	 \forall E_1, E_2 \subseteq \Omega : P(E_1\cup E_2) = P(E_1) + P(E_2) - P(E_1 \cap E_2)\\
	 E=\{e_1,e_2, \dots \} \rightarrow P(E) = P(\{e_1\}) + P(\{e_2\}) + \dots
\end{align}

\subsubsection*{Z-Dichte}
Die Funktion $f_P : \Omega \rightarrow [0,1]$ mit $f_P(w)=P(\{w\})$ heisst Zähldichte von P.
\begin{align}
	 P(E) &=\sum_{e \in E} f_P(e) \\
	 P(E) &=\sum_{e \in \Omega} f_P(e)=1
\end{align}

\subsubsection*{bedingte Wahrscheinlichkeit}
Es sei $B \subseteq \Omega$ mit $P(B) > 0$. Dann heisst $P(A|B)$ (elementare) bedingte Wahrscheibnlichkeit von A unter B.
\begin{equation}
	 P(A|B) := \frac{P(A \cap B)}{P(B)}
\end{equation}

Formel von Bayes:
\begin{equation}
	 P(A|B)=\frac{P(A)}{P(B)} * P(B | A)
\end{equation}

\subsubsection*{totele Wahrscheinlichkeit}
Es sei  $B_i(i \in I)$ eine Zerlegung von $\Omega$ (d.h. die $B_i$ sind paarweise disjunkt und $\Omega=U_{i \in I} B_i$) mit $P(B_i) > 0$.
\begin{equation}
	 P(A) = \sum_{i \in I} P(A | B_i) * P(B_i)
\end{equation}

\subsubsection*{positive prädiktive Wert}
Für $0 < P(A) < 1$ gilt mit $\Omega = A \cup A^c$ insbesondere:
\begin{equation}
	 P(A | B) = \frac{P(A)*P(A | B)}{P(B | A) * P(A) + P(B | A^c) * P(A^c)}
\end{equation}

\subsubsection*{stochastische Unabhängigkeit}
Zwei Ereignisse $A,B \subseteq \Omega$ heissen stochastisch unabhängig, falls 
\begin{equation}
	P(A \cap B) = P(A) * P(B)  \equiv \underbrace{P(A|B)}_{\frac{P(A \cap B)}{P(B)}}=P(A) \leftarrow (B) \neq 0]
\end{equation}

\subsubsection*{Mehrstufige Zufallsexperimente}
Gegeben ist eine Urne mit 4 weissen und 2 schwarzen Kugeln. Wir ziehen dreimal ohne Zurücklegen: Was ist die Warhscheinlichkeit A: \"{ }Dritte Kugel weiss\"{}? \\
$\Omega = \{w,s\} \times \{w,s\} \times \{w,s\}$ und $A= \{(w,w,w),(w,s,w),(s,w,w),(s,s,w)\}$
\begin{equation}
	P(A) = f(w,w,w) + f(w,s,w) + f(s,w,w) + f(s,s,w)
\end{equation}

\subsubsection*{Verteilungsfunktion}
Es sei $X : \Omega \rightarrow X$ eine Zufallsvariable, wobei $X \in \RN$ eine endliche oder abzählbare Menge ist. Zudem sei $f$ die Zähldichte von $X$.
\begin{equation}
	F(x)=P(X \leq x) = \sum_{t \in X:t \leq x} f(t)
\end{equation}

\subsubsection*{Erwartungswert, Varianz \& Standardabweichung}
\begin{align}
	E(X) &=\sum_{x \in X} x * f(x)=\sum_{x \in X} x * P(X = x) \\
	V(X)&= \sum_{x \in X} (x - E)^2 * f(x) = \sum_{x \in X} (x - E)^2 * P(X=x) = E(X^2)-E(X)^2 \\
	\sigma(X) &= \sqrt{V(X)}
\end{align}

\subsubsection*{Bernoulli-Verteilung}
Treffer ($WK$ $p$), nicht Treffer
\begin{align}
	X \sim B(p): P(0) &= 1 - p \\
	P(1) &= p \\
	E(X) &= p \\
	V(X) &= p*(1-p)
\end{align} 

\subsubsection*{Binomial-Verteilung}
Anzahl Treffer ($WK$ $p$) in $n$ unabhängigen Versuchen
\begin{align}
	X \sim Bin(n,p): P(X=k) &= \binom{n}{k} * p^k * (1-p)^{n-k}, k=0,1, \dots, n \\
	E(X) &= n*p \\
	V(X) &= n*p*(1-p)
\end{align} 

\subsubsection*{geometirsche-Verteilung}
Versuche bis erster Treffer ($WK$ $p$) in unabhängigen Veruschen
\begin{align}
	X \sim Geo(n,p): P(X=k) &= (1-p)^{k-1}*p, k=0,1, \dots, n \\
	E(X) &= \frac{1}{p}\\
	V(X) &= \frac{1-p}{p^2}
\end{align} 

\subsubsection*{Poisson-Verteilung}
Verteilung für seltene Ereignisse mit im Schnitt $\lambda$ Ereignisse pro Zeit/Ort
\begin{align}
	X \sim Poi(\lambda): P(X = k) &=\frac{\lambda^k}{k!}*e^{-\lambda}, k=0,1, \dots, n \\
	E(X) &= \lambda\\
	V(X) &= \lambda
\end{align} 

\subsubsection*{Eigenschaften}

\begin{align}
	E(X + Y) &= E(X) + E(Y) \\
	E(a * X) &= a * E(X) \\
	E(X + c) &= E(X) + c\\
	V(X + c) &= V(X) \\
	V(a * X) &= a^2* V(X) \\
	E(g(X)) &= \sum_{x} g(x) * P(X=x) \leftarrow [\RN \rightarrow \RN] \\
	E(g(X)) &\neq g(E(X))
\end{align} 
--------------------------------------------------------------------------------------------------------
\\
\\
by Jan Fässler

\end{document} 