\documentclass[10pt]{article}

%Math
\usepackage{amsmath}
\usepackage{amsfonts}
\usepackage{amssymb}
\usepackage{amsthm}
\usepackage{ulem}
\usepackage{stmaryrd} %f\UTF{00FC}r Blitz!
\usepackage{tikz}

\usepackage{multirow}
\usepackage{multicol}
\usepackage{tikz}
\usetikzlibrary{shapes,arrows}

%PageStyle
\usepackage[ngerman]{babel} % deutsche Silbentrennung
\usepackage[utf8]{inputenc} 
\usepackage{fancyhdr, graphicx}
\usepackage[scaled=0.92]{helvet}
\usepackage{enumitem}
\usepackage{parskip}
\usepackage[a4paper,top=2cm]{geometry}
\setlength{\textwidth}{17cm}
\setlength{\oddsidemargin}{-0.5cm}


%My Commands
\newcommand{\RN}{\mathbb{R}} %Real Number
\newcommand{\NN}{\mathbb{N}} %Natural Number
\newcommand{\QN}{\mathbb{Q}} %Rational Number
\newcommand{\ZN}{\mathbb{Z}} %ganze Zahlen
\newcommand{\CN}{\mathbb{C}} %Komplexi Zahle lawl
\newcommand{\PN}{\mathbb{P}} %Primzahle Bitches
\newcommand{\Teilt}{\mid} %|
\newcommand{\Teiltn}{\nmid} %kein teiler
\newcommand{\Potp}{\mathcal{P}} %Potenzmenge
\newcommand{\Pota}{\mathcal{A}}
\newcommand{\Potr}{\mathcal{R}}
\newcommand{\Potn}{\mathcal{N}}
\newcommand{\Bold}[1]{\textbf{#1}} %Boldface
\newcommand{\Kursiv}[1]{\textit{#1}} %Italic
\newcommand{\T}[1]{\text{#1}} %Textmode
\newcommand{\Nicht}[1]{\T{\sout{$ #1 $}}} %Streicht Shit durch
\newcommand{\lra}{\leftrightarrow} %Arrows
\newcommand{\ra}{\rightarrow}
\newcommand{\la}{\leftarrow}
\newcommand{\lral}{\longleftrightarrow}
\newcommand{\ral}{\longrightarrow}
\newcommand{\lal}{\longleftarrow}
\newcommand{\Lra}{\Leftrightarrow}
\newcommand{\Ra}{\Rightarrow}
\newcommand{\La}{\Leftarrow}
\newcommand{\Lral}{\Longleftrightarrow}
\newcommand{\Ral}{\Longrightarrow}
\newcommand{\Lal}{\Longleftarrow}
\newcommand{\Vektor}[1]{\vec{#1}}
\newcommand{\Brace}[1]{\left( #1 \right)} %()
\newcommand{\Bracel}[1]{\left\lbrace #1 \right.} %(
\newcommand{\Bracer}[1]{\right. #1 \right\rbrace} %)
\newcommand{\Brack}[1]{\left\lbrace #1 \right\rbrace} %{}
\newcommand{\Brackl}[1]{\left\lbrace #1 \right.} %{
\newcommand{\Brackr}[1]{\right. #1 \right\rbrace} %}
\newcommand{\Result}[1]{\underline{\underline{#1}}} %Doppelt unterstrichen
\newcommand{\Abs}[1]{\left| #1 \right|} %Absolutbetrag
\newcommand{\Norm}[1]{\Abs{\Abs{ #1 }}} %Norm
\newcommand{\Arrays}[1]{\left(\begin{array}{c}#1\end{array}\right)} %Array mit einer Kolonne ()
\newcommand{\Array}[2]{\left(\begin{array}{#1}#2\end{array}\right)} %Array mit n Kolonnen ()
\newcommand{\Bracka}[2]{\left\lbrace\begin{array}{#1}#2\end{array}\right\rbrace} %Array mit {}
\newcommand{\Brackal}[2]{\left\lbrace\begin{array}{#1} #2 \end{array}\right.} %Array mit {
\newcommand{\Brackar}[2]{\left.\begin{array}{#1} #2 \end{array}\right\rbrace} %Array mit }
\newcommand{\Sumone}[2]{\sum_{#2=1}^{#1}} %Summe von 1
\newcommand{\Sumz}[2]{\sum_{#2=0}^{#1}} %Summe von 0
\newcommand{\Sum}[2]{\sum_{#2}^{#1}} %Allgemeine Summe
\newcommand{\Oneover}[1]{\frac{1}{#1}} %1 über igendwas
\newcommand{\Tablewt}[3]{\begin{table*}[h]\caption{#1} \begin{tabular}{#2}{#3}\end{tabular}\end{table*}} %Table mit Titel
\newcommand{\Oben}[2]{\overset{#1}{#2}} %etwas über etwas anderem
\newcommand{\Unten}[2]{\underset{#1}{#2}} %etwas unter etwas anderem
\newcommand{\Bildcap}[2]{\begin{figure}[htb]\centering\includegraphics[width=0.2\textwidth]{#1} \caption{#2}\end{figure}} %Bild mit beschriftung
\newcommand{\Bildjpeg}[1]{\includegraphics[width=0.2\textwidth]{#1.jpeg}} %Bilder jpeg!!
\newcommand{\Bildjpg}[1]{\includegraphics[width=0.2\textwidth]{#1.jpg}} %Bilder jpg!!
\newcommand{\Bild}[1]{\includegraphics[width=0.4\textwidth]{#1}} %Bilder jpg!!
%Beispiel für lstlisting \lstinputlisting[label=Aufgabe 4a,caption=Aufgabe 4a]{4a.java}

% Code listenings
\usepackage{color}
\usepackage{xcolor}
\usepackage{listings}
\usepackage{caption}
\DeclareCaptionFont{white}{\color{white}}
\DeclareCaptionFormat{listing}{\colorbox{gray}{\parbox{\textwidth}{#1#2#3}}}
\captionsetup[lstlisting]{format=listing,labelfont=white,textfont=white}
\lstdefinestyle{JavaStyle}{
 language=Java,
 basicstyle=\footnotesize\ttfamily, % Standardschrift
 numbers=left,               % Ort der Zeilennummern
 numberstyle=\tiny,          % Stil der Zeilennummern
 stepnumber=5,              % Abstand zwischen den Zeilennummern
 numbersep=5pt,              % Abstand der Nummern zum Text
 tabsize=2,                  % Groesse von Tabs
 extendedchars=true,         %
 breaklines=true,            % Zeilen werden Umgebrochen
 frame=b,         
 %commentstyle=\itshape\color{LightLime}, Was isch das? O_o
 %keywordstyle=\bfseries\color{DarkPurple}, und das O_o
 basicstyle=\footnotesize\ttfamily,
 stringstyle=\color[RGB]{42,0,255}\ttfamily, % Farbe der String
 keywordstyle=\color[RGB]{127,0,85}\ttfamily, % Farbe der Keywords
 commentstyle=\color[RGB]{63,127,95}\ttfamily, % Farbe des Kommentars
 showspaces=false,           % Leerzeichen anzeigen ?
 showtabs=false,             % Tabs anzeigen ?
 xleftmargin=17pt,
 framexleftmargin=17pt,
 framexrightmargin=5pt,
 framexbottommargin=4pt,
 showstringspaces=false      % Leerzeichen in Strings anzeigen ?        
}

\tikzstyle{decision} = [diamond, draw, fill=blue!20,
   text width=5.8em, text badly centered, node distance=3.5cm, inner sep=0pt]
\tikzstyle{block} = [rectangle, draw, fill=blue!20,
   text width=5em, text centered, rounded corners, minimum height=4em]
\tikzstyle{line} = [draw, very thick, color=black!50, -latex']
\tikzstyle{cloud} = [draw, ellipse,fill=red!20, node distance=2.5cm,
   minimum height=2em]

%Config
\renewcommand{\headrulewidth}{0pt}
\setlength{\headheight}{15.2pt}

%Metadata
\title{
	\vspace{5cm}
	Kryptographie
}
\author{Fabio Oesch,  Michael Künzli \& Jan Fässler}
\date{4. Semester (FS 2013)}


% hier beginnt das Dokument
\begin{document}

\section{Uebungen}
\subsection*{Serie 4}
\subsubsection*{Aufgabe 1}
$m=$\begin{tabular}{|c|c|c|c|}\hline
 0011&0101&0110&0000\\\hline
\end{tabular}\\
\underline{Padding} 
\begin{tabular}{|c|c|c|c|c|c|c|c|}\hline
 1&1&1&1&0&0&0&0\\\hline
\end{tabular} $IV=c_0$ (bekannt)
\subsection*{Aufgabe 4 (Broadcast-attack)}
\Bold{Bem:} Sei $n=100$, $e=3$, $m\in\{0,1,2,3,4\}$, $m^e=(m^e$ mod $n)$\\
\Bold{Annahme:} 
\begin{tabular}{lll}
 &$\nearrow$&$c_1:=m^3$ mod $n_1$\\
 Alice&$\to$&$c_2:=m^3$ mod $n_2$\\
 $m$&$c_3:=m^3$ mod $n_3$
\end{tabular}\\
\Bold{$e=3$ für alle Teilnehmer}\\
$ggT(n_i,n_j)=1$, wenn $i\neq j$\\
$m<min(n_1,n_2,n_3)$

\subsection*{Serie 5}
\subsubsection*{Aufgabe 1}
$(n,e)$, $(n,d)$ RSA-Schlüssel Oscar\\
$(n,e_A)$, $(n,?)$ RSA-Schlüssel Alice\\
\Bold{unbekannt} $p,q$ $(n=p\cdot q)$ bzw. $\varphi(n)$\\
\Bold{Ziel:} Finde $\tilde{d_A}$ mit falls $c=m^{m_A}$ mod $n$ ist, gilt $m=c^{\tilde{d_A}}$ mod $n$\\
\Bold{Oscar:} $h:=e\cdot d-1$ (Es gilt $ed-k\varphi(n)=1$, $\varphi(n)\mid h$)\\
$h:=\frac{\Oben{k\varphi(n)}{h}}{ggT(\Unten{k\varphi(n)}{ed-1},e_A)}$ \hspace*{2cm}($ggT(e_A,\varphi(n))=1$, $\varphi(n)\mid h$)\\
$d:=ggT(h,e_A)$, $h:=\frac{h}{d}$ \hspace*{2cm}($\varphi(n)\mid h$)\\
$e_A\cdot\alpha+h\cdot\beta=1$\\
\fbox{$e_A\cdot\tilde{\alpha}+\varphi(n)\cdot\tilde{\beta}=1$} löst der Provider\\
$\tilde{d_A}:=\alpha$ mod $h$\\
\Bold{Behauptung:} $m = c^{\tilde{d_A}}$ mod $n=(m^{e_A})^{\tilde{d_A}}$ mod $n=m^{e_A\cdot \tilde{d_A}}$ mod $n=m^{1+h^{\tilde{\beta}}}=m\cdot (m^h)^{\tilde{\beta}}$ mod $n$ ($(m^h)^{\tilde{\beta}} = $
\begin{lstlisting}
 n = 78654787
 e = 11
 d = 64339331
 ea = 17
 c = m. power_mod(ea, n)
 h = e * d - 1
 gcd(h, ea) //1
 xgcd(ea, h) //1, alpha, beta
 dd = a % h
 mm = c. power_mod(dd, n)
 m = 1337
\end{lstlisting}

\subsection*{Serie 7}
\subsubsection*{Aufgabe 1}
$exp_{\color{red}a} : \ZN_6 \ra \ZN_7$ \\
$exp_{\color{red}a} : x \ra {\color{red}a}^x$ mod 7 \\
\\
$(\ZN_6,\oplus,0): \ZN_6 = \{0,1,2,3,4,5\}$\\
$(\ZN_7,\oplus,1): \ZN^*_7 = \{1,2,3,4,5,6\}$\\
\\
${\color{red}\overbrace{\hspace{4.2cm}}^{\ZN_6}}$\\
\begin{tabular}{| l | l | l | l | l | l | l |}
	\hline
	a & 0 & 1 & 2 & 3 & 4 & 5 \\
	\hline
	2 & 1 & 2 & 4 & 1 & 2 & 3 \\
	\hline
	3 & 1 & 3 & 2 & 6 & 4 & 5 \\
	\hline
\end{tabular} \\
\\
\Bold{a=3} $\Ra$ $exp_a$ beistz eine Umkehrbabbildung: $ind_a : \ZN^*_p \ra \ZN_{p-1}$ \\
\\
\begin{itemize}
	\item[a)] $ind_3(5)=5$ 
	\item[b)] $ind_3(3)=1$
\end{itemize}

\subsubsection*{Aufgabe 2}
\subsubsection*{a)}
n=403 \\
$[\sqrt{403}]=20$ \\ \\
\begin{tabular}{l | l | l}
	t & $t^2-n$ & $t^2-n=s^2, s \in \NN$?  \\
	\hline
	21 & 441-403 = 23 & nein \\
	22 & 484 - 403 = 81 & $81=9^2$ : ja
\end{tabular} \\
${\color{red}\Ra t=22, s=9}$ ${\color{blue}\ra a=(t+s)=31, b=(t-s)=13}$\\
$\Ra n= 403=13*31$  
\subsubsection*{b)}
$n=187$ $a=2$ $k=10$ \\
Berechne: $ggT(a^k-1,n)=ggT(1023,187)=11$ \\
$p:=11$ \\
$q:=\frac{n}{p}=\frac{187}{11}=17$ \\

{\color{green}
\textbf{Ergänzung}: B=10 \\
\textbf{Gesucht}: \\
$\Brackar{l}{q\in \PN mit q \leq 10 : \{2,3,5,7\}\\
\beta(q,B):q^{\beta(q,B)} \leq \beta < q^{\beta(q,B)+1} \\
\beta(2,10)=3$, $\beta(2,10)=2, \beta(5,10)=\beta(7,10)=1} $
 $k:=\prod q^{\beta(q,B)}=2^3*3^2*5*7=72*35=2520$
} \\

Sage: \\
$gcd(\underbrace{2.powermod(k,n)-1}_{0},n)$ 

\subsubsection*{Aufgabe 3}
$factor(n)$ \\
$10000993$\\
$1000003$

\subsubsection*{Aufgabe 4}
\begin{lstlisting}
factor(n) = p * q
phi = (p-1)(q-1)
d=e.inverse_mod(phi)
(n.nth_root(4)).n() // *.n() = numerisch
\end{lstlisting}
{\color{red}Wieners Attacke: $0<d\leq \frac{1}{3} * \sqrt[4]{n}$} \\\\
$e=18439769619$
\subsection*{Serie 8}
\subsubsection*{Aufgabe 1}
\Bild{hash.png}
\subsubsection*{Aufgabe 2}
$\overbrace{
\begin{tabular}{cc}
 \multicolumn{2}{c}{\begin{tabular}{|c|c|l|c|}\hline
  0&\hspace*{2cm}&\hspace*{1cm}&511\\\hline
 \end{tabular}}\\
 $\underbrace{\hspace*{3cm}}_{448}$&$\underbrace{\hspace*{2cm}}_{64}$
\end{tabular}}^{512}$\\
$\underbrace{0110011011001010}_m10\dots010000$, $\Abs{m}=16=(10000)_2$\\
\begin{tabular}{|c|c|}
 \multicolumn{1}{c}{Nr.}&\multicolumn{1}{c}{Bit}\\\hline
 0&0\\\hline
 8&1\\\hline
 13&0\\\hline
 15&0\\\hline
 16&0\\\hline
 17&0\\\hline
 401&0\\\hline
 500&0\\\hline
 510&0\\\hline
 511&0\\\hline
\end{tabular}
\subsubsection*{Aufgabe 3}
\begin{enumerate}
 \item 11
 \item Padding-Block
 \item $c_3$ und $c_4$
 \item 53
 \item
\end{enumerate}

\end{document} 