\documentclass[a4paper, 11pt]{article}


%Math
\usepackage{amsmath}
\usepackage{amsfonts}
\usepackage{amssymb}
\usepackage{amsthm}
\usepackage{ulem}
\usepackage{stmaryrd} %f\UTF{00FC}r Blitz!

%PageStyle
\usepackage[ngerman]{babel} % deutsche Silbentrennung
\usepackage[utf8]{inputenc} 
\usepackage{fontenc}
\usepackage{fancyhdr, graphicx} %for header/footer
\usepackage{wasysym}
\usepackage{fullpage}
\usepackage{textcomp}

% Listings
\usepackage{color}
\usepackage{xcolor}
\usepackage{listings}
\usepackage{caption}

% Code listenings
\DeclareCaptionFont{white}{\color{white}}
\DeclareCaptionFormat{listing}{\colorbox{gray}{\parbox{\textwidth}{#1#2#3}}}
\captionsetup[lstlisting]{format=listing,labelfont=white,textfont=white}

\definecolor{DarkPurple}{rgb}{0.4,0.1,0.4}
\definecolor{DarkCyan}{rgb}{0.0,0.5,0.4}
\definecolor{LightLime}{rgb}{0.4,0.6,0.5}
\definecolor{Blue}{rgb}{0.0,0.0,1.0}
 
\lstdefinestyle{JavaStyle}{
 language=Java,
 basicstyle=\footnotesize\ttfamily, % Standardschrift
 numbers=left,               % Ort der Zeilennummern
 numberstyle=\tiny,          % Stil der Zeilennummern
 stepnumber=5,              % Abstand zwischen den Zeilennummern
 numbersep=5pt,              % Abstand der Nummern zum Text
 tabsize=2,                  % Groesse von Tabs
 extendedchars=true,         %
 breaklines=true,            % Zeilen werden Umgebrochen
 frame=b,         
 %commentstyle=\itshape\color{LightLime}, Was isch das? O_o
 %keywordstyle=\bfseries\color{DarkPurple}, und das O_o
 basicstyle=\footnotesize\ttfamily,
 stringstyle=\color[RGB]{42,0,255}\ttfamily, % Farbe der String
 keywordstyle=\color[RGB]{127,0,85}\ttfamily, % Farbe der Keywords
 commentstyle=\color[RGB]{63,127,95}\ttfamily, % Farbe des Kommentars
 showspaces=false,           % Leerzeichen anzeigen ?
 showtabs=false,             % Tabs anzeigen ?
 xleftmargin=17pt,
 framexleftmargin=17pt,
 framexrightmargin=3pt,
 framexbottommargin=2pt,
 showstringspaces=false      % Leerzeichen in Strings anzeigen ?        
}

%Config
\renewcommand{\headrulewidth}{0pt}
\setlength{\headheight}{15.2pt}
\pagestyle{plain}


%Metadata
\title{
	\vspace{5cm}
	Hardwarenahe Programmierung
}
\author{Jonas Schwambi Schwammberger}
\date{4. Semester (FS 2013)}

\begin{document}
\maketitle
\pagebreak
\tableofcontents
\pagebreak

\section{Register}
\begin{itemize}
	\item \%esp = stack pointer
	\item \%ebp = base pointer
	\item \%eax = accumulator, return Werte von Funktionen werden hier abgelegt.
	\item \%ebx = base index (array manipulation)
	\item \%ecx = counter (array manipulation)
	\item \%edx = data / general register
	\item \%esi = source index (string manipulation)
	\item \%edi = destination index (string manipulation)
	\item \%eip = instruction pointer
\end{itemize}
Ausser \%eip und \%esp sind alles General Purpose Register, man kann auch \%ebx für eine Array-Manipulation verwenden.

\subsection{MOV Instruktion}
movl kann in drei Varianten verwendet werden:
\begin{itemize}
	\item movl "register", "register"
	\item movl "register, [Expression]
	\item movl [Expression], "register"
\end{itemize}

\subsection{Expression}
Generelle Funktion für Expressions: $D(Rb,Ri,S)= Mem[Reg[Rb]+S \cdot Reg[Ri]+D]$\\
\begin{itemize}
	\item D: Konstante in Byte(4 Byte für 64b)
	\item Rb: Base Register
	\item Ri: Index Register, können alle sein ausser \%esp und \%ebp
	\item S: Skalar in Zweierpotenz
\end{itemize}

Beispiele:
\begin{tabular}{|c|c|c|}
	\hline
	Ausdruck & Berechnung & Adresse im Hauptspeicher \\\hline\hline
	0x8(\%edx) & 0xf000 + 0x8 & 0xf008 \\\hline
	(\%edx,\%ecx,4)&0xf000 + 4*0x100&0xf400\\\hline
	0x80(,\%edx,2)&2*0xf000 + 0x80 & 0x1e080\\\hline
\end{tabular}

\section{Function Call}
\subsection{Stack Frame}
\%ebp zeigt immer auf die "Basis" des stacks, heisst alle Adressen kleiner als \%ebp gehören zur momentan ausgeführten Methode. Die Parameter dieser Methode sind dabei auf den Adressen grösser als \%ebp abgespeichert. Die Speicherstelle, auf die \%ebp hinzeigt, ist der \&ebp Wert der vorherigen Methode. 4(\%ebp) beinhaltet die Return-Adresse für diese Methode, alles höher als 4(\%ebp) sind Parameter der momentanen Methode.\\
\begin{tabular}{|c|c}
	$\vdots$&\\\cline{0-0}
	yp\\\cline{0-0}
	xp\\\cline{0-0}
	ret addr\\\cline{0-0}
	\%ebp & $\leftarrow$ pushl \%ebp 			\%ebp\\\cline{0-0}
 	\%ebx\\\cline{0-0}
\end{tabular}

\subsection{Function Call Setup}
Nachdem der Aufrufer die Parameter auf den Stack abgelegt und "Call Function" ausgeführt hat.
\begin{lstlisting}
	pushl %ebp
	movl %esp, %ebp
\end{lstlisting}

 
\subsection{Function Call Teardown}
Falls die Methode einen Rückgabewert hat, muss dieser vorher noch in\%eax abgelegt werden.
\begin{lstlisting}
	movl %ebp, %esp
	pop %ebp
	return
\end{lstlisting}


\section{Instruktionen}
\subsection{Arithmetische Operatoren}

\subsection{Instruktionen für den Methodenaufruf}
\begin{tabular}{|c|c|}
	push Src &  \\\hline
	pop Dest & \\\hline
	call (label) &  \\\hline
	ret &  \\\hline
\end{tabular}


\subsubsection{Binäre Operatoren}
Alle binären Operatoren lesen aus dem Source Register und den berechneten Wert in das Destination Register.

\begin{tabular}{|c|c|}
	\hline
	Befehl & Beschreibung\\\hline\hline
	addl & Dest += Source\\\hline
	subl & Dest -= Source \\\hline
	imull & Dest *= Source \\\hline
	sall & Dest $<<$ Source\\\hline
	sarl & Dest $>>$ Source, füllt mit 1 auf falls MSB = 1 \\\hline
	shrl & Dest $>>$ Source, füllt immer mit 0 auf\\\hline
	leal & siehe LEA Instruction. \\\hline
	xorl & \ldots \\\hline
	andl &\ldots  \\\hline
	orl & \ldots \\\hline
\end{tabular}

\subsubsection{Unäre Operatoren}
\begin{tabular}{|c|c|}
	\hline
	Befehl & Beschreibung\\\hline\hline
	incl & increment \\\hline
	decl & decrement \\\hline
	negl & negate \\\hline
	notl & not operator\\\hline
\end{tabular}

\subsubsection{LEA Instruction}
Vom Internet: LEA, the only instruction that performs memory addressing calculations but doesn't actually address memory. LEA accepts a standard memory addressing operand, but does nothing more than store the calculated memory offset in the specified register, which may be any general purpose register.\\
What does that give us? Two things that ADD doesn't provide:\\
the ability to perform addition with either two or three operands, and
the ability to store the result in any register; not just one of the source operands.

\section{Vergleiche und Konditionen}
Alle Compare Operationen werden durchgeführt, indem verschiedene Flags überprüft werden. Diese Flags werden von den arithmetischen Operationen selber gesetzt, oder durch die Befehle testl oder cmpl. Zum Beispiel überprüft die JUMP ZERO Instruktion, ob das ZERO FLAG von einer anderen Instruktion zuvor gesetzt wurde.

\subsection{Flags}
\begin{tabular}{|c|c|c|}
	\hline
	Abkürzung & Name & wird gesetzt durch \\\hline 
	ZF & Zero Flag & wird von testl gesetzt.\\\hline
	SF & Signed Flag & wird von testl gesetzt.\\\hline
	OF & Overflow Flag & von arithmetischen Operationen gesetzt.\\\hline
	CF & Carry Flag & von arithmetischen Operationen gesetzt.\\\hline
\end{tabular}

\subsection{Vergleichsoperatoren}
\begin{tabular}{|c|c|}
	\hline
	cmpl Var1, Var2 & Rechnet Var2 - Var1, ohne das Resultat in Var2 zu speichern.\\ 
	& Nur die Flags werden verändert. Heisst Var1 == Var2, \\
	& dann ist das Zero Flag gesetzt.\\\hline
	testl & Macht das gleiche wie cmpl, mit dem Unterschied dass es Bitwise AND\\
	& anstatt einer Substraktion macht.\\\hline
	{\color{red}cmovle} & move src to dest if condition c is true(less or equal).\\\hline
\end{tabular}\\ \\
SetX Befehle verändern die Flags direkt, falls man das möchte:\\
\begin{tabular}{|c|c|c|}
	\hline
	Befehl & Ausdruck & Beschreibung\\\hline 
	sete & $ZF$ & Equal / Zero\\\hline 
	setne & $\sim ZF$ & Not Equal / Not Zero\\\hline 
	sets & $SF$ & Negative\\\hline 
	setns & $\sim SF$ & Nonnegative\\\hline  
	setg & $\sim(SF^\wedge OF)\&\sim ZF$ & Greater (Signed)\\\hline 
	setge & $\sim(SF^\wedge OF)$ & Greater or Equal (Signed)\\\hline 
	setl & $(SF^\wedge OF)$ & Less (Signed)\\\hline 
	setle & $(SF^\wedge OF)\mid ZF$ & Less or Equal (Signed)\\\hline 
	seta & $\sim CF\&\sim ZF$ & Above (unsigned)\\\hline 
	setb & CF & Below (unsigned)\\\hline
\end{tabular}

\subsection{Jump}
\begin{tabular}{|c|c|c|}
	\hline
	Befehl & Flags & Beschreibung\\\hline
	jmp (label) & 1 & Bedingungsloser jump\\\hline
	je (label) & ZF & jump equal or zero \\\hline
	jne (label) & ~ZF & jump not equal or not Zero  \\\hline
	js (label) & SF & jump negative \\\hline
	jns (label) & ~SF & jump not negative  \\\hline
	jg (label) & ~(SF\^OF)\&~ZF & jump greater \\\hline
	jge (label) & ~(SF\^OF) & jump greater or equal \\\hline
	jl (label) & (SF\^OF)& jump less  \\\hline
	jle (label) & (SF\^0F)\textbackslash ZF & jump less or equal \\\hline
	ja (label) & \~CF \& \~ZF & jump above (unsigned) \\\hline
	jb (label) & CF & jump below (unsigned) \\\hline
\end{tabular}


\section{Loops und If's}

\subsection{If Statement}
\subsubsection{Unter 32Bit}
C Code:
\begin{lstlisting}
	int absdiff(int x,int y)
	{
		int result;
		
		if(x > y)
			result = x-y;
		else
			result y-x;
			
		return result;
	}
\end{lstlisting}
Assembler:
\begin{lstlisting}
	absdiff:
		pushl %ebp
		movl %esp,%ebp
		
		movl 8(%ebp),%edx
		movl 12(%ebp),%eax
		cmpl %eax,%edx
		jle .L7			 
		movl %edx,%eax 
	.L8:
		movl %ebp,%esp
		popl %ebp
		ret
	.L7:
		subl %edx, %eax
		jmp .L8
\end{lstlisting}

\subsubsection{Unter 64Bit}
C Code, der Selbe wie unter 32 Bit.
Assembler:
\begin{lstlisting}
	absdiff:
		pushl %ebp
		movl %esp,%ebp
		
		movl %edi, %eax	# v = x
		movl %esi, %edx	# ve = y
		subl %esi, %eax # v -= y
		subl %edi, %edx # ve -= x
		cmpl %esi, %edi # x:y
		cmovle %edx %eax # v=ve if <=
		movl %ebp,%esp
		popl %ebp
		ret
\end{lstlisting}

\subsection{Loops}
\subsubsection{Do While Loops}
C Code:
\begin{lstlisting}
	int fact(int x)
	{
		int result = 1;
		do
		{
			result *= x;
			x = x-1;
		} while(x > 1);
		
		return result;
	}
\end{lstlisting}
Intermediate Code, bevor der Code zu Assembler übersetzt wird:
\begin{lstlisting}
	int fact(int x)
	{
		int result = 1;
		
		loop:
			result *= x;
			x = x-1;
			if (x > 1)
		 		goto loop;
		
		return result;
	}
\end{lstlisting}
Assembler:
\begin{lstlisting}
	fact:
		pushl %ebp
		movl %espm%ebp
		movl $1,%eax
		movl 8(%ebp),%edx
	L11:
		imull %edx,%eax
		decl %edx		# Compare x : 1
		cmpl $1,%edx	# if > goto loop
		jg L11
		
		movl %ebp,%esp
		popl %ebp
		ret
\end{lstlisting}

\subsubsection{while loops}
While loops werden vom GCC in einen Do While loop übersetzt.\\
\paragraph{Alte Übersetzungsart}
Pseudocode While:
\begin{lstlisting}
	while(TEST)
		Body
\end{lstlisting}
Pseudo intermediate Code:
\begin{lstlisting}
	if(TEST)
		goto DONE
	LOOP:
		Body
		if(TEST)
			goto LOOP;
	DONE:
\end{lstlisting}

\paragraph{Neue Übersetzungsart}
In der neuen Übersetzungsart wird der unnötige Test vor dem eigentlichen Loop weggelassen, heutige Prozessoren haben keine Performance einbussen bei unconditional jumps\\
Pseudocode:
\begin{lstlisting}
	goto MIDDLE
	LOOP:
		Body
	MIDDLE:
		if(TEST)
			goto LOOP
\end{lstlisting}

\subsubsection{For Loops}
For loops sind eigentlich nur While loops mit einer speziellen letzten Zeile. For loops werden in einen While loop umgewandelt, der wieder in einen do while \dots\\
Pseudocode for:
\begin{lstlisting}
	for(INIT;TEST;UPDATE)
		body
\end{lstlisting}
Pseudocode in while übersetzt:
\begin{lstlisting}
	INIT
	while(TEST)
	{
		body
		UPDATE
	}
\end{lstlisting}

\subsection{Select Case}
Ein Select Case gibt es zwei Möglichkeiten, entweder es wird als eine Reihe von if then else Anweisungen implementiert, was bei vielen Cases sehr langsam wird, oder mittels einer Jump Table. Der GCC entscheidet selbst, was er macht.\\
Beispiel C Code:
\begin{lstlisting}
	typedef enum {ADD,MULT,MINUS,DIV,MOD,BAD} op_type;
	char unparse_symbol(op_type op) {
		switch(op)
		{
			case ADD:
				return '+';
			case MULT:
				return '*';
			case MINUS:
				return '-';
			case DIV:
				return '/';
			case MOD:
				return '%';
			case BAD:
				return '?';
		}
	}
\end{lstlisting}
Jump Table
\begin{lstlisting}
	.section .rodata
		.align 4
	.L57:
		.long .51	//Addresse fuer Case 0
		.long .L52	//Addresse fuer Case 1
		...
		.long .L56  //Addresse fuer Case 5
\end{lstlisting}
Eigentlicher Switch wird ungefähr so übersetzt. Jeder Case bekommt ein eigenes Label. Ebenso wird hier ein weiteres Label hinzugefügt, mitdem wir zu den Instruktionen springen können, die nach dem Switch ausgeführt werden sollen.
\begin{lstlisting}
	.L51
		movl $43,\%eax  //'+'
		jmp .L49
	.L52
		movl $42,\%eax  //'*'
		jmp .L49
	.L53
		movl $45,\%eax  //'-'
		jmp .L49
	...
	
	.L49
		movl %ebp, %esp
		popl %ebp
		ret
\end{lstlisting}
Methodenaufruf, wie die Jump Tabelle und die
\begin{lstlisting}
	unparse_symbol:
		pushl %ebp
		movl %esp,%ebp
		
		movl 8(%ebp), %eax  #eax = op
		cmpl $5,%eax 		
		ja	.L49			# if > goto end
		jmp *.L57(,%eax,4)	#  .L57[%eax]
\end{lstlisting}
Erklärung der Letzten Instruktion:
$(,\%eax,4)$ wird übersetzt in $(0+ \%eax)\cdot4$, somit haben wir unseren Offset für die Jump Tabelle. Der Rest des Ausdrucks bedeutet: \"Gehe zur Memory Adresse, die das Label L57 hat, addiere den Offset dazu und springe dan zum Wert, der diese Adresse beinhaltet." Dieser Wert ist dann zum Beispiel die Adresse des Labels .L51.
\section{Bitwise Magix}
\begin{lstlisting}
	int bitXor(int x, int y) {
		return ~x & y;
	}
	
	int isEqual(int x, int y) {
		return !(x ^ y);
	}
\end{lstlisting}

\newpage
\section{Recursion}
Die Rekursion wird anhand des Beispiels Fakultät gezeigt.\\
\subsection{Einfache Rekursion}
\begin{lstlisting}
int rfact(int x)
{
  int rval;
  if (x <= 1) return 1;
  rval = rfact(x - 1);
  return rval * x;
}
\end{lstlisting}
in Assembler:
\begin{lstlisting}
rfact:
    pushl %ebp
    movl %esp, %ebp
    pushl %ebx
    movl 8(%ebp), %ebx 		// ebx = x
    cmpl $1, %ebx		// Compare x : 1
    jle .L78			// If <= goto Term
    leal -1(%ebx), %eax		// eax = x - 1
    pushl %eax			// Push x- 1
    call rfact			// rfact (x - 1)
    imull %ebx, %eax		// rval * x
    jmp .L79			// Goto done
    .align 4
.L78: 				// Term:
    movl $1, %eax		// return val = 1
.L79:				// Done:
    movl -4(%ebp), %ebx
    movl %ebp, %esp
    popl %ebp
    ret
\end{lstlisting}
Linien die mit ``.'' beginnen sind Assembler Ordner
\subsection{Multi-Way recursion}
sinnvolles Beispiel ist Quicksort
\begin{lstlisting}
int r_prod(int from, int to) {
	int middle, prodA, prodB;
	if (from >= to) return from;
	middle = (from + to) >> 1;
	prodA = r_prod(from, middle);
	prodB = r_prod(middle+1, to);
	return prodA * prodB;
}
\end{lstlisting}


\end{document}
