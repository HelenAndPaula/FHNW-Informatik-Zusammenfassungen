\documentclass[10pt]{article}

%Math
\usepackage{amsmath}
\usepackage{amsfonts}
\usepackage{amssymb}
\usepackage{amsthm}
\usepackage{ulem}
\usepackage{stmaryrd} %f\UTF{00FC}r Blitz!
\usepackage{tikz}

\usepackage{multirow}

%PageStyle
\usepackage[ngerman]{babel} % deutsche Silbentrennung
\usepackage[utf8]{inputenc} 
\usepackage{fancyhdr, graphicx}
\usepackage[scaled=0.92]{helvet}
\usepackage{enumitem}
\usepackage{parskip}
\usepackage[a4paper,top=2cm]{geometry}
\setlength{\textwidth}{17cm}
\setlength{\oddsidemargin}{-0.5cm}


%My Commands
\newcommand{\RN}{\mathbb{R}} %Real Number
\newcommand{\NN}{\mathbb{N}} %Natural Number
\newcommand{\QN}{\mathbb{Q}} %Rational Number
\newcommand{\ZN}{\mathbb{Z}} %ganze Zahlen
\newcommand{\CN}{\mathbb{C}}
\newcommand{\Teilt}{\mid} %|
\newcommand{\Teiltn}{\nmid} %kein teiler
\newcommand{\Potp}{\mathcal{P}} %Potenzmenge
\newcommand{\Pota}{\mathcal{A}}
\newcommand{\Potr}{\mathcal{R}}
\newcommand{\Potn}{\mathcal{N}}
\newcommand{\Bold}[1]{\textbf{#1}} %Boldface
\newcommand{\Kursiv}[1]{\textit{#1}} %Italic
\newcommand{\T}[1]{\text{#1}} %Textmode
\newcommand{\Nicht}[1]{\T{\sout{$ #1 $}}} %Streicht Shit durch
\newcommand{\lra}{\leftrightarrow} %Arrows
\newcommand{\ra}{\rightarrow}
\newcommand{\la}{\leftarrow}
\newcommand{\lral}{\longleftrightarrow}
\newcommand{\ral}{\longrightarrow}
\newcommand{\lal}{\longleftarrow}
\newcommand{\Lra}{\Leftrightarrow}
\newcommand{\Ra}{\Rightarrow}
\newcommand{\La}{\Leftarrow}
\newcommand{\Lral}{\Longleftrightarrow}
\newcommand{\Ral}{\Longrightarrow}
\newcommand{\Lal}{\Longleftarrow}
\newcommand{\Vektor}[1]{\vec{#1}}
\newcommand{\Brace}[1]{\left( #1 \right)} %()
\newcommand{\Bracel}[1]{\left\lbrace #1 \right.} %(
\newcommand{\Bracer}[1]{\right. #1 \right\rbrace} %)
\newcommand{\Brack}[1]{\left\lbrace #1 \right\rbrace} %{}
\newcommand{\Brackl}[1]{\left\lbrace #1 \right.} %{
\newcommand{\Brackr}[1]{\right. #1 \right\rbrace} %}
\newcommand{\Result}[1]{\underline{\underline{#1}}} %Doppelt unterstrichen
\newcommand{\Abs}[1]{\left| #1 \right|} %Absolutbetrag
\newcommand{\Norm}[1]{\Abs{\Abs{ #1 }}} %Norm
\newcommand{\Arrays}[1]{\left(\begin{array}{c}#1\end{array}\right)} %Array mit einer Kolonne ()
\newcommand{\Array}[2]{\left(\begin{array}{#1}#2\end{array}\right)} %Array mit n Kolonnen ()
\newcommand{\Bracka}[2]{\left\lbrace\begin{array}{#1}#2\end{array}\right\rbrace} %Array mit {}
\newcommand{\Brackal}[2]{\left\lbrace\begin{array}{#1} #2 \end{array}\right.} %Array mit {
\newcommand{\Brackar}[2]{\left.\begin{array}{#1} #2 \end{array}\right\rbrace} %Array mit }
\newcommand{\Sumone}[2]{\sum_{#2=1}^{#1}} %Summe von 1
\newcommand{\Sumz}[2]{\sum_{#2=0}^{#1}} %Summe von 0
\newcommand{\Sum}[2]{\sum_{#2}^{#1}} %Allgemeine Summe
\newcommand{\Oneover}[1]{\frac{1}{#1}} %1 über igendwas
\newcommand{\Tablewt}[3]{\begin{table*}[h]\caption{#1} \begin{tabular}{#2}{#3}\end{tabular}\end{table*}} %Table mit Titel
\newcommand{\Oben}[2]{\overset{#1}{#2}} %etwas über etwas anderem
\newcommand{\Unten}[2]{\underset{#1}{#2}} %etwas unter etwas anderem
\newcommand{\Bildcap}[2]{\begin{figure}[htb]\centering\includegraphics[width=0.2\textwidth]{#1} \caption{#2}\end{figure}} %Bild mit beschriftung
\newcommand{\Bildjpeg}[1]{\includegraphics[width=0.2\textwidth]{#1.jpeg}} %Bilder jpeg!!
\newcommand{\Bildjpg}[1]{\includegraphics[width=0.2\textwidth]{#1.jpg}} %Bilder jpg!!
\newcommand{\Bild}[1]{\includegraphics[width=0.4\textwidth]{#1}} %Bilder jpg!!
%Beispiel für lstlisting \lstinputlisting[label=Aufgabe 4a,caption=Aufgabe 4a]{4a.java}

% Code listenings
\usepackage{color}
\usepackage{xcolor}
\usepackage{listings}
\usepackage{caption}
\DeclareCaptionFont{white}{\color{white}}
\DeclareCaptionFormat{listing}{\colorbox{gray}{\parbox{\textwidth}{#1#2#3}}}
\captionsetup[lstlisting]{format=listing,labelfont=white,textfont=white}
\lstdefinestyle{JavaStyle}{
 language=Java,
 basicstyle=\footnotesize\ttfamily, % Standardschrift
 numbers=left,               % Ort der Zeilennummern
 numberstyle=\tiny,          % Stil der Zeilennummern
 stepnumber=5,              % Abstand zwischen den Zeilennummern
 numbersep=5pt,              % Abstand der Nummern zum Text
 tabsize=2,                  % Groesse von Tabs
 extendedchars=true,         %
 breaklines=true,            % Zeilen werden Umgebrochen
 frame=b,         
 %commentstyle=\itshape\color{LightLime}, Was isch das? O_o
 %keywordstyle=\bfseries\color{DarkPurple}, und das O_o
 basicstyle=\footnotesize\ttfamily,
 stringstyle=\color[RGB]{42,0,255}\ttfamily, % Farbe der String
 keywordstyle=\color[RGB]{127,0,85}\ttfamily, % Farbe der Keywords
 commentstyle=\color[RGB]{63,127,95}\ttfamily, % Farbe des Kommentars
 showspaces=false,           % Leerzeichen anzeigen ?
 showtabs=false,             % Tabs anzeigen ?
 xleftmargin=17pt,
 framexleftmargin=17pt,
 framexrightmargin=5pt,
 framexbottommargin=4pt,
 showstringspaces=false      % Leerzeichen in Strings anzeigen ?        
}

%Config
\renewcommand{\headrulewidth}{0pt}
\setlength{\headheight}{15.2pt}

%Metadata
\title{
	\vspace{5cm}
	Kryptographie - Prüfungsbesprechung
}
\author{Jan Fässler}
\date{4. Semester (FS 2013)}


% hier beginnt das Dokument
\begin{document}
\section{Aufgabe 1}
\subsection{A}
$C: \alpha \ra Perm(S)$ // C injektiv $\ra |K| \leq |Perm(S)|$ \\
$|S| = 2^n$ \\
$|Perm(S)|=(2^n)!$
\subsection{B}
$f:A\ra B$ \\
$f:x\ra f(x)$ \\ \\
$\Brackar{l}{geg.: y \in f(A) \subset B \\ ges.: x \in A : f(x)=y}$ schwierig
\subsection{G}
Der Koinzidenzindex ist Wahrscheinlichkeit wenn ich zwei mal ein Buchstaben aus einem Text herausgreiffe das ich den selben Buchstaben erwische

\section{Aufgabe 2: RSA}
$p=23$, $q=29$ \\
$n= 23*29=667$ \\
$\varphi(n)=23*28=616=2^3*7*11$
\subsection{A}
$e \in \{3, 5, 13\}$ // $ggT(e,\varphi(n))=1$ \\
\subsection{B}
$d*e \equiv 1 mod \varphi(n)$ \\

\begin{tabular}{l l l l l}
	i & q & r & s & t \\
	\hline
	0 & - & 616 & 1 & 0 \\
	1 & 205 & 3 & 0 & 1 \\
	\hline
	& & 1 & 1 & -206
\end{tabular} \\ \\
$-205 * 3 + 1 * 616 = 1$ \\
$d= -205 \equiv 411 mod \varphi(n)$

\section{Aufgabe 3}
\subsection{A}
\begin{tabular}{l | l}
	a & 4 \\
	b & 3 \\
	c & 5 \\
	d & 6
\end{tabular} \\
$IC_T= \frac{4*3+3*2+5*4+6*5}{18*17}=\frac{68}{306}=\frac{2}{8}$
\subsection{B}
\begin{description}
	\item[Schlüssellänge] Text in p Teile teilen von diesen den Koinsidenzindex vergleichen ob er in dem Bereich einer Sprache ist.
	\item[Schlüssel] Wir nehmen die Abschnitte und machen pro Abschnitt eine Statistik und verschiebe dan die Buchstaben bis es einen Sinn ergiebt.
\end{description}

\section{Aufgabe 4: RSA}
(n,e) = (91,7), (n,d)=(91,31) \\
Alice: (n, $e_A$)=(91,5) \\
a) Gesucht: $d_A mit d_A \equiv d_A mod \varphi(n)$ \\ \\
{\color{red} $e*d-1=k*\varphi(n)=216$} // $ggT(d_A,\varphi(n))=1$\\
{\color{red} $h:=e*d-1$}, // $g=ggT(e*d-1,e_A)=1$\\
\\
{\color{blue} 
Es könnte sein: \\
$e*d-1=k*\varphi(n)= 2 * 5^2*13*\varphi(n)=h$ \\
$h:= \frac{h}{g}$, $g:=ggT(h,e_A)=5$ \\
} \\
Zu lösen: $d_A*e+f*h=1$ \\
$d_A=-43 \equiv 173$ mod h (-43 + 216= 173) \\
\\
b) Gesucht: c=3 entschlüsseln \\
$c^{d_A} \equiv m$ mod n \\
$3^173$ mod 91 (173=128+32+8+4+1) \\
$3^2 \equiv 9$ mod 91 \\
$2^4 \equiv (3^2)^2 \equiv 81$ mod $91$ \\
$3^8= (3^4)^2 \equiv 81^2=(-10)^2=10^2 \equiv 9$ mod 91\\
$3^16 \equiv 81$ mod 91 \\
$3^32 \equiv 9$ mod 91 \\
$3^64 \equiv 81$ mod 91 \\
$3^128 \equiv 9$ mod 91

\section{Aufgabe 5}
m= $\underbrace{0111\hspace{0.25cm} 0100}_{m_1}$ $\underbrace{001{\color{red}1\hspace{0.25cm}0000}}_{m_2}$ \\
\\

\begin{tabular}{l l l}
	$IV \oplus m_1$ & 0000 & 1001 \\
	& 	 & 0011 \\
	& 	 & 0101 \\
	& 	 & 0000 \\
	$runde_1$ & 1001 & 0110 \\
	& 	 & 1001 \\
	& 	 & 1100 \\
	& 	 & 1001 \\
	$c_1$= & 0110 & 1100 \\
\end{tabular}

\begin{tabular}{l l l}
	$c_1 \oplus m_2 $ & 0101 & 1100 \\
	& 	 & 1001 \\
	& 	 & 0101 \\
	& 	 & 0101 \\
	$runde_1$ & 1100 & 1001 \\
	& 	 & 0110 \\
	& 	 & 1100 \\
	& 	 & 1100 \\
	$c_2$ =  & 1001 & 0110 \\
\end{tabular}

\section{Aufgabe 6}
Fixes kleines e $\Ra$ d gross \\
$n=p*Q = 19*31=589$ \\
$\varphi(n)=540$ \\
$e=7$ \\
$d=463$ \\
\subsection{A}
[$log_2(e)$] (abgerundet) + EB-1 \hspace{2cm} mit EB=\#Einsen in binärer Darstellung von e. \\
$e=7 \Ra$ \#Mult. = 2+2 \\
$7=(111)_2$
\subsection{B}
$d=463 > 2^8=256 \Ra [log_2(463)]=8$ \\
$463=256+128+64+8+4+2+1=\underbrace{(111001111)_2}_{EB=7}$ \\
\#Mult= 8+EB-1=8+8=14
\subsection{D}
$d_p \equiv 13$ mod p \\
$d_p \equiv 13$ mod q \\
\\
$m_q \equiv c^{d_q}$ mod q \\
$m_p \equiv c^{d_p}$ mod $p \equiv 3$ mod 19 \\
\end{document} 