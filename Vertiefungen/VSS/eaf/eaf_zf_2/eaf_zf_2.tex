\documentclass[a4paper,10pt]{scrreprt}
\usepackage[top=2cm,bottom=2cm,left=2cm,right=2cm]{geometry}
\usepackage[utf8]{inputenc}
\usepackage{graphicx}
\usepackage[german]{babel}
\usepackage{pdfpages}
%opening
\title{Enterprise Application Frameworks Teil 2 - MSP}
\author{Roland Hediger}
\usepackage{fancyhdr}
\renewcommand{\familydefault}{\sfdefault}
\newcommand{\pic}[2][figure]{\begin{figure}[h]
 \centering
 \includegraphics[scale=0.3]{#2}
 % rsc.png: 0x0 pixel, 0dpi, 0.00x0.00 cm, bb=
 \caption{#1}
\end{figure}
}
\usepackage{framed}
% Code listenings
\usepackage{color}
\usepackage{xcolor}
\usepackage{listings}
\usepackage{caption}
\usepackage[T1]{fontenc}
\DeclareCaptionFont{white}{\color{white}}
\DeclareCaptionFormat{listing}{\colorbox{gray}{\parbox{\textwidth}{#1#2#3}}}
\captionsetup[lstlisting]{format=listing,labelfont=white,textfont=white}
\lstset{
 language=Java,
 basicstyle=\footnotesize\ttfamily, % Standardschrift
 numbers=left,               % Ort der Zeilennummern
 numberstyle=\tiny,          % Stil der Zeilennummern
 stepnumber=5,              % Abstand zwischen den Zeilennummern
 numbersep=5pt,              % Abstand der Nummern zum Text
 tabsize=2,                  % Groesse von Tabs
 extendedchars=true,         %
 breaklines=true,            % Zeilen werden Umgebrochen
 frame=b,         
 %commentstyle=\itshape\color{LightLime}, Was isch das? O_o
 %keywordstyle=\bfseries\color{DarkPurple}, und das O_o
 basicstyle=\footnotesize\ttfamily,
 stringstyle=\color[RGB]{42,0,255}\ttfamily, % Farbe der String
 keywordstyle=\color[RGB]{127,0,85}\ttfamily, % Farbe der Keywords
 commentstyle=\color[RGB]{63,127,95}\ttfamily, % Farbe des Kommentars
 showspaces=false,           % Leerzeichen anzeigen ?
 showtabs=false,             % Tabs anzeigen ?
 xleftmargin=17pt,
 framexleftmargin=17pt,
 framexrightmargin=5pt,
 framexbottommargin=4pt,
 showstringspaces=false      % Leerzeichen in Strings anzeigen ?        
}

\begin{document}

\maketitle
\tableofcontents
\newpage
 \pagestyle{fancy}
\part{Theorie}

\chapter{Einführung  Enterpise Beans}

\begin{description}
 \item [Definition] Ein Enterprise Bean ist eine serverseitige Komponente die die Businesslogik einer Applikation Kapselt ( Facade).
 \begin{itemize}
  \item Beitet eine externe interface an (Remote)
  \item Versteckt komplexität der Implementation.
 \end{itemize}
\item [Zusätzliche Services] Transaktionsmanagement, Sicherheitsdienste, Concurrency Management.
\end{description}

\section{EJB Component Architecture}
\pic{ejbca.png}

\subsection{Synchrone Enterprise Beans}
\begin{description}
 \item [Stateless Session Bean] Bietet Dienste an, kann ein Webservice Implementieren.Ahnlich wie Spring Beans mit \texttt{scope=Singleton}
 \item [Stateful Session Bean] Instanzvariablen representieren eine einzigartige Client-Bean Session. Ähnlich wie Spring Beans mit \texttt{scope=Prototype}.
\end{description}

\subsection{Asynchrone Enterprise Beans}
\begin{description}
 \item [Message Driven Beans] Haltet die Daten oder der Zustand eines spezifischen Clients nicht. Kann auf JMS basiert sein.
\end{description}

\section{Implementation von Enterprise Beans}
\begin{description}
 \item [Enterprise Bean] Primäre Artifakt. Annotiertes POJO. Die Annotation geben die Semantik sowohl als die Vorraussetzungen an dem EJB Container.
 \item [Enterprise Business Interface] Gewöhnliche Interface. Ein EJB\footnote{Enterprise Java Bean} kann mehrere ``Business Interfaces'' implementieren.\\
 \begin{description}
  \item [Remote Interfaces] \texttt{@Remote} : Entfernte Client kann auf einer anderen JVM laufen. Ort der EJB ist für Client Transparent.
  \item [Local Interfaces] Die müssen \textit{auf der gleichen JVM wie EJB laufen.} Ort ist für den Client nicht Transparent.
  \item [Unterschiede] Performanz, Isolation von Paramter. 
 \end{description}
\item [JNDI Access] \hfill \\
\begin{lstlisting}[caption=JNDI]
 Properties props = new Properties();
props.put(Context.INITIAL_CONTEXT_FACTORY,
"org.jboss.naming.remote.client.InitialContextFactory");
props.put(Context.PROVIDER_URL, "remote://localhost:4447");
props.put(Context.SECURITY_PRINCIPAL, "testuser");
props.put(Context.SECURITY_CREDENTIALS, "testpassword");
context = new InitialContext(props);
context.lookup(...)
\end{lstlisting}

\end{description}

\section{Stateless Session Beans}
\begin{description}
 \item [Eigenschaften.] Bietet ein Dienst an für Clients. Verwaltet keine ``Conversational State'' mit Clients. Ein Stateless Bean kann \textbf{Instanzvariablen beinhalten, aber die sind nur für den Dauer des Aufrufs gültig.} EJB Container kann eine \textbf{beliebige Instanz} der Bean an dem Client geben bei einem Methodenaufruf. Client-to-Bean Assoziation ist daher kurz. Es kann jedoch \textbf{globale Zustandsinformation} beinhalten wie ein JNDI Context oder JDBC Verbindung. Diese Instanz ist nicht geteilt - 1 Client pro Instanz.
\end{description}
\begin{lstlisting}[caption=Calculator Example Stateless Bean]
public interface Calculator {
  double sum (double x, double y);
  double difference (double x, double y);
  double product (double x, double y);
  double quotient (double x, double y);
}
// No interface inheritance - no remote xceptions!

// Implementation

@Stateless
@Remote(value = {Calculator,class})
public class CalculatorBean implements Calculator {
  ...//selbsterklärend normale implementation.
}

// Client
public class Client {
public static void main(String[] args throws Exception{
 InitialContext ctx = new InitialContext();
Calculator calculator = (Calculator) ctx.lookup(
"calculator/CalculatorBean!calculator.bean.Calculator");
System.out.println("1 + 1 = "+calculator.add(1, 1));
System.out.println("1 - 1 = "+calculator.subtract(1, 1));
}
}
\end{lstlisting}
\begin{description}
 \item [Remarks] Die Stateless Annotation sagt dem EJB Container dass der CalculatorBean Stateless ist. Es hat folgende Eingabeeigenschaften : 
 \begin{itemize}
  \item Name : Names des Beans default : Unqualified Class Name
  \item Description : Muss in tools angezeigt werden - Beschreibung der Bean.
  \item mappedName : mapping to a JNDI Name. 
 \end{itemize}
 \item[@Remote] Ist dann business Interface. Value ist dann spezifiziert falls die Annotation auf eine Klasse angewendet ist. Interface nicht unbedingt durch einer Bean Klasse implementiert. Variante : Interface selbst annotieren.
 \item[@Local] Ist dan ein Local Interface. Funktioniet nicht falls ein Lokale Client auf eine andere VM ist. Mögliche Anwendung - Web Container. \textbf{Das gleiche Business Interface kann nicht gleichzeitig Lokal und Remote sein.}
\end{description}

\begin{framed}
 Bean Implementation muss ein Defaultkonstruktor haben. 
\end{framed}

\section{Statelss Bean Lifecycle}
\pic{slsbl.png}

\section{Callback Methods for Stateless Session Beans}
\begin{description}
 \item[@PostConstruct] Nur einmal auf jeden Instanz nachdem Dependency Injections fertig sind. Aufgerufen bevor Ausführung der ersten Methode. Nur eine methode kann mit dieser Annotation markiert sein.
 \item[@PreDestroy] Aufgerufen wenn der Beaninstanz zerstört ist. Sollte alle Resourcen schliessen, ähnlich wie finalize. Nur eine methode kann mit dieser Annotation markiert sein.
 \item[Bemerkung] Diese Methoden sind in eine \textit{noch nicht spezifizierte} Transaktions und Sicherheitskontext aufgerufen. Annotierte methoden sollen folgende Kritieren erfüllen: 
 \begin{itemize}
  \item Keine Parameter, return type void. Sichtbarkeit ist irrelevant.
  \item Muss keine CheckedExceptions werfen.
 \end{itemize}

\end{description}

\section{Pooling}
Ein Stateless Bean behandelt die Aufrufe von Clients. Kein Conversational State - Keine Abhängigkeit von 
Client.Teilen von ein Bean zwischen Clients bedeutet weniger Resourcen. \textbf{Beans sind Single Threaded}\\
\textbf{Clustering:} Bean Instanzen können auf mehrere Server sein.

Entfernung der Instanz von einem Stateless Bean ist transparent und von Container behandelt.


\section{EJB Reuse - Dependency Injection}
\begin{lstlisting}[caption=JNDI Lookup DI]
// Service = remote, DAO = local binding
 private MovieDAO movieDAO;
@PostConstruct
private void init(){
try {
InitialContext ctx = new InitialContext();
movieDAO = (MovieDAO) ctx.lookup("...MovieDAO");
}
catch(NamingException e){
throw new RuntimeException(e);
}
}
\end{lstlisting}

\begin{lstlisting}[caption=Dependency Injection with Annotation]
 @EJB(name="...MovieDAO")
private MovieDAO movieDAO;
\end{lstlisting}
Parameters: 
\begin{description}
 \item [Name] JNDI Name von EJB wenn nichts angegeben ist, ist es aus dem Feld oder property genommen 
(Containerspezifisch)
\item[beanName] Name die gegeben ist in Stateful oder Stateless Annotation
\item[mappedName] nichts
\item [description] nichts
\item [beanInterface] Object.class??
\end{description}

\subsection{Persistence Context}
\begin{lstlisting}
 @PersistenceContext(unitName="movie-db")
private EntityManager em;
\end{lstlisting}
Parameters:
\begin{description}
 \item [Name] Entity Manager Name nicht von DI benutzt.
 \item [UnitName] Name der Persistence Unit aus persistence.xml
 \item [type] Transaction/extended.
 \item [Properties] key value pair.
\end{description}

\section{Contexts und Dependency Injection (CDI)}

\begin{lstlisting}
// @Inject
// Field injection
@Inject
private T t;
//Method injection
private T t;
@Inject
public void setT(T t) { this.t = t; }
// Constructor injection
private T t;
@Inject
public C(T t) { this.t = t; }
\end{lstlisting}

\begin{description}
 \item [@Produces] Ein Bean kann mit CDI benutzt werden falls es ein Defaultkonstruktor hat oder ein Konstruktor der 
mit Inject annotiert ist.
Produces erlaubt die benutzung von beliebigen klassen (Factory Methods)

\begin{lstlisting}
 public class Resources {
@Produces
public Logger produceLog(InjectionPoint
produceLog() {
p) {
Logger.getLogger("MovieRental");
return Logger.getLogger(
}
p.getMember().getDeclaringClass().getName());
}
}
//Default Scope ist DependantScope, neue Instanz per Injection.
\end{lstlisting}

\item[CDI Extras] \hfill \\
\begin{lstlisting}[caption=CDI mit Qualifier]
 @Inject @MovieService
Private Service movieService;
@MovieService
public class MovieServiceImp implements Service { ... }

//They are used on the implementation classes and on the injection points

\end{lstlisting}

\chapter{Stateful Session Beans (SFSB)}
\begin{itemize}
 \item Behalten Zustandinformation zwischen Methodenaufrufe.
 \item Bean per Client
 \item State initializiert durch Client nach Erzeugung.
 \item Client muss siene Instanz entfernen oder entlassen.
 \subitem Container entfernt Bean nach Timeout falls kein Client es nutzt.
 \subitem Timeout kann per Annotation definiert werden: \\
 \begin{verbatim}
  @StatefulTimeout(value=5, unit=TimeUnit.SECONDS)
 \end{verbatim}
\item \textbf{Conversational State nicht im DB obwohl es sollche Daten zugreifen kann und updaten kann \footnote{wtf}}
\item \textbf{Stateful Bean kann ``Transaction Aware'' sein}
\end{itemize}

\end{description}

\section{SFSB Examples}
\begin{lstlisting}
 @Remote
public interface Cart {
void
initialize(String owner);
void add
void (double amount);
double remove(double amount);
void getTotal();
    addTax();
void close();
}
@Stateful
public class CartBean implements Cart {
private String owner;
private double total;
private boolean taxAdded = false;
public void initialize(String owner) {
if(owner == null) throw new IllegalArgumentException();
this.owner = owner;
}
public double getTotal() {
return total;
}
@Remove public void close() {}public void add (double amount) {
if(owner==null || taxAdded)
throw new IllegalStateException();
total += amount;
}
public void remove(double amount) {
if(owner==null || taxAdded)
throw new IllegalStateException();
total -= amount;
}
public void addTax() {
if(owner==null || taxAdded)
throw new IllegalStateException();
total = total + total * 8.0 / 100;
taxAdded = true;
}
}

public class Client {
public static void main(String[] args) throws Exception
String JNDI_NAME = "cart/CartBean!ch.fhnw.....Cart";
InitialContext ctx = new InitialContext();
Cart cart1 = (Cart) ctx.lookup(JNDI_NAME);
Cart cart2 = (Cart) ctx.lookup(JNDI_NAME);
cart1.initialize("User1");
cart2.initialize("User2");
cart1.add(100);
cart2.add(50);
System.out.println("Cart1: "+cart1.getTotal());
System.out.println("Cart2: "+cart2.getTotal());
cart1.close();
cart2.close();
}
}



\end{lstlisting}


\section{SFSB Passivation und Activation}
Schwierig zu übersetzen.
\begin{description}
 \item [Conversational State] The conversational state of a stateful session object is defined as the
session bean instance field values, plus the transitive closure of the
objects from the instance fields reached by following java object
references.
\item [Passivation/Activation] \begin{itemize}
                                \item To efficiently manage the size of its [memory], a session bean container
may need to temporarily transfer the state of an idle stateful session bean
instance to some form of secondary storage.
\item Passivation: Transfer from the working set to secondary storage
\item Activation: Transfer back
 \item Typically, the EJB container uses a LRU algorithm to select a bean for
passivation
                               \end{itemize}

\end{description}

\section{SFSB Serializierung}
Was kann Serializiert werden?
\begin{itemize}
 \item Primitives (byte / short / int / long / boolean / float / double)
\item A reference to a Serializable object
\item A null reference
\item A reference to an EJBs business interface
\item A reference to a naming context
\item A reference to the UserTransaction interface
\item A reference to a resource manager
\item A reference to a javax.ejb.Timer object
\end{itemize}

\subsection{passivation Ausschalten}
\begin{verbatim}
 @Stateful(passivationCapable=false)
\end{verbatim}

\section{SFSB Lifecycle}
\pic{sfsbl.png}

\subsection{Callback Methoden}
\begin{description}
 \item [@PostConstruct] Nach konstruktion nach DependencyInjections, bevor erste Methode ausgeführt ist.
 \item [@PreDestroy] Nachdem alle Methoden die mit @Remote annotiert worden sind, fertig sind oder nach Timeout von 30 
min falls Timeout aktiviert.
\item [@PrePassivate] Bevor Container Instanz ``passiviert''. Assert : Nicht Transiente Felder Serializierbar sind. 
Beispiel JDBC Conn geschlossen und auf null gesetzt.
\item[@PostActivate] Nachdem ein Instanz ``reactivated'' ist. Felder die im Prepassivate null sind mussen neu gesetzt 
werden.

\end{description}

\section{JBOSS Spezifisch : Limited Cache}
\begin{lstlisting}[caption=Limited Cache,language=xml]
 caches>
<cache name="simple" aliases="NoPassivationCache"/>
<cache name="small" passivation-store-ref="small"/>
</caches>
<passivation-stores>
<file-passivation-store name="small"
idle-timeout="10" idle-timeout-unit="SECONDS"
max-size="2"/>
</passivation-stores>
\end{lstlisting}
\begin{itemize}
 \item Falls max-size Beans aktiv sind, kann der Container mit  Passivation anfangen.
 \item Fals ein Bean mehr als idle timeout inaktiv ist, wird er passiviert.
 \subitem Instanzen von passivierte Beans werden zerstört.
  \subitem Activationinstanzen erzeugt mit new, PostConstruct nicht benutzt hier.
  \subitem Zusätzliche init mit @PostActivate.
\end{itemize}

\chapter{Transaktionen}
\begin{description}
 \item [Rollback] Transaktion mit Rollback, dann sind die Veränderungen mit SQL nicht commited. Peristenzkontext wird 
entfernt, Detached Objekten können ungültige Daten beinhalten aber ist unproblematisch weil sobald die Entitäten neu 
geladen sind für eine Transaktion, werden die dann die richtigen Daten haben.
\begin{itemize}
 \item Stateful Session Beans sind \textbf{keine Transaktionelle Resourcen} Veränderte Instanzvariablen innerhalb von 
eine Stateful Bean Sessuin werden nicht wiederhergestellt bei Rollback.
\item Stateful Session beans mit CMT können das SessionSyncronization Interface implementieren.
\end{itemize}

\section{SessionSyncronization Callback Methoden}
\begin{description}
 \item [void afterBegin()] Transaktion gerade gestartet.
 \item [void beforeCompletion()] \hfill \\
 \begin{itemize}
  \item Cached database Updates sollte geschrieben werden.
  \item Letzte Chance für Rollback.
  \item Nur aufgerufen falls Transaktion noch nicht für Rollback gesetzt ist.
  
 \end{itemize}
\item [afterCompletion(boolean commited)] Parameter : commited oder rollback? Bean muss möglicherweise seine Zustand 
manuell zurücksetzen.
\end{description}

\section{SFSB die SessionSyncronization implementieren}
\begin{itemize}
 \item Falls ein EnterpriseBean diese Interface implementiert können nur die folgende Werte für Transaktionen benutzt 
werden : 
 \subitem Required
 \subitem Requires\_new
 \subitem Mandatory
 \item Diese Einshränkung versichert dass ein Bean immer aufgerufen ist in eine Transaktion sonnst, werden die 
Callbackmethoden nutzlos.
\end{itemize}

\end{description}

\section{Extended Persistence Context}
\begin{description}
 \item [Persistence Context] Verwaltet eine Menge von EntityInstanzen. Durch Entitymanager zugreifbar.\\
 \begin{lstlisting}
//Applikation verwaltet manager.
 @PersistenceUnit
EntityManagerFactory emf;
emf.createEntityManager();

//Container verwaltet Manager
@PersistenceContext
EntityManager em;

 \end{lstlisting}
\item[Transaction Scoped Entity Manager] \hfill \\
\begin{lstlisting}
 @PersistenceContext(unitName="movieDS",
type=PersistenceContextType.TRANSACTION)
EntityManager em;
\end{lstlisting}
\begin{itemize}
 \item Entitymanager zustandslos.
 \item Abhängig von JTA transactions
 \item PersistenceContext immer mit laufende Transaktion verbunden.
 \subitem Wenn eine Operation ausgeführt ist, die PersistenceContext mit dem der Transaktion verbunden ist, wird 
benutzt.ODER eine neue ist erzeugt und mit Transaktion verbunden.
\item Konsequenzen :: Entitymanger muss innerhalb Transaktion zugreifbar sein. Referenz könnte immer das gleiche sein.
\end{itemize}
\item [SFSB Scoped (Extended Entity Manager) ] \hfill \\
\begin{lstlisting}
 @PersistenceContext(unitName="movieDS",
type=PersistenceContextType.EXTENDED)
EntityManager em;

\end{lstlisting}
\begin{itemize}
 \item Persistence Context:
  \subitem Erzeugt wenn SFSB Instanziert wird.
  \subitem entfernt wenn SFSB entfernt wird.
  \item Konsequenzen:
  \subitem Entities können als ``Conversational State'' gespreichert werden.
  \subitem Veränderungen in Transaktionen sind in der Datenbank gespeichert.
  \subitem Keine Probleme mit detached Instanzen (alles ist immer attached)
\end{itemize}

\end{description}

\section{Extended Entity Manager Beispiel}
\begin{lstlisting}
 @Stateful
public class UserManagerImpl implements UserManager {
@PersistenceContext(unitName="movieDS")
// Transaction scoped
EntityManager em;
User user;
public void init(int id) { user = em.find(User.class, id); }
public void rentMovie(int mvoieId) {
rentMovie(user, em.find(Movie.class, movieId));
}
public void setName(String name) {
user.setName(name); // no effect as detached instance is
}
// changed => em.merge is necessary!
@Remove
public void finished() {}
}

@Stateful
public class UserManagerImpl implements UserManager {
@PersistenceContext(unitName="movieDS",
type=PersistenceContextType.EXTENDED)
EntityManager em;
User user;
public void init(int id) { user = em.find(User.class, id); }
public void rentMovie(int mvoieId) {
rentMovie(user, em.find(Movie.class, movieId));
}
public void setName(String name) {
user.setName(name); // saved after commit
}
@Remove
public void finished() {}
}
\end{lstlisting}

\section{Fazit}
\pic{fz.png}

\part{Code,Übungen,usw.}

\end{document}