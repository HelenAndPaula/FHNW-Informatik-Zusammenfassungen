\documentclass[a4paper,10pt,titlepage=false]{scrreprt}
\usepackage[top=2cm,bottom=2cm,left=2cm,right=2cm]{geometry}
\usepackage[utf8]{inputenc}
\usepackage{graphicx}
\usepackage[german]{babel}
\usepackage{pdfpages}
\usepackage{amsmath}

%opening
\title{Mobile und Verteilte Datenbanksysteme}
\author{Roland Hediger, Egemen Kaba}
\date{\today}
\usepackage{fancyhdr}
\renewcommand{\familydefault}{\sfdefault}
\usepackage{amssymb}
\newcommand{\pic}[2][figure]{\begin{figure}[h]
 \centering
 \includegraphics[scale=0.3]{#2}
 % rsc.png: 0x0 pixel, 0dpi, 0.00x0.00 cm, bb=
 \caption{#1}
\end{figure}
}
\newcommand{\bigpic}[3][figure]{\begin{figure}[h]
 \centering
 \includegraphics[width=#3]{#2}
 % rsc.png: 0x0 pixel, 0dpi, 0.00x0.00 cm, bb=
 \caption{#1}
\end{figure}
}
\usepackage{framed}
% Code listenings
\usepackage{color}
\usepackage{xcolor}
\usepackage{listings}
\usepackage{caption}
\usepackage[T1]{fontenc}
\DeclareCaptionFont{white}{\color{black}}
\DeclareCaptionFormat{listing}{\colorbox{white}{\parbox{\textwidth}{#1#2#3}}}
\captionsetup[lstlisting]{format=listing,labelfont=white,textfont=white,font=bf}
\lstdefinestyle{sqlNoTitle}{
   language=SQL,
   basicstyle=\footnotesize\ttfamily, % Standardschrift
   backgroundcolor=\color[RGB]{230,230,230}, % Hintergrundfarbe
   numbers=left,               % Ort der Zeilennummern
   numberstyle=\tiny,          % Stil der Zeilennummern
   stepnumber=1,               % Abstand zwischen den Zeilennummern
   numbersep=5pt,              % Abstand der Nummern zum Text
   tabsize=2,                  % Groesse von Tabs
   extendedchars=true,         % Erlaubt non-ascii Charakter, aber nicht mit UTF8
   literate={ä}{{\"a}}1 {ü}{{\"u}}1 {ö}{{\"o}}1,
   breaklines=true,            % Zeilen werden Umgebrochen
   frame=trbl,                 % Rahmen
   stringstyle=\color[RGB]{42,0,255} \ttfamily, % Farbe der String
   keywordstyle=\color[RGB]{127,0,85} \ttfamily, % Farbe der Keywords
   commentstyle=\color[RGB]{63,127,95} \ttfamily, % Farbe des Kommentars
   showspaces=false,           % Leerzeichen anzeigen ?
   showtabs=false,             % Tabs anzeigen ?
   xleftmargin=17pt,
   framexleftmargin=17pt,
   framexrightmargin=5pt,
   framexbottommargin=5pt,
   framextopmargin=5pt,
   showstringspaces=false,     % Leerzeichen in Strings anzeigen ?
}


\lstdefinestyle{sql}{
   style=sqlNoTitle,
   title=SQL-Skript
}

\lstdefinestyle{queryResult}{
  basicstyle=\footnotesize\ttfamily, % Standardschrift
  backgroundcolor=\color[RGB]{240,255,240}, % Hintergrundfarbe
  frame=trbl,                 % Rahmen
  literate={ä}{{\"a}}1 {ü}{{\"u}}1 {ö}{{\"o}}1,
}

\lstdefinestyle{queryResultSmall}{
  style=queryResult,
  basicstyle=\scriptsize\ttfamily, % Standardschrift
}
\setuptoc{toc}{leveldown}

\begin{document}
 \maketitle
 \tableofcontents
\pagestyle{fancy}
\part{Theorie} % (fold)
\label{prt:theorie_}
\chapter{Einführung}
\begin{description}
 \item [Was wird Verteilt] \hfill \\
 \begin{itemize}
  \item Aufbaulogik, Verarbeitungselemente 
  \item Funktion
  \item Daten
  \item Steuerung
 \end{itemize}
\end{description}
\section{Fundamental Definitionen}
\begin{description}
 \item [Fundamental Principle] To the user, a distributed system should look exactly like a nondistributed
system.
\item[Data Processing and Mobility] It has become a common appproach to turn any location and situation a job
of ce.
\item [Warum Parallel] 	Performance
Performance
And more Performance.
\item[Big Data] Volume,Variety,Velocity.
\end{description}

\section{Verteilte Datenbank Definitionen}
\begin{description}
 \item [Verteilte Datenbank - DDB] ine verteilte Datenbank ist eine Sammlung mehrerer,
untereinander logisch zusammengehöriger Datenbanken,
die über ein Computernetzwerk verteilt sind.
\item[Verteiltes Datenbankverwaltungssystem (D-DBMS)] Ein verteiltes Datenbankverwaltungssystem ist die
Software, die die verteilte Datenbank verwaltet und
gegenüber den Nutzern einen transparenten Zugang
erbringt.
\item[Verteiltes Datenbank System]DDBS = DBS + D-DBMS
\end{description}

\subsection{Klassifizierung}
\begin{description}
 \item [Heterogenität] Hardware, Netzwerkprotokolle,
Datenverwaltung, Datenmodell, Abfragesprache,
Transaktionsverwaltung
2 Ausprägungen: homogen, heterogen.
\item[Verteilung]Verteilung: betrifft die Verteilung der Daten
2 Ausprägungen: verteilt, zentral
\item[Autonomie] Autonomie: betrifft die Verteilung der Steuerung
3 Ausprägungen: stark integriert, halbautonom,
isoliert
\end{description}
\section{Dates 12 Regeln}
\begin{enumerate}
\item Lokale Autonomie
\item  Unabhängigkeit von zentralen Systemfunktionen
\item Hohe Verfügbarkeit
\item Ortstransparenz
\item  Fragmentierungstransparenz
\item  Replikationstransparenz
\item  Verteilte Anfragebearbeitung
\item Verteilte Transaktionsverarbeitung
\item  Hardware Unabhängigkeit
\item Betriebssystem Unabhängigkeit
\item Netzwerkunabhängigkeit
\item Datenbanksystem Unabhängigkeit
\end{enumerate}
\section{Aspekte Verteilte Datenbanksysteme}
\begin{description}
 \item [Verteilter Datenbankentwurf]\hfill \\ \begin{itemize}
                                      \item wie die Datenbank verteilen
\item Verteilung der DB mit Replikaten oder ohne
\item Verzeichnisverwaltung
                                     \end{itemize}
\item[Anfragebearbeitung] \hfill \\ \begin{itemize}
                           \item Zerlegung der Anfragen in ausführbare Instruktionen
\item Anfrageoptimierung
\item berücksichtigen von Verarbeitungs- und
Datentransferkosten
                          \end{itemize}
\item[Nebenläufigkeit] \hfill \\ \begin{itemize}
                        \item Synchronisation konkurrierender Transaktionen
\item Konsistenz und Isolation
\item Deadlock Erkennung
                       \end{itemize}
\item[Zuverlässigkeit] \hfill \\\begin{itemize}
                        \item Robustheit gegenüber Fehler
\item Atomarität und Dauerhaftigkeit
                       \end{itemize}


\end{description}

\section{Paralelle Datenbanksysteme}
Parallele DB Systeme kombinieren
Datenbankverwaltung und
Parallel Verarbeitung zur
Verbesserung der Performance und der Verfügbarkeit.
\subsection{Architekturen}
\begin{itemize}
 \item Shared Memory Architecture
\item Shared Disk Architecture
\item Shared Nothing Architecture
\end{itemize}
\subsection{Aspekte}
\begin{itemize}
 \item Parallele Anfrageverarbeitung
\item Daten Partitionierung
\item Parallelisierung von Operationen
\item Lastausgleich
\item Verfügbarkeit
\end{itemize}
\section{Nosql Systeme}
 \begin{itemize}
  \item Big Data
\item Performance vs Scalability
\item Latency vs Throughput
\item Availability vs Consistency
 \end{itemize}
\subsection{Konzepte}
\begin{itemize}
 \item CAP Theorem \footnote{Consistency, Availability , Partition Tolerance, will never achieve all three}
\item ACID vs BASE\footnote{Basically Available, Soft State, Eventual Consistency}
\item Speicherstrukturen : Keyvalue, Document, Wide Column, Graph DB.
\item Map / Reduce : Verarbeitung
\item Consistent Hashing : Verteilung
\item Multiversion Concurrency Control
\item Paxos\footnote{Protokolle : Verfahren definiert - Griescisches Parlament lausig - wie kann man zu Entscheidung 
kommen}
\end{itemize}

\section{Mobile Datenbank Systeme}
Mobile Datenbank Systeme sind Verteilte Datenbank
Systeme mit zusätzlichen Eigenschaften und
Einschränkungen:
\begin{itemize}
\item beschränkte Ressourcen
\item häufig nicht verbunden
\item verlangt andere Transaktions Modelle
\item verlangt andere Replikationsstrategien
\item Ortsabhängigkeit
\end{itemize}


\section{Semantic Web}
Die aktuelle Web Infrastruktur unterstützt ein verteiltes
Ge echt von Webseiten, die gegenseitig mittels den sog.
Uniform Resource Locators (URL) verknüpft sein können:
\begin{itemize}
 \item Das Semantic Web
unterstützt das Web auf der Ebene der Daten statt nur
auf der Ebene der Darstellung
\item Datenelemente können gegenseitig verknüpft sein,
nicht nur Webseiten
\item Datenelemente können gegenseitig verknüpft sein,
nicht nur Webseiten
\item Information über einzelne Entitäten können verteilt
sein
\item Daten Modell: Resource Description Framework (RDF)
\end{itemize}
$ 3^2 + 4x + 3x_{bla}$
\chapter{Entwurf Verteilte Datenbanken}
\pic[Beispiel DB]{bspdb.png}
Anwendungen auf verschiedenen Knoten benötigen
verschiedene Daten, unterschiedlich häufig:\\
A1: Liste der Aufträge mit günstigen Bikes (auf ORION
in Basel)\\
A2: Änderungen an Bikes (auf CALYPSO in Aarau)\\
A3: Liste der Aufträge mit teuren Bikes vom Typ
Mountain (auf TELESTO in Olten)\\
A4: Liste aller Kunden (auf ANANKE in Zürich)\\
Die Daten sollten dort abgelegt sein, wo sie am häu gsten
gebraucht werden.\\

\section{Einfführung}
\begin{itemize}
\item Entscheid über die Platzierung von Daten auf den
Knoten eines Computernetzwerks
\item beeinflüsst die Performance
der DDB und der Anwendungen
\item lokaler Zugriff ist günstiger
als Zugriff auf entfernte Knoten
\item Analyse:
\subitem welche Anwendungen (Queries)
\subitem auf welchen Knoten
\subitem benötigen welche Daten
\subitem mit welcher Häu gkeit
\item Resultat:
\subitem Menge von Fragmenten (Ausschnitte der Daten)
\subitem zugeteilt auf verschiedene Knote
\end{itemize}

\section{Entwürfe}
\subsection{Top Down}
beim Entwurf from scratch
in homogenen Systemen
nachgelagert an den konzeptionellen Entwurf.
\pic{bup.png}

\begin{enumerate}
\item wozu überhaupt fragmentieren?
\item  wie fragmentieren?
\item  wieviel fragmentieren?
\item  wie prüfen der Korrektheit?
\item  wie Fragmente zuteilen?
\item  welche Informationen werden benötigt?
\end{enumerate}

\subsubsection{Grad der Fragmentierung}
\begin{itemize}
\item Vollständige Relationen sind zu grob, einzelne
Attributswerte sind zu fein.
\subitem geeignete Teile (Fragmente) der Relationen
bestimmen
\subitem Nebenläufikeit steigern bei mehreren Transaktionen
mit Zugriff auf verschiedene Fragmente
\subitem Zusatzkosten für Transaktionen mit Zugriff auf
Fragmente an verschiedenen Knoten
\end{itemize}

\subsubsection{Horizontal}
Zerlegt aufgrund Riehen, mit Where klausel.
%pic
\subsubsection{Projektion - Vertikal}
Spaltenliste. In allen Fragmente PK muss dabei sein damit es rekonstruierbar ist-
%pic
Kann kombiniert weden mit Horizontal für optimale Verteilung.

\subsection{Bottom Up}
\begin{itemize}
 \item Multidatenbank Anwendungen
\item Datenbanken schon auf verschieden Knoten vorhanden
\item Problem der Datenintegration, Schemaintegration
\item Web Services
\end{itemize}

\section{Korrektheit der Fragmentierung}
\begin{description}
 \item [Vollständig] Wenn R zerlegt wird $R1,R2 \\ldots R_n$ dann muss jedes Datenelement aus R in einem $R_i$ 
enthalten sein. (Join mit gemeinsamen Pk - wiederzusammenstellung). (Horizontal umkehroperation - Union)%todo
\item [Rekonstruierbar] Wenn wenn R zerlegt wird in , dann muss es $R1,R2 \\ldots R_n$
relationale Operatoren geben, so dass R wiederhergestellt
werden kann.
\item[Disjunkt] wenn R horizontal zerlegt wird in $R1,R2 \\ldots R_n$, dann
müssen die Fragmente paarweise disjunkt sein.
wenn R vertikal zerlegt wird in $R1,R2 \\ldots R_n$, dann müssen
die Fragmente bezogen auf die nichtprimen Attribute
paarweise disjunkt sein.
\end{description}

\section{Fragmentierungsabkürzungen}
\begin{itemize}
 \item Horizontale Fragmentierung (HF)
\subitem Primäre horizontale Fragmentierung (PHF)
\subitem Abgeleitete horizontale Fragmentierung (DHF) Fragmente gehören zu andere Fragmente wegen PK oder FK. Ähnliche 
Fragmentierung sinnvoll.
\item Vertikale Fragmentierung (VF)
\item Gemischte Fragmentierung (MF)
\end{itemize}

\section{PHF}
\pic{phfbsp.png}

Quantitive Informationen:\\
Bedingungen in Queries bestehen aus:
\begin{description}
 \item [Simple Predicate p] Vergleich eines Attributs mit einem Wert : Name = ``bla'' Preis < 2000. Allgemein 
bestehenbedingungen aus boolschen Kombinierung von simple Predicates : $ P = \{p_1,p_2 \ldots , p_m\}$ vom simple 
predicates bilden wir
\item[Mintem predicate] $M(P)$ \textbf{Verknupfungen aller simple predicates aus P mit AND und NOT}. bname= bla and not 
preis < 2000 z.B \\
Bemerkung : \\
$p_1 : $ bname = `Sienna'\\
$p_2 : $ preis < 2000 \\
$ p_1 \wedge p_2 \\
p_1 \wedge \neg p_2 \\
\neg p_1 \wedge p_2\\
\neg p_1 \wedge \neg p_2$

\end{description}
\pic{mtbsp.png} %todo
\newpage
Quantitive Informationen : Brauche ich das minterm oder nicht?
\begin{description}
 \item [Minterm selectivity] $sel(m_i)$ Anzahl Tuppel die mit dem Minterm $m_i$ augewählt werden.
 \item [Access frequency] $acc(m_i)$ Häufigkeit der Anwendungen auf Daten mit dem
minterm $m_i$ zugreifen
\end{description}

\section{PHF Schritt für Schritt}
\begin{description}
 \item [Vollständigkeit] Eine Menge von simple predicates ist vollständig genau
dann, wenn auf beliebige 2 Tupel im gleichen Fragment
von allen Anwendungen mit der gleichen Häu gkeit
zugegriffen wird
\item[Minimalität] Wird durch ein simple predicate ein Fragment weiter
aufgeteilt, dann muss es mindestens eine Anwendung
geben, die auf diese Fragmente verschieden zugreift
Ein simple predicate soll also relevant sein für die
Bestimmung einer Fragmentierung
Sind alle simple predicate eine Menge P relevant, dann ist
P minimal
\end{description}

\begin{itemize}
 \item Menge von Simple predicates mit der Eigenschaft der Vollständigkeit und Minimalität.
 
\end{itemize}

\section{PHF Beispiel Bikes} %todo in schritten
Zugriffshäufigkeit gegeben je nach art des Fahrrads. \\
Anwendung 2 : Bikes aufgrund preise verwaltet. \\ Predicates aus beobachtungen\\
Anwendung 1:
Zugriffshäufigkeit der Anwendung in excel, schauen. Wenn grosse verschieden  - schlussfolgerung kann unterscheiden 
durch eine WHERE bedingung
Anwendung 2 :
Preis spielt hier eine rolle - ist relevante Bedingungen. Simple predicates aufnehmen die RELEVANT sind.

Verfahren erklären :
Minimale Menge von simple predicates ausarbeiten aufgrund der Beobachtung Anwendungen.
wahle eines aus von allen predicates das relevant ist - typ = mounten. 
erganze meine menge p' um die nächsten nur relevanten \\ 
\\ sehe phf beispiel. \\
p1..p6 = P
p1 ist relevant wege häufigkeit.\\
p2, p3 hinzufügen  relevant \\
p4 nicht relevant weil : typ  =city kann ich ausdrucken im minterm von anderen (Restmenge) p4 ist aber einfacher zu 
benutzen.\\
entweder p5, oder p6 aufnehmen sehe oben. MINIMALE MENGE.
2 HOCH 6 kombinationen. durch impl und wiedersprüche nur 8 sinnvoll. \\
Beispiel : \\
Typ = Road und Typ = Mountain ist ein wiederspruch. Bike für beides geht nicht.\\
Impl beispiel :
Typ = road folgt nicht alle anderen typen. alle extrabedingungen mit nots reduzieren. \\ (Schritt 3)
Fragmente bilden sehe folie.\\
homogen zugriff auf fragment ?? nutzung alle tupel gleich. zugriffhäufigkeit konform pro fragment. \\
wichtigsten anwendung betrachte 20 prozent 80 prozent last.
kleinste sinnvolle einheit. \\

\section{DHF}
\begin{itemize}
 \item ist eine horizontale Fragmentierung, die auf einer
horizontalen Fragmentierung einer übergeordneten
Relation basiert
\item stellt sicher, dass Fragmente verschiedener
Relationen, auf die häu g im Verbund zugegriffen
wird, dem gleichen Knoten zugeteilt werden
\item defnieren mit einem Semijoin (Dreickzeichen) \footnote{Mit subquieries definiert: SELECT * FROM AUFTRÄGE WHERE 
EXISTS (SELECT 1  FROM $KUNDEN_1$ WHERE $KUNDEN_1$.KNR = AUFTRÄGE.KNR) }: Kann als normallen Join geschrieben werden :
Projektion $\pi_{knr,datum,auftrage.knr}$ (Aufträge $\ltimes $ $Kunden_1$) 

\end{itemize}

\pic{dhfbsp.png}


\chapter{Vertikale Fragmentierung}

Geht um ``ähnliche Nutzung''.
\begin{itemize}

\item Vertikale Fragmentierung einer einzelnen Relation
\item Modellieren des Zugriffs auf die Relationen
\subitem Welche Anwendungen (Queries)
\subitem verwenden welche Attribute
\item und werden auf welchen Knoten
\item wie häufig ausgeführt
\item mit dem Ziel zu entscheiden
\subitem welche Attribute werden häu g zusammen verwendet
\subitem und sollen deshalb zum selben Fragment gehöre
\end{itemize}

\pic{dbvf.png}
\section{Zugriffseigenschaften}
\pic{dbze.png}
Die Verwendung (Usage) der Attribute durch diese
Anwendungen (Queries) wird in einer Matrix dargestellt:
Matrix U.
eine 1 bedeutet: Query verwendet Attribut
eine 0 bedeutet: Query verwendet Attribut nicht
\begin{tabular}{c c c c c}
 &\textbf{BName} & \textbf{Preis}  & \textbf{Typ} & \textbf{Bestand}\\
$q_1$ & 1 & 0 & 0 & 1 \\
$q_2$ & 0 & 1 &0 & 1 \\
$q_3$ & 0 & 1 & 1 & 0 \\
$q_4$ & 0 & 0 & 1 & 1 \\
\end{tabular}

Anwendung wird ebenfalls in einer Matrix festgehalten:
Matrix Acc.

\begin{tabular}{c c c c c}
 &\textbf{S1} & \textbf{S2}  & \textbf{S3}\\
$q_1$ & 15 & 20 & 10 \\
$q_2$ & 5 & 0 & 0 \\
$q_3$ & 25 & 25 & 25 \\
$q_4$ 3 & 0 & 0 \\
\end{tabular}

die Zahl 15 oben links besagt z.B.: Query $q_1$ wird auf
Knoten S1 15 Mal pro Tag ausgeführt

\section{Cluster Methode}
\begin{itemize}
\item  Der Zugriff auf jede Attributsmenge durch die
Anwendungen ist sehr unterschiedlich
\subitem Dadurch würde jedes Attribut sein eigenes Fragment
bilden, was wenig sinnvoll ist
\subitem Besser wird nach Attributsmengen gesucht, auf die
ähnlich zugegriffen wird
\item Die Clustering Methode erkennt ähnliche Zugriffsmuster auf
Attribute
\subitem Für jedes Paar von Attributen bestimmen, wie oft auf
sie gemeinsam zugegriffen wird
\subitem Cluster bilden von Attributen mit hoher Af nität
\end{itemize}

\subsection{Äffinität}
\begin{framed}
 \begin{equation}
  aff (A_i,A_j) = \sum_{k : U_{k_i}=1,U_{k_j}=1} \sum_{l=1}^s acc_{k_l} 
 \end{equation}

\end{framed}

\pic{affin.png}
%todo tex extra workflowy.

Matrix AA Zeigt Äffinität zweier Attribute :
\begin{tabular}{c c c c c}
 &\textbf{BName} & \textbf{Preis}  & \textbf{Typ} & \textbf{Bestand}\\
BName & 45 & 0 & 0 & 45 \\
Preis & 0 & 80 &75 & 5 \\
Typ & 0 & 75 & 78 & 3 \\
Bestand & 45 & 5 & 3 & 53 \\
\end{tabular}

\begin{framed}
 Skalarprodukt benachbarter Spalten
misst die Ähnlichkeit des
Zugriffsmusters: Af nität der
Nachbarschaft (bond):
\begin{equation}
 \sum_{z=1}^{n} = aff(A_z,A_x)*aff(A_z,A_y)
\end{equation}
Die Summe über alle Skalarprodukte ist
die globale Af nität der Nachbarschaft

\end{framed}
BName, Preis: 45 x 5 = 225\\
Preis, Typ:\\
80 x 75 + 75 x 78 + 5 x 3 = 11865\\
Typ, Bestand:
75 x 5 + 78 x 3 + 3 x 53 = 768\\

globale Affinität: 12858\\

\textbf{Vertauschen ist auch möglich:}

\subsection{Vertauschen bei Affinität}
Durch Austausch von Spalten (und den entsprechenden Zeilen) in
der Matrix AA verändert sich die globale Af nität der
Nachbarschaft:

\begin{tabular}{c c c c c}
  &\textbf{BName} & \textbf{Preis}  & \textbf{Typ} & \textbf{Bestand}\\
BName & 45 & 0 & 0 & 45 \\
Preis & 0 & 80 &75 & 5 \\
Typ & 0 & 75 & 78 & 3 \\
Bestand & 45 & 5 & 3 & 53 \\
\end{tabular}


Preis, BName: 5 x 45 = 225\\
BName, Typ: 45 x 3 = 135\\
Typ, Bestand:\\
75 x 5 + 78 x 3 + 3 x 53 = 768\\
Globale Affinität: 1128\\

\section{Bond Energie Algorythmus}
\begin{itemize}
\item Gegeben $n \times n$ Matrix AA der Affinitäten
\subitem Beliebige 2 Spalte aus AA wählen und in Resultats Matrix CA stellen
\item Iteration:
\subitem eine der übrigen $n - i$ Spalten so in Resultats Matrix
positionieren ($i + 1$ mögliche Positionen), dass sich der
grösste Beitrag an die globale Affinität der
Nachbarschaft ergibt
\subitem Die Zeilen entsprechend den Spalten anordnen
\item Beitrag einer Spalte $A_k$ wenn zwischen $A_i$ und $A_j$ platziert:\\
$ \text{cont} (A_i , A_k , A_j ) = \text{bond}(A_i , A_k ) + \text{bond}(A_k , A_j ) - bond(A_i , A_j )$
\end{itemize}
\section{VF - Vertical Fragmentation Example}
\pic[Gegebene Matrix der AA Affinitäten]{vfb1.png}
\pic[Aus der Affinitäts Matrix AA werden
die Spalten BName und Typ beliebig
gewählt]{vfb2.png}
Spalte Preis an 1. Position
platzieren und Beitrag an globaler
Affinität berechnen:\\
cont(\_, Preis, BName) =
0 + 225 - 0 = 225\\
Spalte Preis an 2. Position platzieren
und Beitrag an globaler Af nität
berechnen:\\
cont(BName, Preis, Typ) =
225 + 11865 - 135 = 11955\\
Spalte Preis an 3. Position platzieren
und Beitrag an globaler Af nität
berechnen:\\
cont(Typ, Preis, \_) = 11865 + 0 - 0 =
11865  \\

Preis an der 2. Position bietet den grössten Beitrag.

\pic{vfb5.png}

\section{VF : Cluster möglichketen}
  \pic{vfcm.png}
\begin{itemize}
 \item Trennpunkt kann entlang der Diagonalen verschoben
werden
\item Für jede der Möglichkeiten berechnen, wie oft:
\subitem ausschliesslich auf Attribute eines Fragments
zugegriffen wird (gut)
\subitem auf Attribute beider Fragmente zugegriffen wird
(schlecht)
\item Trennqualität $sq = acc(VF_1 ) \times acc(VF_2 ) - acc(VF_1 , VF_2 )^2$
maximieren
\subitem acc(VF 1 ): Summe Zugriffshäu gkeiten der Queries
ausschliesslich auf VF 1
\subitem acc(VF 2 ): Summe Zugriffshäu gkeiten der Queries
ausschliesslich auf VF 2
\subitem acc(VF 1 , VF 2 ): Summe Zugriffshäu gkeiten der Queries
auf VF 1 und VF 2
\end{itemize}
\pic{vfse.png}

\section{Schluss : VF}
Aus der erhaltenen Clusterung der Attribute, ergänzt mit
dem Primärschlüssel ergeben sich die Fragmente:\\
BIKES 1 : $\pi_{\text{BNr, BName, Bestand}}$(BIKES)\\
BIKES 2 : $\pi_{\text{BNr, Preis, Typ}}$ (BIKES)\\

\chapter{Verteilte Anfrage Verarbeitung}

\begin{framed}
 Beispiel Query:
$KUN_1$ = $\pi_{KNR,KNAME}(KUN)$ \\
$KUN_2$ = $\pi_{KNR,ORT}(KUN)$\\
$KUN = KUN_1 \bowtie KUN_2$ \\
\end{framed}

\section{Query Processor}
Umformen der SQl Queries in ausführbare Operationen für den Datenzugriff:

\pic{qproc.png}

\section{Zentralisiert vs Dezentraliziert}
\begin{description}
 \item [Zentraliziertes System] 
 \begin{itemize}
  \item Umformen der SQL Anfrage in ausführbare Einheiten - Operationen der Rationalen Algebra.
  \item Auswählen des besten Ausführungsplans
 \end{itemize}
 \item[verteiltes System] \begin{itemize}
                           \item berücksichtigen der Kommunikationskosten
                           \item auswählen des besten Knotens.
                          \end{itemize}
Relationen werden mit den ersten drei Buchstaben von der DB Tabellenname zufkünfig benannt.
\section{Überblick}
\subsection{Ziel der Optimierung}
\pic{zdo.png}

\begin{itemize}
 \item durch Umformen und Optimieren soll eine Kosenfunktion mimimiert werden
 \subitem CPU Zeit
 \subitem IO Zeit
 \subitem Comm Zeit
 \item verschiedene Gewichte in unterschiedlich verteilten Umgebungen. WAN - Kommzeit LAN - ähnlich gewichtet.
 
\end{itemize}

\subsection{Komplexität}
$ \theta \pi $ mit Duplikate : O(n) \\
$ \pi$ Ohne Duplikate - GROUP O(n log n) \\
$ \bowtie \cup \cap \div$ : O(n log n) \\
$ \times $ : $O(n^2)$ 


\end{description}

\section{Methode}
\subsection{Verteilte Verarbeitung}
\pic{vab}

\subsection{Zerlegung}
Sql Query zerlegen und umformen in relationale Algebra unter verwendung des globalen Schemas :
\begin{itemize}
 \item Normalisierung (Bedingung in Where Klausel)
 \item Analyse um inkorrekte Queries zurückzuwechseln.
 \subitem Analyse nach Typ
 \subitem Analyse nach Semantik
 \item Vereinfachung, Redundanz beseitigen.
 \item Umformen in optimalen Ausdruck der relationalen Algrbra.
 
\end{itemize}

\section{Lokalisierung}
\begin{itemize}
 \item verwenden des Fragmentierunggsschema.
 \item verteilte Anfrage mit globalen Relationen abbilden in Anfragen mit Fragmenten:
 \subitem Ersetzen der globalen Relation mit den Fragmenten :
 \subitem $\cup$ für Horizontale Fragmentierung
 \subitem $\bowtie$ für Vertikale Fragmentierung.
 \item Optimierng der lokalisierten Anfrage dürch Reduktion mit Selektion oder Join.
\end{itemize}

\subsection{PHF}
\pic{phfr.png}
\textbf{Bemerkung:} Eine Selektion auf einem Fragment mit einer Bedingung, die der Bedingung für die Fragmentierung 
wiederspricht , ergibt leere Resultatsrelationen.

\subsection{Schritten für Lokalisierung PHF}
\begin{description}
 \item [Schritt 1] globaler Query Baum erstellen
 \item[Schritt 2] globale Relation mit Fragmenten ersetzen.
 \pic{phflok.png}
 \item[Schritt 3] Reduktion: \\
 \pic{phflok1.png}
 Eine Reduktion ist möglich wenn die Fragmentierung mit dem Join Attribute erfolgte. Wenn sich die Bedingungen der 
Fragmente wiedersprechen, resultiert eine leere Relation. Vor einer möglichen Reduktion wird das Distributivgesetz 
angewendet : \\
$ (R_1 \cup R_2) \bowtie S = (R1 \bowtie S) \cup (R_2 \bowtie S) $
\end{description}

\subsection{Reduktion mit Join für PHF}
\pic{redjphf.png}
\pic{redjphf2.png}
\pic{redjphf3.png}
\pic{redjphf4.png}


\section{Reduktion für VF}
\pic{redvf1.png}
Eine Projektion auf einem Fragment ist nutzlos wenn die Attribute der Projektion nicht im Fragment enthalten sind.
\pic{redvf2.png}
\pic{redvf3.png}
\pic{redvf4.png}
  
\chapter{Concurrency}

\section{Notation + Definitionen für Concurrency}
\begin{tabular}{|c|c|}
\hline 
 \textbf{Notation} & \textbf{Bedeutung} \\ \hline
 $r_i[x]$ & Read(x) durch Transaktion $T_i$ \\ \hline
 $w_i[x]$ & Write(x) durch Transaktion $T_i$ \\ \hline
 $o_i[x]$ & Read(x) oder Write(x) durch $T_i$ \\ \hline
 $c_i$ & Commit durch Transaktion $T_i$ \\ \hline
 $a_i$ & Commit durch Transaktion $T_i$ \\ \hline
 $e_i$ & Commit oder Abort durch $T_i$ \\ \hline
 
\end{tabular}

\begin{description}
 \item [Transaktion] Folge von Read und Write Operationen auf beliebiegen Datenelementen x. Letzte Operation entweder 
Commit oder Abort.
\item[Schedule,History] ist eine foge von Datenbank Scheduler auf der Datenbank ausgeführten Lese und 
Screiboperationen.Es besteht aus Operationen verschiedener Transaktionen.Ordnung der Operationen innerhalb einer 
Transaktion muss beibehalten werden.
\item[Serielle History] Eine History H heisst seriell, wenn jeweils alle Operationen einer Transaktion direkt 
hintereinander ausgeführt werden (keine überlappte Ausführung der Transaktionen).
\item[Konflikt Operation] Zwei Datenbank-Operationen $o_i$ und $o_k$ heissen Konflikt-Operationen wenn :
\begin{itemize}
 \item beide auf dasselbe Element x zugreifen.
 \item sie zu verschieden Transaktionen gehören.
 \item mindestens eine der beiden eine Schreiboperation ist.
\end{itemize}
\item[Konflikt Serialisierbarkeit] Eine History heisst konflikt-serialisierbar , wenn es eine serielle History mit 
gleicher Reihenfolge der Konflikt-Operationen gibt.
\item[Recoverable History] Eine History heisst Recoverable falls alle Commits der Transaktionen, von denen eine 
Transaktion $T_i$ liest vor dem $c_i$ erfolgen.
\item[Vermeiden von cascading Aborts] Eine History vermeidet cascading Aborts falls alle Commits der Transaktionen, von 
denen eine Transaktion $T_i$ liest vor dem $r_i[x]$  erfolgen.
\item[Strickte History] Eine History heisst strikt falls jedes $r_i[x]$ oder $w_i[x]$ einer Transaktion $T_i$ nach den 
Commits aller anderen Transaktionen erfolgt, die auf $x$ geschreiben haben.
\item[2 Phasen Sperrprotokoll] Two Phase Lock : Eine Transaktion folgt dem 2 Phasen Sperrprotokoll
\begin{itemize}
 \item vor einem Read bzw Write muss ein share Lock bzw exclusive lock verlangt und erlangt werden.
 \item die Sperre darf erst freigegeb werden wenn die zugehörige Operation vollständig durchgeführt worden ist.
 \item nach der ersten Freigabe einer Sperre darf keine neue Sperre verlangt werden.
\end{itemize}
\item[Striktes 2 Phasen Sperrprotokoll] 
\begin{itemize}
 \item Alle Sperren müssen auf einmal mit dem Commit atomar freigegeben werden.
\end{itemize}


\end{description}


\section{Einführung}
\begin{description}
\item[Atomicity] Alles oder Nichts
\item[Consistency] keine Verletzung von Integritätsregeln
\item[Isolation] keine Nebeneffekte bei Nebenläu gkeit
\item[Durability] Bestätigte Änderungen sind persistent
\end{description}

Transaktionen bieten :
\begin{itemize}
 \item unteilbare und zuverlässige Ausführung auch bei
Ausfällen
\item korrekte Ausführung auch bei gleichzeitigem Zugriff
durch mehrere Benutzer
\item korrekte Verwaltung von Replikaten (falls unterstützt)

\subsection{Aspekte der Transaktionsverarbeitung}
\begin{itemize}
 \item Transaktionsstruktur (Transaktionsmodell)
\subitem flache Transaktion, verschachtelte Transaktion
\item Konsistenzerhaltung der Datenbank
\item Zuverlässigkeitsprotokolle
\subitem Unteilbarkeit, Dauerhaftigkeit
\subitem lokale Wiederherstellung
\subitem globale Commit Protokolle
\item Nebenläufigkeitskontrolle
\subitem Ausführung nebenläufiger Transaktionen (Korrektheit?)
\subitem Konsistenzerhaltung zwischen Transaktionen, Isolation
\item Kontrolle der Replikate
\subitem Kontrolle der gegenseitigen Konsistenz von Replikaten
\item One-Copy Equivalence, ROWA
\end{itemize}

\section{Arhitektur}
\pic{dbarchcca.png}
\pic{dbarchcc1.png}

\section{Nebenläufigkeit}
Nebenläufigkeitskontrolle wie in zentralisierten DBS
\item Ablaufplan, History, Schedule
\item serielle Ausführung, serielle History
\item Äquivalenz von Histories
\subitem Konflikt Äquivalenz
\subitem Konflikt Operationen
\item Serialisierbarkeit
\item zwei verschiedene Arten Histories berücksichtigen
\subitem lokale Ablaufpläne
\subitem globaler Ablaufplan
\item für Serialisierbarkeit des globalen Ablaufplans sind
zwei Bedingungen nötig
\subitem jeder lokale Ablaufplan muss serialisierbar sein
\subitem zwei Konflikt Operationen müssen in der gleichen
Reihenfolge auftreten in allen lokalen Ablaufplänen, in
denen sie zusammen auftreten
\end{itemize}

\subsection{Beispiel : Global nicht serialisierbar}
\pic{globalns.png}

serialisierbar auf Knoten aber global nicht wegen verscheidene Reihenfolge.Wir wissen im Zentral wie man es macht, aber

\section{Realisierung} % (fold)
\label{sec:realisierung}
\begin{itemize}
  \item Basierend auf 2 Phasen Sperrprotokoll
  \subitem Primary Copy 2PL.
  \subitem Verteiltes 2PL
  \item häufig mit Snapshot verfahren.
  \subitem Read Constency bei Oracle
  \subitem verwenden der SCN Synchronization bei verteilten DBS.\\
  \pic{scn.png}

\end{itemize}
% section realisierung (end)
\subsection{Zentrales 2PL} % (fold)
\label{sub:zentrales_2pl}
\begin{itemize}
  \item einzelner Knoten verwaltet alle Sperrinformation.
  \item ein Lock manager fur das gesmate DDBMS
  \item jede Sperre muss beim zentralen Lock manger verlangt werden.
  \item einfach zu relisieren.
  \item Flachenhäls, weniger zuverlässig.
\end{itemize}
% subsection primary_copy_2pl (end)
\subsection{Primary Copy 2pl} % (fold)
\label{sub:primary_copy_2pl}
\begin{itemize}

\item Lock Manager auf einigen Knoten verteilt
\item jeder Lock Manager verantowrtlich für die Sperren einer Menge von Daten 
\item bei Replikaten wird eine Kopie als Primary Copy bestimmt, die anderen sind dann slave copies.
\item Schreibsperre nur auf Primary Copy.
\item nach Änderung der Primary Copy,Änderungen zu den slaves bringen.
\item komplexeres Deadlock Handling
\item weniger Kimmunikation bessere Performance. 
\end{itemize}
% subsection zentrales_2pl (end)
\subsection{Verteiltes 2PL} % (fold)
\label{sub:verteiltes_2pl}
\begin{itemize}
  \item Jeder Lock manager auf allen Knoten verteilt. 
  \item jeder Lock anager vrantwortlich für die Sperren siner Daten
  \item ohne Replikate gleich wie Primary Copy 2PL
  \item mit Replikaten wird Read One Write All Protokoll implementiert. 
  \subitem irgendiene Kope kann zm Lesen verwendet werden  
  \subitem alle Kopien erhalten Schreibsperre vor Änderung.
  \item \textit{komplexeres Daeadlock Handling}
  \item \textit{höhere Kommunikationskosten als Primary Copy 2PL}

\end{itemize}
% subsection verteiltes_2pl (end)
 \section{Deadlocks} % (fold)
 \label{sec:deadlocks}
 Transaktionen sind im einem Deadlock wenn sie \texttt{blockiert} sind und \textbf{es bleiben bis das System eingreift.}\\
 Deadlocks treten in 2PL auf zur Vermeidung von nicht serialiserbare Aufbaupläne.  Um einen Deadlock aufzulösen muss eine Transaktion abgebrochen werden.
 In einem \textbf{Wait for Graph oder WFG} können Deadlocks erkannt werden.

 WFG ist ein gerichtete Graph und dtellt die warte auf Beziehung zwischen Transaktionen dar. Transaktionen sind die Knoten im WFG.\\
 \begin{framed}
   Kante $T_i \leftarrow T_j$ beduetet $T_i$ wartet auf die Freigabe der Sperre auf $T_j$ Zyklen in WFG sind dann deadlocks.  
 \end{framed}

\subsection{Deadlock Erkennung} % (fold)
\label{sub:deadlock_erkennung}
\begin{itemize}
  \item Ansatz automatisch Deadlocks erkennen und eine Betroffene Transaktion abbrechen.
  \item Bevorzugter einsatz braucht viel Resourcen ist aber ein einfaches Verfahren.
  \item Timeout: Transaktion die zu lange blockiert ist abbrechen.
  \subitem einfach zu realiseren. 
  \subitem bricht Transaktionen unnötigerweise an.
  \subitem Deadlocks können lange bestehen.
\end{itemize}
WFG kann benutzt werden um :
\begin{itemize}
  \item Deadlock zu finden -
  \subitem  wird eine Transaktion blockiert, Kante in WFG einfügen 
  \subitem WFG periodisch für Zyklen durchsuchen.
  \item Wird ein Deadlock ekannt, eine Transaktion im Zyklus wählen und abbrechen.
\end{itemize}
% subsection deadlock_erkennung (end)

\subsection{Verteilte Deadlocks} % (fold)
\label{sub:verteilte_deadlocks}
\begin{itemize}
  \item Wenn Deadlock zwi Knoten umfasst, kann ihn \textbf{keiner der Knoten erkennen.}
  \item Problem lokale und globaler WFG.
  \item Szenario : $t_1 $ und $T_2$ auf Knoten 1 und $T_3,T_4$ auf Konten 2. $T_1 \rightarrow T_2 \rightarrow T_3 \rightarrow T_1 $
\end{itemize}
\pic{wfg.png}
% subsection verteilte_deadlocks_ (end)
 % section deadlocks (end)
\chapter{Verteilte Transaktionen 2} % (fold)
\label{cha:verteilte_transaktionen_2}

Eine Datenbank ist zuverlässig wenn sie robust und zur Widerherstellung fähig ist.
Diese Idee steckt auch in den ACID eigenschaften :
\begin{description}
  \item[Atomicity] requires that each transaction is "all or nothing": if one part of the transaction fails, the entire transaction fails, and the database state is left unchanged. An atomic system must guarantee atomicity in each and every situation, including power failures, errors, and crashes. To the outside world, a committed transaction appears (by its effects on the database) to be indivisible ("atomic"), and an aborted transaction does not happen.

\item[Consistency] The consistency property ensures that any transaction will bring the database from one valid state to another. Any data written to the database must be valid according to all defined rules, including but not limited to constraints, cascades, triggers, and any combination thereof. This does not guarantee correctness of the transaction in all ways the application programmer might have wanted (that is the responsibility of application-level code) but merely that any programming errors do not violate any defined rules.

\item[Isolation]The isolation property ensures that the concurrent execution of transactions results in a system state that would be obtained if transactions were executed serially, i.e. one after the other. Providing isolation is the main goal of concurrency control. Depending on concurrency control method, the effects of an incomplete transaction might not even be visible to another transaction.[citation needed]

\item[Durability] Durability means that once a transaction has been committed, it will remain so, even in the event of power loss, crashes, or errors. In a relational database, for instance, once a group of SQL statements execute, the results need to be stored permanently (even if the database crashes immediately thereafter). To defend against power loss, transactions (or their effects) must be recorded in a non-volatile memory.
\end{description}

Wir werden auf \texttt{Atomicity} und \texttt{Durability} fokusieren.

\textbf{Fehlerarten:}
\begin{description}
  \item[Transaktionsfehler] Transaktionssabbruch, einseitig, Deadlock
  \item[System oder Kontenausfall] Ausfall von CPU , Haputsapeicher , Stromversogung - Hauptspeicherintervale gehen verloren. Teil oder Totalausfall.
  \item[Plattenfehler] Plattenfehler führen zu verlust gespeicherte Daten. Headcrash, Ausfalldes Controllers.
  \item[Verbindungsausfall] Verlotene oder nicht zuteilbare Nachrichten. Trennung des Netzwerks.
\end{description}

\section{Lokale oder Verteilte Zuverlässigkeit} % (fold)
\label{sec:lokale_oder_verteilte_zuverl_ssigkeit}

\begin{description}
  \item[Zentrales DBS] Behandelt transaktionsfehler, System oder Knotenausfall und Plattenfehler. Bei Systemausfällen oder Plattenfehler handelt es immer um einen vollständigen Ausfall.
  \item[Verteiltes DBS] \underline{Behandelt zusätzlich} koordination bei Transaktionsfehler, System oder Knotenausfall, Plattenfehler \textit{Verbindungsausfall, erhöhte Verfügbarkeit.} 
\end{description}
% section lokale_oder_verteilte_zuverl_ssigkeit (end)

\section{Lokale Widerherstellung} % (fold)
\label{sec:lokale_widerherstellung}
\pic{lres.png}
Einzelnbe DBS sind fähig zur Widerherstellung durch :
\begin{itemize}
  \item angemessene Schreibstrategie - force oder no force.
  \item Angemessene Seitenersetzung (Paging Stategy) - Steal oder No Steal.
  \item Protokollierung von Änderungen - undo/redo.
  \item Write Ahead Logging (WAL) 
  \subitem Vor Commit (Redo) Log Einträge sichern.
  \subitem Vor Änderung der permanenten Datenbank (Undo) Log einträge sichern.
  \item Robustes verfahren benutzen
  \subitem Undo, Redo indempotent.
  \subitem ARIES
\end{itemize}
% section lokale_widerherstellung (end)
\section{Zuverlässigkeit} % (fold)
\label{sec:zuverl_ssigkeit}
\begin{itemize}
  \item Sicherstellen von Atomicity und Durability verteilter Transaktionen.
  \item Jeder Knoten kann
  \subitem lokaler Teil der Transaktion zuverlässig verarbeiten
  \subitem nach Bedarf lokales Commit, Rollback, Recovery durchführen.
  \item Übergeordnete Protokolle müssen sich mit der Koordination der beiteiligten Knoten befassen.
  \subitem Start der Transaktion beim Ursprungsknoten verarbeiten
  \subitem  Read und Write an den Zielknoten verarbeiten, spezielle vorkehrungen bei Replikation.
  \subitem Abort, Commit, und Recover spezifisch für DDB.
\end{itemize}
\subsection{Komponenten} % (fold)
\label{sub:komponenten}
Unterschieden in :
\begin{description}
  \item[Coordinator Prozess]beim Ursprungsknoten, steuert die Ausführung.
  \item[Participant Prozess] bei den anderen Knoten die an der Transaktion beteiligt sind. 
\end{description}
\textbf{Protokolle:}\\
\begin{description}
  \item[Commit Protokolle] Wie das Commit einer Transaktion verarbeiten
wenn mehr als ein Knoten am Commit beteiligt ist?
Regelt Atomicity und Durability einer verteilten Transaktion
\item[Termination Protokolle] Verwendet von Knoten, die vom Ausfall anderer Knoten
betroffen sind.
Regelt für einen Knoten das Commit/Abort,
wenn andere Knoten ausfallen.
Blocking/Non-Blocking: Müssen Knoten auf Recovery anderer
Knoten warten, um Transaktion abzuschliessen?
\item [Recovery Protokolle]Verwendet von Knoten, die ausgefallen sind.
Regelt das Recoveryverfahren beim Wiederanlauf
nach einem Ausfall
Unabhängig: Ein ausgefallener Knoten kann das Ergebnis einer
Transaktion bestimmen, ohne Information von anderen Knoten
zu haben.

\end{description}
% subsection komponenten (end)
% section zuverl_ssigkeit (end)
\section{2 Phase Commit} % (fold)
\label{sec:2_phase_commit}
\begin{enumerate}
  \item Choice: Der Koordinator fragt alle Teilnehmer, ob sie bereit sind für ein
Commit: Prepare.
Jeder Teilnehmer teilt dem Koordinator seine Entscheidung mit:
Vote-Commit oder Vote-Abort.
\item Descision:Der Koordinator trifft endgültige Entscheidung:
Commit: Falls alle Teilnehmer bereit zum Commit
Abort: In allen anderen Fällen
und teilt sie den Teilnehmern mit: Global-Commit oder
Global-Abort.
Diese müssen sich entsprechend verhalten und bestätigen:
Acknowledge.
\end{enumerate}
\pic{2pcabl.png}

\textbf{Beobachtungen}
\begin{enumerate}
\item Ein Teilnehmer kann einseitig Abort durchführen,
vorausgesetzt, er hat nicht mit Vote-Commit geantwortet.
\item Nachdem ein Teilnehmer mit Vote-Commit geantwortet hat,
muss er zum Commit bereit sein und kann nicht mehr
umentscheiden.
\item Ein Teilnehmer im Ready-Zustand muss bereit sein zum
Commit oder Abort, je nach Entscheid des Koordinators.
\item Die endgültige Entscheidung ist Commit, falls alle
Teilnehmer Vote-Commit geantwortet haben, oder Abort,
falls irgendein Teilnehmer mit Vote-Abort geantwortet hat.
\item Koordinator und Teilnehmer verwenden Time-Out
Methoden, um möglichen Wartezustand zu verlassen.

\end{enumerate}
% section 2_phase_commit (end)
\pic[Zentraliziertes 2pc]{z2pc.png}
\pic[Lineares 2PC]{l2pc.png}
\pic[Verteiltes 2PC]{v2pc.png}
\subsection{Zustandsübergänge} % (fold)
\label{sub:zustands_berg_nge and Termination Protocols}
\pic[Zustandsübergänge 2PC]{zu2pc.png}
\begin{itemize}
  \item Time out des Koordinates im Zustand \texttt{wait,abort,commit} möglich.
  \item \texttt{wait} Zustand :
  \subitem Koordinator wartet auf Entscheid der Teilnehmer
  \subitem \textit{Lösung:} Koordinator entscheidet Global-Abort der Transaktion.
  \item \texttt{commit} oder \texttt{abort} : 
  \subitem Koordinator ist nicht sicher ob das Cimmit oder Abort verfahren durch die LRMs\footnote{Local Recovery Manager} alle Teilnehmer durchgeführt wurde.
  \subitem \textit{Lösung:} Global commit oder Global Abort an alle Teilnehmer die nicht geantwortet haben, erneut senden.
  \item Time-Out eines Teilnehmers im Zustand \texttt{initial} oder \texttt{ready} möglich.
  \item \texttt{initial} Zustand : 
  \subitem Teilnehmer wartet auf das Prepare des Koordinators.
  \subitem Lösung: Teilnehmer führt einseitig Abort durch. Falls Prepare später ankommt, mit Vote Abort antworten oder ignorieren.
  \item \texttt{ready} Zustand : 
  \subitem Teilnehmer darf nach dem Vote Commit sich nicht umentscheiden und einseitig Abort durchführen.
  \subitem \textit{Lösung:} Teilnehmer ist \underline{blockiert} und muss auf den entgültigen Entscheid des Koordinators warten.
\end{itemize}
% subsection zustands_berg_nge (end)
\subsection{Recovery Protocols} % (fold)
\label{sub:recovery_protocols}
Protokolle die ein Koordinator oder Teilnehmer nach einem Ausfall verwenden für die Wiederherstellung beim Neustart. \\
Annahmen: 
\begin{enumerate}
  \item Logbeschreibungen und Versenden von Nachrichten sind atomare Aktionen.
  \item Zustandsübergänge erfolgen nach Versenden der Nachricht.
\end{enumerate}
% subsection recovery_protocols (end)
\subsection{Ausfälle} % (fold)
\label{sub:ausf_lle}
\begin{description}
  \item[Koordinator] \hfill \\
  \begin{itemize}
     \item Koordinator im Zustand \texttt{initial} fällt aus - neustart der Transaktion
     \item \texttt{wait} - Neustart des Commit Verfahrens mit nochmailigem Versenden der Prepare Nachricht.
     \item \texttt{commit} oder \texttt{abort} 
     \subitem Falls alle Acknowledge Meldungen erhalten sind nichts tun sonnst Termination Protokolls answenden.
   \end{itemize} 
   \item[Teilnehmer] \hfill \\
   \begin{itemize}
     \item \texttt{initial} Nach der Wiederherstellung führt Teilnehmer einsieitig ein Abort der Transaktion durch.
     \item \texttt{ready} Ablauf wie nach Time out im \texttt{ready} Zustand und Termination Protocols anwenden.
     \item \texttt{commit, abort} Nichts tun.
   \end{itemize}
\end{description}
% subsection ausf_lle (end)
\subsection{Probleme mit 2pc} % (fold)
\label{sub:probleme_mit_2pc}
\begin{itemize}
  \item Blockierend. Im Zustand \texttt{ready} muss Teilnehmer auf Entscheid des Koordinators warten. Fällt Koordinator aus, ist Teilnehmer blockiert. Verfügbarkeit reduziert.
  \item Recovery eines Teilnehmers nach Ausfall im Zustand \texttt{ready }kann nicht unabhängig erfolgen.Verfügbarkeit des Koordinators oder eines Teilnehmers, der
die Rolle des Koordinators übernimmt ist nötig.

\end{itemize}
% subsection probleme_mit_2pc (end)
\section{3 Phase Commit} % (fold)
\label{sec:3_phase_commit}
\begin{itemize}
  \item Nicht Blockierend.
  \item Commit Protocol ist \textbf{nicht blockierend genau dann wenn:}
  \subitem Synchron innerhalb Zustandübergang
  \subitem Keine Zustände hat die benachbahrt sind zu einem Commit und Abort Zustand. Oder die die zu einem Commit zustand \textit{nicht} benachbart sind aber nicht commitable sind.
  \subitem Committable : Zustand in dem sichergestellt ist, dass alle
Teilnehmer mit Vote-Commit geantwortet haben
\end{itemize}
% section 3_phase_commit (end)

% chapter verteilte_transaktionen_2 (end)
\chapter{Replikation I} % (fold)
\label{cha:replikation_i}

\section{Einleitung} % (fold)
\label{sec:einleitung}
\subsection{Grunde für Replikation} % (fold)
\label{sub:grunde_f_r_replikation}
\begin{description}
  \item[Züverlassigkeit] vemeidet "single points of failiure."
  \item[Perormance] Vermeidet Kommunikationskosten durch lokalen Zugriffe.
  \item[Skalierbarkeit] Unterstützt Wachstum des Systems
  \item[Andforderungen durch Anwendungen] Spezifikation??
  \item[Probleme] \begin{itemize}
    \item Transparenz der Replikation - Abbilden logische Zugriffe in Physische Zugriffe auf Kopieen.
    \item Fragen der Konsistenz : Kritieren und Sync der Kopieen.
  \end{itemize} 
\end{description}

% subsection grunde_f_r_replikation (end)
\subsection{Ausführungsmodell} % (fold)
\label{sub:ausf_hrungsmodell}
 Es gibt physische Kopien der logischen Objekte im System. Zugriffe betreffen logische Objekte, werden aber in Zugriffe auf pysische Objekte umgesetzt.: \\
 \pic{amodel.png}
% subsection ausf_hrungsmodell (end)
% section einleitung (end)
\section{Konsistenzmodelle} % (fold)
\label{sec:konsistenzmodelle}
Wie wird die Konsiszenz des globalen DB Zustands definiert?
\begin{itemize}
  \item Transaktionsbezogene Kritieren :
  \subitem One Copy Serializability
  \subitem Snapshot Isolation
  \item Datenbezeogene Kritieren : Eventual Consistency
  \item Begrentzte Inkonsistenz : Epsilon Serializability
\end{itemize}
Mit welchen Verfahren wird geneseitige Konsistenz der Replikate erreicht?

\begin{description}
  \item[Transaktionsbezoegene Kritieren (Strong Consistency)] \hfill \\
  \begin{itemize}
     \item One Copy Equivalenz : Gegenseitige Konsistenz wenn alle Kopien identische Werte haben.Wirkung einer Transaktion auf replizierte Daten ist die gleiche, wenn sie auf einder einzelnen Datenmenge operierte.
     \item One Copy Serializability : Wirkung von Transaktionen auf replizierte Daten ist die gleiche wenn sie eine nach der anderen auf eine einzelnen Datenmenge operierten. Histories sind äquivalenz zu einer seriellen Ausführung auf nicht replizierten Daten.
   \end{itemize} 
   \item[Datenbezogene Kritieren] \hfill \\\
   \begin{itemize}
     \item Angeschwächte formen der Konsistenz:
     \item Eventual Consistency : letzendlich oder später erreicht. Wenn weitere Updates ausblieben kovergieren die Replikate zu identischen Kopien.
     \item Nur die Ausbreitung der Updates muss guarantiert sein. Das ist kein problem solange die Nutzer immer auf das gleiche Replikat zugreifen :
     \subitem Übergang zu Client-Centric Consistency. Guarantiert einem einzelnen Nutzer konsistenten Zugriff auf daten.

   \end{itemize}
   \item[Client Centric Consistency] \hfill  \\
   \begin{itemize}
     \item Monotonic Reads - (gleichbleibend) Liest ein Prozess Datenelement $x$, dann gibt jedes nachfolgende Lesen von x durch diesen Prozess denselben oder neueren Wert zurück.
     \item Monotonic Writes - Schreiboperation auf Datenelement x ist abgeschlossen, bevor derselbe Prozess nachfolgende Schreiboperationen auf x ausführt.
     \item Read your Writes - Die Wirkung von Schreiboperation auf Datenelement x wird immer in nachfolgenden Leseoperationen auf x durch denselben Prozesse gesehen.
     \item Writes follow Reads - Schreiboperationen auf Datenelement x, die auf ein Lesen durch denselben Prozess folgen, überschreiben immer denselben oder neueren Wert von x. 
   \end{itemize}
\end{description}
% section konsistenzmodelle (end)
\section{Update Propogation Strategies} % (fold)
\label{sec:update_propogation_strategies}
Können entweder \texttt{Eager(sync)} oder \texttt{Lazy(async)} im kombination mit \texttt{Primary Copy (Master)} oder \texttt{Update Everywhere (Group)}.
\pic{ups.png}
\begin{description}
  \item[Eager(Sync Replikation)] Jede Änderung wird sofot zu allen Kopien übertragen. Übertragung der Änderungen erfolgt innerhalb der Grenzen der Transaktion.ACID Eignenschaften gelten für alle Kopien.
  \item[Lazy(Async Replikation) ] Zuerst werden die lokalen Kopein geändert. Anschliessend werden die Änderungen zu allen anderen Kopien übertragen \textit{Push/Pull}. Während der Übertraging sind Kopien inkonsistent. Zeitraum der Inkonsistenz kann in Abhängigkeit der Anwendung angepasst werden.
  \item[Master/Primary Copy Replication] Es gibt eine einzigge Kopie auf der Änderungen ausgeführt werden können (Master). Alle anderen Kopien (Slaves) übernehmen die Änderungen vom Master. Für verschiedene Datenelemente können verschidene knoten Master sein.
  \item[Group/Update everywhere Replication] Änderungen können auf jeder Kopie ausgeführt werden. D.h jeder Knoten der eine Kopie besitzt, kann auf ihr Änderungen ausführen.

\end{description}
% section update_propogation_strategies (end)

\section{Replikationsstrategieen} % (fold)
\label{sec:replikationsstrategieen}
\pic{ups1.png}
\begin{description}
  \item[Eager Master (Synchronous Primary Copy) Replication] \hfill \\
  \begin{itemize}
    \item Primary Copy 
    \subitem \textbf{Read}  Lokales lesen (eigene Kopie), Resultat zurückgeben.
    \subitem \textbf{Write}  lokales Schreiben, \textbf{Write an alle slaves weiterleiten (FIFO/Timestamp reihenfolge)}, Kontrolle sofort dem Nutzer zurückgeben.
    \subitem \textbf{Commit} verwende 2PC als Koordinator
    \subitem \textbf{Abort} lokales Abort, slaves informieren.
    \item Slave:
      \subitem \textbf{Read}  Lokales lesen (eigene Kopie), Resultat zurückgeben.
    \subitem \textbf{Write}  \textbf{Write von Master :} ausführen der writes in richtiger Riehenfolge,\textbf{Writes von Client:} zurückweisen oder an Master wieterleiten. 
    \subitem Slave ist Teilnehmer an 2PC der Transaktionen vom Master.
    \item Vorteile : 
    \subitem Änderungen müssen nicht koordiniert werden.
\subitem Keine Inkonsistenzen.
\item Nachteile:
\subitem Längste Antwortzeit
\subitem Nur bei wenigen Updates sinnvoll (Master ist
Flaschenhals)
\subitem Lokale Kopien sind beinahe nutzlos
\subitem selten eingesetzt

  \end{itemize}
\item[Eager Group (Synchronous Update Everywhere) Replication] \hfill  \\
\begin{itemize}
  \item Read One Write All (ROWA)
\subitem jeder Knoten verwendet 2 Phasen Sperrprotokoll
\subitem Leseoperationen werden lokal durchgeführt
\subitem Schreiboperationen werden auf allen Knoten
ausgeführt mithilfe eines verteilten Sperrprotokolls

\item Vorteile:
\subitem Keine Inkonsistenzen
\subitem elegante Lösung (symmetrisch)
\item Nachteile:
\subitem Vielzahl von Nachrichten
\subitem Antwortzeiten der Transaktionen sind sehr lang
\subitem beschränkte Skalierbarkeit (Deadlock
Wahrscheinlichkeit wächst mit Anzahl Knoten)
\end{itemize}
\item[Lazy Master (Asynchronous Primary Copy) Replication] \hfill \\
\begin{itemize}
  \item Primary Copy
\subitem \textbf{Read:} lokales Lesen (eigene Kopie), Resultat
zurückgeben
\subitem \textbf{Write:} lokales Schreiben Kontrolle dem Nutzer
zurückgeben
\subitem \textbf{Commit, Abort:} Transaktion lokal beenden
Irgendwann nach dem Commit: an alle Knoten die
Änderungen in einer einzigen Nachricht übermitteln
(FIFO oder Timestamp Reihenfolge)
\item Slave
\subitem \textbf{Read:} lokales Lesen (eigene Kopie), Resultat
zurückgeben
\subitem Nachricht von Master: Änderungen in richtiger
Reihenfolge (FIFO oder Timestamp) anwenden
\subitem \textbf{Write von Client:} zurückweisen oder an Master
weiterleiten
\subitem \textbf{Commit, Abort:} nur für lokale Read-only Transaktionen
\item Vorteile:
\subitem Keine Koordination nötig
\subitem kurze Antwortzeiten (Transaktionen sind lokal)
\item Nachteile
\subitem lokale Kopien sind nicht aktuell
\subitem Inkonsistenzen (verschiedene Knoten haben
unterschiedliche Werte für das gleiche Datenelement)
\end{itemize}
\item[Lazy Group (Asynchronous Update Everywhere) Replication] jeder Knoten:
\begin{itemize}
\item textbf{Read:} lokales Lesen (eigene Kopie), Resultat
zurückgeben
\item \textbf{Write:} lokales Schreiben Kontrolle dem Nutzer
zurückgeben
\item \textbf{Commit, Abort:} Transaktion lokal beenden
Irgendwann nach dem Commit: an alle Knoten die
Änderungen in einer einzigen Nachricht übermitteln
(FIFO oder Timestamp Reihenfolge)
\item Nachricht von anderem Knoten:
\subitem Erkennen von Konflikten
\subitem Änderungen anwenden
\subitem Reconciliation (Abgleichen)
\item Vorteile:
\subitem Keine Koordination nötig
\subitem kürzeste Antwortzeiten
\item Nachteile
\subitem Inkonsistenzen
\subitem Änderungen könne verloren gehen (Abgleich)

\end{itemize}
\end{description}
% section replikationsstrategieen (end)

% chapter replikation_i (end)

\chapter{Nosql}

\section{Einführung} % (fold)
\label{sec:einf_hrung}
A blogger, often referred to as having made the term popular is Rackspace employee Eric Evans who later
described the ambition of the NoSQL movement as “the whole point of seeking alternatives is that you
need to solve a problem that relational databases are a bad fit for” (cf. [Eva09b]).
This section will discuss rationales of practitioners for developing and using nonrelational databases and
display theoretical work in this field. Furthermore, it will treat the origins and main drivers of the NoSQL
movement.

\subsection{Motivation} % (fold)
\label{sub:motivation}
\begin{description}
  \item[Avoidance of unneeded complexity:]  databases provide a variety of features and strict data
consistency. But this rich feature set and the ACID properties implemented by RDBMSs might be more
than necessary for particular applications and use cases.
As an example, Adobe’s ConnectNow holds three copies of user session data; these replicas do not neither
have to undergo all consistency checks of a relational database management systems nor do they have to
be persisted. Hence, it is fully sufficient to hold them in memory (cf. [Com09b]).
\item[High Throughput]Some NoSQL databases provide a significantly higher data throughput than traditional
RDBMSs. For instance, the column-store Hypertable which pursues Google’s Bigtable approach allows the
local search engine Zvent to store one billion data cells per day [Jud09]. To give another example, Google
is able to process 20 petabyte a day stored in Bigtable via it’s MapReduce approach [Com09b].
\item[Horizontal Scalability and Running on Commodity Hardware ] NoSql trying to address problems such as Data Scaling, Performance of single servers, Rigid schema design.Nosql databases are designed to scale well in the horizontal direction and not rely on highly available hardware. Sharding is easier.
\item[Avoidance of expensive Object Relational Mapping]Most of the NoSQL databases are designed
to store data structures that are either simple or more similar to the ones of object-oriented program-
ming languages compared to relational data structures. They do not make expensive object-relational
mapping necessary (such as Key/Value-Stores or Document-Stores). This is particularly important for
applications with data structures of low complexity that can hardly benefit from the features of a re-
lational database. Dare Obasanjo claims a little provokingly that “all you really need [as a web de-
veloper] is a key<->value or tuple store that supports some level of query functionality and has de-
cent persistence semantics.” (cf. [Oba09a]). The blogger and database-analyst Curt Monash iterates
on this aspect: ”SQL is an awkward fit for procedural code, and almost all code is procedural. [For
data upon which users expect to do heavy, repeated manipulations, the cost of mapping data into
SQL is] well worth paying [. . . ] But when your database structure is very, very simple, SQL may not
seem that beneficial.“ Jon Travis, an engineer at SpringSource agrees with that: “Relational databases
give you too much. They force you to twist your object data to fit a RDBMS.” (cited in [Com09a]).
\item[Complexity and Cost of Setting up Database Clusters] See easier sharding.
\item[Compromising Reliability for better performance]  “different scenarios
where applications would be willing to compromise reliability for better performance.” As an example of
such a scenario favoring performance over reliability, he mentions HTTP session data which “needs to be
shared between various web servers but since the data is transient in nature (it goes away when the user
logs off) there is no need to store it in persistent storage.”
\item[Porblems with one size fits all thinking] \begin{enumerate}
  \item Continuous growth of data
  \item Process large amounts of data in short amounts of time.
  \item Social media example - unrelated islands of data, low user/transaction value - no need for string data integrity.
\end{enumerate}
\item[Myth of Efffortless Dist and Partitioning of centralised data models] and nor work correct any longer. The professionals of Ajatus agree with this in a blog post stating that
if a database grows, at first, replication is configured. In addition, as the amount of data grows further,
the database is sharded by expensive system admins requiring large financial sums or a fortune worth of
money for commercial DBMS-vendors are needed to operate the sharded database (cf. [Aja09]).Shalom concludes that in his opinion relational database management systems will not disappear soon.
However, there is definitely a place for more specialized solutions as a ”one size fits all“ thinking was and
is wrong with regards to databases.
\item[Developments in Programming] ORMs\footnote{Object Relational Mappers} hide SQL in many frameworks.The NoSQL databases react on this trend and try to provide data structures in their APIs that are closer to the ones of programming languages (e. g. key/value-structures, documents, graphs).
\item[Cloud Computing] \begin{enumerate}
  \item High -> inf scalability. 
  \item Low admin/overhead.
\end{enumerate}
Following databases work well:
\begin{itemize}
  \item Data warehousing specific dbs
  \item Simple scalable and fast key value stores.
 \item %todo
\end{itemize}
\item The RDBMS plus Caching-Layer Pattern/Workaround vs. Systems Built from Scratch with Scalability in Mind : As scalability requirements grow and these technologies are less and less capable to suit with them. In
addition, as NoSQL datastores are arising Hoff comes to the conclusion that “[with] a little perspective,
it’s clear the MySQL + memcached era is passing. It will stick around for a while. Old technologies
seldom fade away completely.” (cf. [Hof10c]). As examples, he cites big websites and players that have
moved towards non-relational datastores including LinkedIn, Amazon, Digg and Twitter. Hoff mentions
the following reasons for using NoSQL solutions which have been explained earlier in this paper:
\begin{itemize}
  \item  Relational databases place computation on reads, which is considered wrong for large-scale web
applications such as Digg. NoSQL databases therefore do not offer or avoid complex read operations.
\item The serial nature of applications1 often waiting for I/O from the data store which does no good to
scalability and low response times.
\item  Huge amounts of data and a high growth factor lead Twitter towards facilitating Cassandra, which
is designed to operate with large scale data.
\item  Furthermore, operational costs of running and maintaining systems like Twitter escalate. Web ap-
plications of this size therefore “need a system that can grow in a more automated fashion and be
highly available.” (cited in [Hof10c]).
\end{itemize}
\item[Yesterday vs Today]by Stonebraker, see below). In the 1960s and 1970s databases have been designed for single, large high-
end machines. In contrast to this, today, many large (web) companies use commodity hardware which will
predictably fail. Applications are consequently designed to handle such failures which are considered the
“standard mode of operation”, as Amazon refers to it (cf. [DHJ+ 07, p. 205]). Furthermore, relational
databases fit well for data that is rigidly structured with relations and allows for dynamic queries expressed
in a sophisticated language. Lehnhardt and Lang point out that today, particularly in the web sector, data
is neither rigidly structured nor are dynamic quieries needed as most applications already use prepared
statements or stored procedures. Therefore, it is sufficient to predefine queries within the database and
assign values to their variables dynamically (cf. [PLL09]).
Furthermore, relational databases were initially designed for centralized deployments and not for distribu-
tion. Although enhancements for clustering have been added on top of them it still leaks through that
traditional were not designed having distribution concepts in mind at the beginning (like the issues adverted
by the “fallaclies of network computing” quoted below). As an example, synchronization is often not im-
plemented efficiently but requires expensive protocols like two or three phase commit. Another difficulty
Lehnhardt and Lang see is that clusters of relational databases try to be “transparent” towards applications.
This means that the application should not contain any notion if talking to a singly machine or a cluster
since all distribution aspects are tried to be hidden from the application. They question this approach to
keep the application unaware of all consequences of distribution that are e. g. stated in the famous eight
fallacies of distributed computing (cf. [Gos07]2 :
\begin{enumerate}
  \item  The network is reliable
\item  Latency is zero
\item  Bandwidth is infinite
\item  The network is secure
\item  Topology doesn’t change
\item  There is one administrator
\item  Transport cost is zero
\item  The network is homogeneous

\end{enumerate}
\end{description}
% subsection motivation (end)
% section einf_hrung (end)
\section{Cassandra} % (fold)
\label{sec:cassandra}
\begin{enumerate}
  \item \textbf{Welches sind die wichtigsten Gründe, die zur Entwicklung Ihres NoSQL Systems führten? Welche Anforderungen erfüllen diese Systeme?}\\
  \begin{itemize}
    \item Verarbeitung von grossen Datenmengen
\item Availability
\item Reliability   
\item Kein Single Point of Failure
\item Läuft auf Cheap Comodity Hardware
\item High Write Output
\item Inbox Search Support (Suche durch die Facebook Inbox)
\item Geographisch verteilt

  \end{itemize}
\item \textbf{Welches „logische“ Datenmodell bietet das System? Welches API bietet das System? Wie werden Daten manipuliert und abgefragt?}\\
\begin{description}
  \item[Datenmodell:] Eine Instanz von Cassadra besteht aus nur 1 Tabelle, die eine verteilte, multidimensionale Map bezeichnet, die durch einen Schlüssel indexiert ist. Schlüssel: 16-36 Bytes, Value: Object.
Struktur:\\
 Zeilen mit String als Schlüssel\\
Spalten: haben Namen und speichern eine Anzahl Werte, die durch einen Zeitstempel identifiziert sind. Jede Zeile kann eine unterschiedliche Zahl von Spalten haben.\\
 Spaltenfamilien: willkürliche Zahl Spalten pro Zeile können zu Familien zusammengefasst werden. Es können beliebig viele Familien spezifiziert werden, aber effektiv nur wenige. Familien können dynamisch zur Laufzeit mit Spalten und Superspalten ergänzt werden.\\
-Superspalten: weitere Gruppierung: Name und willkürliche Anzahl Spalten
Ansprache von Daten über Tripel (Zeilenschlüssel, Spaltenschlüssel, Zeitstempel). Zeilenschlüssel über column-family:supercolumn:column
Sortierung nach Zeitstempel oder Name möglich
\item[API:] drei Operationen:\\
- get(table, key, columnName)\\
- insert(table, key, rowMutation)\\
- delete(table, key, columnName) à Familie, Superzeile, Zeile\\
Operationen sind atomar per replica, Anzahl betroffene Spalten ist egal.\\
-read: Konsistenzgarantie kann spezifiziert werden (nächster Knoten, Antworten verschiedener Knoten)\\
-insert/update: „a quorum of replica nodes has to acknowledge the completion of the writes“. Weiterleitung zu willkürlichem Server eines Cassandra Clusters, so dass Partitionierung und Redistribution ermöglicht werden. 
\end{description}
\item \textbf{Verteilungsarchitektur}\\
Ringarchitektur mit allen Knoten. Jedem Knoten ist ein bestimmter Range der Hashkeys zugeteilt.
Replikation\\
\textbf{SimpleStrategy:} Replikation auf den N-1 nächsten Knoten\\
\textbf{NetworkTopologyArchiteture:} definierbare Anzahl Replikate pro Datencenter\\
\item \textbf{ Wie werden Daten repliziert? Wie wird die Konsistenz der Daten sichergestellt?}\\
\begin{itemize}
  \item Auf jeder Instanz ist ein Replikationsfaktor N definiert, wobei alle Daten auf N Hosts repliziert werden
\item Jeder Schlüsser wird an einen "Coordinator Node" zugewiesen, welcher einerseits die Daten dazu lokal speichert und andererseits die Daten auf N-1 Knoten verteilt
\item Daten können nach verschiedenen Richtlinien verteilt werden z.B. Rack Unaware, Rack Aware und Datacenter Aware um Risiken besser zu vermindern
\item Cassandra bestimmt mit hilfe von Zookeeper einen Leader, welcher die Verantwortlichkeit für Datenbereiche an die Knoten verteilt
\item Metadaten werden lokal und fehlertolerant in Zookeeper gespeichert
\item Die Daten werden über mehrere Datacenter repliziert und sollten auch bei Ausfall eines ganzen Datacenters verfügbar sein.
\item Konsistenz wird sichergestellt indem ein Quorum festgelegt wird, wieviel Knoten einen Schreibvorgang bestätigen müssen, auch für Lesevorgänge kann ein Quorum festgelegt werden, muss aber nicht
\end{itemize}
\item \textbf{Welche Vorkehrungen werden für die Verfügbarkeit und Ausfallsicherheit getroffen?} \\
\begin{itemize}
  \item Daten bzw. Nodes sind geographisch verteilt und werden über mehrere Datencenter repliziert. Dadurch wird gewährleistet, dass die Latenzzeiten klein gehalten werden und der Ausfall eines Knotenpunktes keinen Einfluss auf das System hat. Durch das abschwächen des Quorum-System, der die Konsistenz aufrecht erhalten soll, wird die Beständigkeit des Systems garantiert.
\item Fehlerdetektion durch Accrual Failure Detector: Jedem Knoten wird ein Wert zugeteilt, anhand dessen man abschätzen kann, wie hoch die Chance ist, dass dieser Knoten ausfällt, statt diesem Knoten einen boolean-Wert zuzuteilen.
\item Der Knoten wählt beim Starten ein Random-Token, um seine Position auf dem Ring zu ermitteln. Beim joinen in ein Cluster, wird der Node dessen Config-File lesen. Bei einem Ausfall eines Teilsystems, wird nun aber kein Re-Balancing vorgenommen, da angenommen wird, dass es wieder online kommen wird. Um vorzubeugen, dass nicht erreichbare Nodes anderen Instanzen beitreten, wird beim Kommunizieren der Cluster-Name der Cassandra-Instanz mitgeschickt, um seinen Quellort zu bestimmen. Das hinzufügen oder entfernen eines Knoten aufgrund eines manuellen Fehlers im Konfiguration-File muss ebenfalls manuell über ein Command Line Tool erfolgen.
\end{itemize}
\item \textbf{Wie werden die Daten lokal auf den einzelnen Knoten gespeichert?} Eine Schreiboperation schreiben in ein Commit-Log. Auf jeder Maschine befindet sich eine Disk welche nur dem Commit-Log zugeteilt ist (dedicated). Das Write-Log ist dabei sequentiell. Konnten die Daten erfolgreich in's Commit-Log geschreiben werden, werden die Daten zusätzlich in eine In-Memory Struktur geladen. Ab einer gewissen Speichergrösse, wird die In-Memory-Datenstruktur auf eine Disk geschrieben. Alle Schreiboperationen sind dabei sequentiell und die Daten werden gleichzeitig für einen schnelleren Zugriff indexiert. Mit der Zeit entstehen somit mehrere Files mit den entsprechenden Dump Daten. Ein Merge Prozess fügt die verschiedenen Dump-Files wieder zu einer Datei zusammen.

Ein Query auf den Daten arbeitet Primär auf den In-Daten-Memory. Ein Lookup auf den Files geht von neueren Files über zu den älteren Files. Die Daten werden dabei mit Hilfe eines keys gesucht. Um zu verhindern, dass Files durchsucht werden, welche den Filter nicht enthalten wird ein Bloom Filter über die vorhandenen Keys verwendet. Eine Tabelle der Bloom Filter wird sowohl auf jedem Dump-File als auch In-Memory gehalten.
\end{enumerate}
% section cassandra (end)

\section{Neo4J} % (fold)
\label{sec:neo4j}
\begin{enumerate}
  \item \textbf{Welches sind die wichtigsten Gründe, die zur Entwicklung Ihres NoSQL Systems führten? Welche Anforderungen erfüllen diese Systeme?}\\
  Neo4j gehört in die Kategorie der Graphendatenbanken und wurde ursprünglich für die Echtzeitsuche von verschlagworteten(indexierten) Dokumenten in einem Online Dokumentationssystem entwickelt.
„Viele aktuelle Anwendungen müssen stark vernetzte, rekursive Daten behandeln, welche sich nur unzureichend auf relationalen Datenbanken abbilden lassen. Ein gutes Beispiel dafür sind soziale Netzwerke. Auch ist es schwierig, das exakte Schema der Daten bereits zur Entwurfszeit zu kennen, speziell im Umgang mit Benutzergesteuertem Inhalt.“
Frei übersetzt aus: http://highscalability.com/neo4j-graph-database-kicks-buttox
Neo4j ist in Java implementiert und sowohl als Bibliothek in die JVM eingebettet als auch als Server-Version verfügbar. Die Datenbank nutzt einen eigenen, optimierten Persistenzmechanismus für die Speicherung und Verwaltung der Graphen.
Neo4j ist als open source und kommerzielle Version verfügbar.
Die Datenbank ist gut dokumentiert und verfügt über einen breiten Community Support
Verfügt über Schnittstellen zu diversen Programmiersprachen, ein REST Interface ist der empfohlene Zugriffs-weg.
Hoch skalierbar.
Das .jar ist kleiner als 500kb und hat nur eine dependency.
Einfaches API.
\item \textbf{Welches „logische“ Datenmodell bietet das System? Welches API bietet das System? Wie werden Daten manipuliert und abgefragt?}\\
Das Datenmodell hinter neo4j ist ein Graph Datenbank System.Dabei gibt es Knoten und Kanten welche in einem Property Graphen das System ausmachen. Das heisst, zwischen den Knoten gibt es Relationships und die Kanten und Knoten haben verschiedene Properties :
\pic{n4jkk.png}
Um die Daten zu erhalten, spielt die Traversierung eine wesentliche Rolle. Jeder Knoten und jede Kante speichert einen “Mini-Index” mit den eigenen verbundenen Objekten. Globale Indizes werden als Startpunkte für die Traversierung festgelegt. Weiter werden Indizes benötigt um schnell Knoten mit bestimmten Werten zu erhalten. Danach wird nach bestimmten Mustern im Graphen gesucht.\newpage
\textbf{API:}  \pic{n4japi.png} 
\item \textbf{Wie werden Daten über mehrere Knoten verteilt (Verteilungsarchitektur)?}\\
\pic{n4jvt.png}
\item \textbf{Wie werden Daten repliziert? Wie wird die Konsistenz der Daten sichergestellt?} \\
Replikationen sind nur mit Neo4j Enterprise möglich. Dabei werden Cluster eingesetzt, die aus einem Master und mehreren Slaves bestehen. Die Daten werden automatisch auf die Slaves repliziert. Es sollte dabei aber darauf geachtet werden, dass Schreibvorgänge nur auf dem Master passieren. Die Replikation kann auf zwei Arten geschehen:
\begin{itemize}
\item Automatisch durch Abruf der Schreibtransaktionen seitens des Slaves auf dem Master (Millisekunden - paar Minuten)
\item  Automatisch durch Propagieren der Schreibtransaktionen seitens des Master auf eine vorgegebene Anzahl an Slaves (sofort). Dies wird über den push-factor definiert.
Jeder Slave besitzt die ganze Datenbank. Die Skalierbarkeit ist somit nahezu linear (Read-Scaling), da Lesevorgänge beliebig auf die Slaves verteilt werden können.
Bei Graphen Datenbanken ist nie die ganze Datenbank im Arbeitsspeicher, sondern nur der Teil, welcher bei der letzten Abfrage geladen wurde. Dies bietet nun die Möglichkeit über ein intelligentes und konsistentes Routing dieselben Klassen von Anfragen auf die gleichen Empfänger weiterzuleiten. So müssen nicht nach jeder Anfrage neue Teile geladen werden (Cache Sharding).
\end{itemize}
\item \textbf{Welche Vorkehrungen werden für die Verfügbarkeit und Ausfallsicherheit getroffen?}\\
Dies ist nur bei einem Cluster möglich. Es sind zwei Arten des Ausfalls möglich.
Fällt ein Slave aus, wird dies von den anderen Datenbankinstanzen im Netzwerk erkannt und er wird als vorübergehend nicht verfügbar gekennzeichnet. Sobald er wieder verfügbar ist, sorgt er selbst dafür, dass er wieder auf dem aktuellen Stand des Masters ist.

Fällt der Master aus, sieht das ganze etwas anders aus. Der Cluster entscheidet, welcher Slave zum neuen Master wird. Bedingung ist lediglich, dass der Slave zum Master werden darf. Somit ist ein Failover kein Problem. Sobald der Master nicht mehr verfügbar ist, werden alle laufenden Schreibtransaktionen zurück gerollt. Der Wechsel vom Slave zum Master geschieht dann im Normalfall innert Sekunden. Während dieser Zeit sind keine Schreibtransaktionen möglich.
Die einzige Ausnahme ist, wenn auf dem alten Master noch Schreibvorgänge vorhanden sind, die noch auf keine andere Datenbankinstanz repliziert werden konnten und bereits Änderungen am neuen Master gemacht wurden. In diesem Fall wird der alte Master zwei „Branches“ führen. Der erste „Branch“ ist seine alte Datenbank. Dieser „Branch“ wird abgezogen, da er nicht mehr aktuell ist, jedoch noch nicht replizierte Änderungen enthält. Dann lädt er sich für den zweiten „Branch“ eine komplette Kopie des aktuellen Zustands herunter und wird dann selbst zum Slave.
\item \textbf{Wie werden die Daten lokal auf den einzelnen Knoten gespeichert?}Nodes, Properties und Relationships werden separat voneinander gespeichert. (*.nodestore.db, *.relationshipstore.db, *.propertystore.db)
\end{enumerate}
% section neo4j (end)
\section{Bigtable} % (fold)
\label{sec:bigtable}
\begin{enumerate}
  \item \textbf{Welches sind die wichtigsten Gründe, die zur Entwicklung Ihres NoSQL Systems führten?}\\
  Google brauchte ein unterliegendes System für Ihre Dienste. Diese Dienste arbeiten mit grossen Datenmengen (Petabytes), sind auf viele tausende Server im Cluster verteilt und generieren viele neue Einträge, aber wenige Änderungen an vorhandenen Einträgen. 
Eingesetzt u. a. für folgende Produkte: Google Suche (web indexing)
Google Earth
Google Finance\\

\textbf{Welche Anforderungen erfüllen diese Systeme?} \\
breite Anwendungspalette\\
Skalierbarkeit\\
Hochverfügbarkeit\\
Hochleitungsfähigkeit\\
\item \textbf{Speicherung und Verteilungsarchitektur:}\\
\begin{description}
  \item[Speicherung im Dateisystem:] A Bigtable is a sparse, distributed, persistent multidimensional sorted map.
Der Schlüssel für den Index besteht aus:\\
row key\\
column key\\
timestamp\\
Jeder Wert ist ein Array aus beliebigen Bytes (String).\\
(row:string, column:string, time:int64) -> string 
\item [Speicherung der Tabellen:]Tabellen werden in „tablets“ unterteilt und so gespeichert. Die Grösse eines solchen Segments ist etwa 200 MB. Dies ist optimiert für das Google File System (GFS). Wenn viele Daten vorhanden sind werden diese komprimiert. Es wird das „SSTable file format“ verwendet.
\item[Zugriff auf die Daten]Pro Tabelle existiert ein Index, der komplett(?) in den Speicher geladen wird. Das auffinden der Daten im Dateisystem ist als Binär-Suche möglich. Der Index bezieht sich direkt auf den Block auf der HDD, so dass Zugriffe auf HDD optimiert werden können.
Die Daten sind grundsätzlich unveränderbar (SSTable ist immutable). Somit ist Zugriff immer Thread-Safe. Es werden Serien von SSTables verwendet, so dass Änderungen möglich sind. Änderungen kommen aber zuerst ins änderbare „memtable“. Darin sind also immer die neuesten Änderungen gespeichert. Zugriff ist hier immer noch parallel möglich dank Copy-On-Write.
Ein Tablet-Server verwaltet etwa 10-1000 Tablets.
\item[Garbage Collection]Das GFS hat eine GC (Mark-And-Sweep). Der Master-Server muss sich darum kümmern.
\item[Verteilung der Daten:] Speziell ist hier dass das GFS bereits verteilt ist. BigTable benötigt ein Cluster-Management-System zum Scheduling, Resourcen-Management und für Erkennung und Verarbeitung von Fehlern. 
Es gibt einen Master-Server und mehrere Tablet-Server. Solche Tablet-Servers können dynamisch entfernt oder hinzugefügt werden. Chubby wird verwendet damit Tablet-Server sich im Cluster anmelden können.
\item[Datenmodell:]persistent multidimensional sorted map.\\
Zeilen werden lexographisch sortiert. (Row Keys 64KB)\\
Tabelle wird unterteilt in Tablets. Mehrere Rows ergeben ein Tablet. Diese werden auf verschiedene Server verteilt.\\
Keine Limitierung in der Anzahl Spalten\\
\item[Gruppierung der Spalten -> Column Families]\hfill \\
Auf den Column Families werden die Access Control Rules definiert\\
Lesbarer Name\\
best Practice: wenige Column Families. Keine Änderungen an diesen.\\
Eine Gruppe -> Ein file\\
\item[Timestamps]Jede Zelle kann mehrer Versionen der selben Daten beinhalten. Daher wird jeder Eintrag mit einem 64bit timestamp versehen.\\
\pic{tsbt.png}
\item[API] \hfill \\
Read Operations: Row by Key, Limitation of Column Families + TimeStamps, Column iterators\\
Write for Rows: create, delete values of column\\
Write for Columns Familiies: create, delete\\
Administration: cluster, table, column family metadata --> Access Control jede read/write operation in einer Zeile ist atomar\\
\end{description}
\item \textbf{Wie werden Daten repliziert? Wie wird die Konsistenz der Daten sichergestellt?} BigTable verwendet ein Cluster-Management-System (Chubby) um mit verteilten Knoten zu interagieren. Jede Chubby Instanz besitzt einen Cluster von 5 aktiven Kopien, von welchem jeweils einer der Master ist und Anfragen bearbeitet. Um die Kopien konsistent zu halten wird der Paxos Algorithmus verwendet. Die gesplitteten Tablets werden auf die verschiedenen Konten verteilt.

Chubby ist ein verstreuter Lock-Service, der ein BigTable Cluster mit mehrere tausenden
Knoten koordiniert und 5 Replikationen hat. Der Klient verbindet sich jeweils zu Chubby um mit den Daten zu arbeiten. Falls für eine bestimmte Zeit Chubby unerreichbar wird, wird BigTable auch unerreichbar.
 
Für Bigtable sind zurzeit zwei Replikationsmechanismen verfügbar. Einerseits die Master/Slave Replikation, bei der ein Datacenter als Master ausgewählt wird, und sämtliche Schreibzugriffe über dieses Center abgewickelt werden. Hierbei werden die Änderungen asynchron auf andere Datacenter repliziert. 

Andererseits wird der High Replication Datastore angeboten, der mittels des sog. axos-Algorithmus die Schreibanfragen auf verschiedene Datacenter verteilt und diese dann synchron auf andere Datacenter repliziert. Bigtable besitzt eigene Algorithmen, die eine automatische Lastanpassung (Skalierung) durchführen, daher ist kein manueller Eingriff notwendig.

HBase selbst kennt keine Replikation. Dort wird mit Hilfe von Hadoop die Replikation durchgeführt. 
\begin{description}
  \item[Verfügbarkeit]    Jeder TabletServer schreibt seine CommitLogs durch 2 Threads, jeder Thread schreibt ins eigene LogFile. Zu 1 Zeitpunkt ist nur 1 Thread aktiv. Wenn Writes durch den aktiven Thread zu langsam werden, wird der andere Thread als der aktive gesetzt. Alle Commits auf der Queue, die auf Write beim langsamen Thread gewartet haben, werden vom neuen Thread übernommen.
·         Performance: Caching! The Scan Cache is a higher-level cache that caches the key-value pairs returned by the SST able Interface to the tablet server code (bestens geeignet fürs Lesen der gleichen Daten). The Block Cache is a lower-level cache that caches SST ables blocks that were read from GFS (bestens geeignet für Lesen der Daten, die in der Nähe der neulich gelesenen Daten sind, d.h. sequentielles Lesen oder Lesen im gleichen Block).
\item[Ausfallsicherheit] Wenn ein Tablet Server stirbt: Seine Tablets werden auf die andere Server verteilt. Um den State jedes Tablets nachzubauen, muss der CommitLog durchgesucht werden auf betreffende Commits, um diese auf den Tablets auszuführen. D.h. 100 Server – 100 Reads auf 1 File. Da es schlechtes Vorgehen ist: Der CommitLog wird sortiert nach Table-Key. CommitLog wird zuerst auf 64 MB grosse Blocks aufgeteilt, das Sortieren der Blöcke passiert parallel. Das sortierte Output ermöglicht blockweises sequentielles Lesen der nötigen Daten.
\end{description}
\end{enumerate}
% section bigtable (end)
\section{MongoDB} % (fold)
Bemerkung : Entschuldigung für CAPS LOCK.
\label{sec:mongodb}
\begin{enumerate}
  \item \textbf{Welches sind die wichtigsten Gründe, die zur Entwicklung Ihres NoSQL Systems führten? Welche Anforderungen erfüllen diese Systeme?}\\
  \begin{description}
    \item[DOKUMENTORIENTIERT:] Verwendet BSON (Binary JSON), da sehr effizient
   - Bessere Interaktion mit modernen (objektorientierten) Programmiersprachen als Relationale Datenbanken
   \item[FLEXIBILITÄT:]  Schemaloses Datenmodell.\\
   - Dokumente können jederzeit erweitert werden ohne dass hohe Aufwände entstehen (Einfaches hinzufügen von Feldern und Listen, kein Fehler beim Lesen von noch nicht existierenden Datenbanken).\\
   - Iterative Entwicklung möglich \\
   \item[MÄCHTIGKEIT] - Hoher Funktionsumfang\\
   - Komplexe Datentypen möglich (Strukturen mit dynamischen Attributen, Arrays, usw.)
   \item[GESCHWINDIGKEIT/SKLAIERBARKEIT]    Schneller als SQL-Systeme durch:
   - Key-Value Prinzip: Zugriff direkt auf den Key anstatt diesen via Indexierung zu suchen
   - Vermeidung von Joins. Alle Daten können direkt über die aktuelle Liste im Dokument abgefragt werden
   - In-Memory lesen/schreiben
  \end{description}
  \item \textbf{Welches „logische“ Datenmodell bietet das System?} \hfill  \\
  \begin{itemize}
    \item MongoDB-Server besteht  aus mehrere MongoDB Datenbanken
\item MongoDB Datenbank besteht aus einer odere mehreren Sammlungen
eine Sammlung (hat einen Namen) besteht aus einem oder mehreren Dokumenten.
\item Es ist möglich eine hierarchische Struktur der Sammlung zu bilden, indem ein Punkt eingefügt wird. Dies wird jedoch nicht so logisch abgebildet.
\item ein Dokument besteht aus mehreren Feldern (Werte, Arrays, binär Daten, sub-Dokumente)
\item Jedes Dokument in der gleichen Sammlung kann verschiedene Felder haben.
\item Das Format wird BSON genannt und ist ähnlich wie JSON, jedoch in binärer Form damit es schneller Verarbeitet werden kann.
\item \pic{mdbapi.png}
\item \pic{mdbapi2.png}
\item \pic{mdbapiextra.png}
\item \pic{mdbsh.png}
  \end{itemize}
\item \textbf{Wie werden Daten repliziert? Wie wird die Konsistenz der Daten sichergestellt?}\\
a) MASTER-SLAVE-REPLIKATION (Alte Variante)\\
MongoDB unterstützt die Master-Slave-Replikation. Ein Master kann Lese- („Reads“) und Schreibzugriffe („Writes“) ausführen. Ein Slave kopiert die Daten vom Master und kann nur für Lesezugriffe oder Backups eingesetzt werden, nicht aber für Schreibzugriffe. Fällt der Masterknoten aus, so kann einer der Slaveknoten zum neuen Masterknoten umgewandelt werden (muss manuell erfolgen, Administrationsaufwand).

b) REPLICA SET (Neue Variante)\\
Ein Replica Set ist eine Gruppe von von Knoten, die Kopien voneinander sind. Einer dieser Gruppe wird als PrimärerKnoten gewählt. Nur über diesen werden Schreiboperationen abgewickelt. Der Primärknoten reicht diese an die anderen Knoten der Gruppe weiter. Wenn der Primärknoten ausfällt, übernimmt ein beliebiger anderer Knoten aus der Gruppe dessen Rolle (dies kann automatisch erfolgen, kein Administrationsaufwand). Vorteile der neuen Variante sind z.B:
- Geringerer Administrationsaufwand\\
- Konsistenzbedingungen können einfach gesetzt werden (z.B. Schreiben ist erfolgreich wenn mehr als die Hälfte des Replica Sets die Änderungen übernommen hat)\\
- Grosse Skalierbarkeit, da nahezu jeder Knoten aus einem Replica Set für Leseoperationen verwendet werden kann\\
\item \textbf{Welche Vorkehrungen werden für die Verfügbarkeit und Ausfallsicherheit getroffen?} \\
\begin{description}
\item[REPLICATION]
Operationen welche die Datenbank auf dem PRIMARY verändern, werden in einem Log namens
oplog gespeichert. Der 'oplog' enthält eine geordnete Menge von idempotenten Operationen,
welche auf den "Secondaries" repliziert werden. Die Grösse des oplogs beträgt standardmässig 5%
des freien Festplattenspeichers. Der oplog enthält Änderungen einiger Stunden.
Sollte ein Secondary ausfallen, kann er so wieder "aufholen". Sollte ein Secondary länger als
die Periode vom oplog beträgt ausfallen, dann werden mittels "initial synchronization" alle Datenbanken usw. komplett
repliziert.

\item[ELECTIONS AND FAILOVER]
Falls der Primary aus irgendeinem Grund ausfallen sollte, dann bestimmen die Secondaries unter sich
einen neuen Primary. Dieser Prozess wird "Election" genannt. Sobald der neue Primary bestimmt wurde,
konfigurieren sich die anderen Seconaries selber, so dass sie nun Updates vom neuen Primary erhalten.
Sollte der ausgefallene Primary wieder online kommen, erkennt er, dass er nicht mehr der Primary ist
und wird zu einem Secondary.

\item[ELECTION PRIORITY] Der "Election" Prozess kann mit einer Prioritäts-Konfiguration beinflusst werden. Der Secondary
mit der höchsten Priorität wird dann Primary. Zum Beispiel könnte man alle Instanzen in
einem "Secondary Data Center" so konfigurieren, dass nur wenn das komplette "Primary Data Center" ausfällt,
einer der Secondarys dann Primary wird.

\item[CONFIGURABLE WRITE AVAILABILITY]
MongoDB besitzt die Möglichkeit, dass ein Client bei jeder Abfrage einen "write concern" mitteilen kann.
Dieser gibt an, wann der Client eine Antwort erhält. Der Client wartet dann solange auf eine Antwort bis der
gewünschte "write concern" eingetreten ist und kann so z.B. sicherstellen, dass die Daten definitiv geschrieben wurden.

Folgende Optionen sind z.B. möglich: \\
- Client braucht keine Antwort, (One-Way) -> Keine Garantie, dass Daten geschrieben werden\\
- Client will eine Bestätigung sobald die Änderungen auf mehreren Replicas vollzogen worden sind.\\
- Client will eine Bestätigung sobald die Änderungen auf der Mehrheit der Replicas vollzogen worden sind.\\
- Client will eine Bestätigung sobald die Änderungen auf allen Replicas vollzogen worden sind.\\
- Und weitere, einstellbare, Möglichkeiten wären z.B: Mind. 2 Replicas enthalten die Änderungen im Primary-Datacenter und min. 1 Replica im Secondary Datacenter.\\

\end{description}
\item \textbf{Abspeicherung Lokal} \pic{extents.png}
\end{enumerate}
% section mongodb (end)

\part{Labs} % (fold)
\label{prt:labs_}
\section{Lab 1 - Trigger}
\subsection{Event Logging}
\subsubsection{Aufgabenstellung}
Manipulationen auf der Tabelle REGISTRIERUNGEN loggen. Trigger und Log-Tabelle befinden sich auf dem gleichen System wie REGISTRIERUNGEN.\\
\subsubsection{Lösung}
\begin{lstlisting}[style=sql, title=Log-Tabelle]
-- Erste Möglichkeit
CREATE TABLE registrierungen_log (
  username VARCHAR2(15) NOT NULL,
  systemdate DATE NOT NULL,
  maniptype VARCHAR2(3) NOT NULL,
  oldval VARCHAR(15), 
  newval VARCHAR(15)
);
-- Zweite Möglichkeit
CREATE TABLE registrierungen_log (
  username VARCHAR2(15) NOT NULL,
  systemdate DATE NOT NULL,
  maniptype VARCHAR2(3) NOT NULL,
  oldmnr VARCHAR2(4) NULL,
  newmnr VARCHAR2(4) NULL,
  oldfnr VARCHAR2(3) NULL,
  newfnr VARCHAR2(3) NULL,
  oldanr VARCHAR2(4) NULL,
  newanr VARCHAR2(4) NULL,
  olddatum DATE NULL,
  newdatum DATE NULL
);
\end{lstlisting}
\begin{lstlisting}[style=sql, title=Trigger]
-- Erste Möglichkeit
CREATE OR REPLACE TRIGGER registrierungen_change
  BEFORE 
    INSERT OR
    UPDATE OR
    DELETE
  ON registrierungen
  FOR EACH ROW
  
  DECLARE 
    username VARCHAR(15);
    maniptype VARCHAR2(3);
    systemdatum DATE := SYSDATE;
  BEGIN
    SELECT USER INTO username FROM DUAL;
    CASE
      WHEN INSERTING THEN
        maniptype := 'INS';
        INSERT INTO registrierungen_log(username, systemdate, maniptype, oldval, newval) VALUES(username, systemdatum, maniptype, NULL, :NEW.mnr);
        INSERT INTO registrierungen_log(username, systemdate, maniptype, oldval, newval) VALUES(username, systemdatum, maniptype, NULL, :NEW.fnr);
        INSERT INTO registrierungen_log(username, systemdate, maniptype, oldval, newval) VALUES(username, systemdatum, maniptype, NULL, :NEW.anr);
        INSERT INTO registrierungen_log(username, systemdate, maniptype, oldval, newval) VALUES(username, systemdatum, maniptype, NULL, TO_CHAR(:NEW.datum, 'DD.MM.YYYY'));
      WHEN DELETING THEN
        maniptype := 'DEL';
        INSERT INTO registrierungen_log(username, systemdate, maniptype, oldval, newval) VALUES(username, systemdatum, maniptype, :OLD.mnr, NULL);
        INSERT INTO registrierungen_log(username, systemdate, maniptype, oldval, newval) VALUES(username, systemdatum, maniptype, :OLD.fnr, NULL);
        INSERT INTO registrierungen_log(username, systemdate, maniptype, oldval, newval) VALUES(username, systemdatum, maniptype, :OLD.anr, NULL);
        INSERT INTO registrierungen_log(username, systemdate, maniptype, oldval, newval) VALUES(username, systemdatum, maniptype, TO_CHAR(:OLD.datum, 'DD.MM.YYYY'), NULL);
      WHEN UPDATING THEN
        maniptype := 'UPD';
        CASE
          WHEN :OLD.mnr != :NEW.mnr THEN
            INSERT INTO registrierungen_log(username, systemdate, maniptype, oldval, newval) VALUES(username, systemdatum, maniptype, :OLD.mnr, :NEW.mnr);
          WHEN :OLD.fnr != :NEW.fnr THEN
            INSERT INTO registrierungen_log(username, systemdate, maniptype, oldval, newval) VALUES(username, systemdatum, maniptype, :OLD.fnr, :NEW.fnr);
          WHEN :OLD.anr != :NEW.anr THEN
            INSERT INTO registrierungen_log(username, systemdate, maniptype, oldval, newval) VALUES(username, systemdatum, maniptype, :OLD.anr, :NEW.anr);
          WHEN :OLD.datum != :NEW.datum THEN
            INSERT INTO registrierungen_log(username, systemdate, maniptype, oldval, newval) VALUES(username, systemdatum, maniptype, TO_CHAR(:OLD.datum, 'DD.MM.YYYY'), TO_CHAR(:NEW.datum, 'DD.MM.YYYY'));
        END CASE;
    END CASE;
  END;

-- Zweite Möglichkeit
CREATE OR REPLACE TRIGGER registrierungen_change
  AFTER 
    INSERT OR
    UPDATE OR
    DELETE
  ON registrierungen
  FOR EACH ROW
  DECLARE
    log_action VARCHAR2(3);
  BEGIN
    IF INSERTING THEN
      log_action := 'INS';
    ELSIF UPDATING THEN
      log_action := 'UPD';
    ELSIF DELETING THEN
      log_action := 'DEL';
    ELSE
      DBMS_OUTPUT.PUT_LINE('This code is not reachable.');
    END IF;
    INSERT INTO registrierungen_log(username, systemdate, maniptype, oldmnr, newmnr, oldfnr, newfnr, oldanr, newanr, olddatum, newdatum) 
    VALUES (USER, SYSDATE, log_action, :OLD.mnr, :NEW.mnr, :OLD.fnr, :NEW.fnr, :OLD.anr, :NEW.anr, :OLD.datum, :NEW.datum);
  END
;
\end{lstlisting}
\subsubsection{Lösungsbeschrieb}
\begin{tabular}{| p{5cm} | p{5cm} | p{5cm} |}
\hline
Code & Zweck & Kommentar\\ \hline
BEFORE, AFTER, INSTEAD OF & Das Timing des Triggers, wobei INSTEAD OF die auslösende Operation auf eine View mit Operationen des Triggers ersetzt.&\\ \hline
FOR EACH ROW & Der Trigger soll für jede Reihe einzeln ausgelöst werden, statt für ein Statement. & Für diesen Trigger sehr wichtig, da Manipulationen auf der Granularität von Zeilen geloggt werden müssen. Dadurch hat man auch Zugriff auf alte und neue Werte.\\ \hline
systemdatum DATE := SYSDATE;\newline
SELECT USER INTO username FROM DUAL; & Zwei verschiedene Möglichkeiten Systemvariablen in einer Variable zu speichern. & FROM DUAL muss verwendet werden, da ein SELECT nicht ohne FROM verwendet werden darf. Dabei wird DUAL als eine DUMMY Tabelle verwendet und enthaltet keine Informationen.\\ \hline
:NEW.mnr\newline
:OLD.mnr & Falls die Granularität des Triggers auf Zeilenebene fest gelegt ist, kann durch :NEW und :OLD auf die Werte zugegriffen werden, die vor und nach der Manipulation in der Tabelle stehen würden.&\\ \hline
INSERTING, DELETING, UPDATING & Können in einem WHEN-CASE verwendet werden, um abzufragen, bei welcher Operation der Trigger ausgelöst wurde. & Ist nur nötig, falls der Trigger bei mehr als eine Operation triggert.\\ \hline
\end{tabular}
\subsection{Referential Integrity}
\subsubsection{Aufgabenstellung}
In einer zentralisierten Datenbank (sämtliche Tabellen auf einem System) wird die referentielle Integrität vom Datenbanksystem überprüft. In einer verteilten Datenbank ist dies nicht mehr möglich, da die Datenbanksysteme voneinander getrennt sind. Die referentielle Integrität muss somit manuell über Trigger überprüft werden.\\
Für die Aufgabe werden die Tabellen Filialen[PK:FNr] und DVDKopien[PK:DVDNr, FK1:FNr] behandelt.
\subsubsection{Lösung}
\begin{lstlisting}[style=sql, title=Link ananke <=> helios]
CREATE DATABASE LINK orion.helios.fhnw.ch
CONNECT TO mvdbs00 IDENTIFIED BY mvdbs00
USING 'orion';

-- Mit Views, filialen-Trigger auf ananke
CREATE OR REPLACE VIEW filialen AS
SELECT * FROM filialen@orion.helios.fhnw.ch;

-- Ohne Views, filialen-Trigger auf helios
CREATE SYNONYM filialen FOR filialen@orion.helios.fhnw.ch;

CREATE SYNONYM dvdkopien FOR dvdkopien@ananke.hades.fhnw.ch;
\end{lstlisting}
\begin{lstlisting}[style=sql, title=FILIALEN-Trigger auf ananke]
CREATE OR REPLACE TRIGGER dvdkopien_ref_check_ins_upd
  BEFORE
    INSERT OR
    UPDATE
  ON dvdkopien
  FOR EACH ROW
  DECLARE
    msg VARCHAR2(100) := 'Integritäts-Constraint verletzt - übergeordneter Schlüssel nicht gefunden';
    cnt NUMBER;
  BEGIN
    SELECT COUNT(*) INTO cnt FROM filialen WHERE filialen.fnr = :NEW.fnr;
    IF (cnt = 0) THEN RAISE_APPLICATION_ERROR(-20000, msg); END IF;
  END;
\end{lstlisting}
\begin{lstlisting}[style=sql, title=DVDKOPIEN-Trigger auf helios]
CREATE OR REPLACE TRIGGER filialen_del_upd
  BEFORE DELETE OR UPDATE ON filialen
  FOR EACH ROW
  DECLARE 
    msg VARCHAR2(100) := 'Integritäts-Constraint verletzt - untergeordneter Datensatz gefunden';
    cnt NUMBER;
  BEGIN
    SELECT COUNT(*) INTO cnt FROM dvdkopien WHERE dvdkopien.fnr = :OLD.fnr;
    IF (cnt > 0) THEN RAISE_APPLICATION_ERROR(-20000, msg);
    END IF;
  END;
\end{lstlisting}
\subsubsection{Lösungsbeschrieb}
\begin{tabular}{| p{5cm} | p{5cm} | p{5cm} |}
\hline
Voraussetzungen & Synonym von FILIALEN auf ananke & Auf dem ursprünglichen System muss eine  ein Synonym (inkl. Datenbanklink) zu FILIALEN auf helios bestehen, damit auf dessen Daten zugegriffen werden kann, um die Integrität zu überprüfen.\\ \hline
Trigger-Lokalität & \multicolumn{2}{p{10cm}|}{Die Trigger sollten auf den Systemen installiert werden, auf denen sich ebenfalls die Tabellen befinden. Sämtliche Trigger können aber auch auf einem System installiert werden. Dann müssen die Trigger jedoch auf Views der entfernten Tabellen triggern und die Operation durch INSTEAD OF ersetzen, damit zuerst die Integrität überprüft und danach die ursprüngliche Operation ausgeführt werden kann.}\\ \hline
RAISE\_APPLICATION\_ERROR(-20000, msg); & Hier wird eine Fehlermeldung erzeugt. & Der Code ist dabei entscheidend, was angezeigt wird. -20000 ist nicht obligatorisch für dieses Lab, ein Besseres konnte jedoch nicht auf die Schnelle gefunden werden.\\ \hline
BEFORE & Logik des Triggers wird vor der Ausführung der Operation ausgeführt & In diesem Lab ist ein BEFORE-Trigger nötig, da durch das Werfen eines Fehlers die folgende Operationen abgebrochen wird.\\ \hline
\end{tabular}
\section{Lab 3 - Verteilter Datenbankentwurf}
\subsection{Aufgabenstellung}
Anhand von Anwendungen und deren Ausführungshäufigkeiten eine verteilte Datenbank entwerfen.\\
\begin{lstlisting}[style=sql, title=Anwendungen]
-- Anwendung q1
SELECT name, typ, maxmiete
FROM kunde;
-- Anwendung q2
SELECT telefonnummer, name, typ
FROM kunde
WHERE maxmiete <= 2000;
-- Anwendung q3
SELECT telefonnummer, maxmiete
FROM kunde;
-- Anwendung q4
SELECT DISTINCT(typ)
FROM kunde;
-- Anwendung q5
SELECT telefonnummer, name
FROM kunde
WHERE typ = 'Wohnung';
\end{lstlisting}
\textbf{Ausführungshäufigkeiten}\\
\begin{tabular}{|p{3cm}|p{3cm}|p{3cm}|}
\hline
Anwendung & System & Zugriffshäufigkeit\\ \hline
q1 & ananke & 10\\ \hline
q1 & telesto & 20\\ \hline
q2 & ananke & 5\\ \hline
q2 & calypso & 10\\ \hline
q3 & telesto & 35\\ \hline
q3 & calypso & 5\\ \hline
q4 & telesto & 10\\ \hline
q5 & telesto & 15\\ \hline
\end{tabular}
\subsection{Lösung}
\bigpic[Globale Affinität]{Labs/Lab3_VertikaleFragmentierung/lab3_global_aff.png}{17cm}
\bigpic[Bond Energy Algorithm]{Labs/Lab3_VertikaleFragmentierung/lab3_bea.png}{18cm}
\bigpic[Bond Energy Algorithm]{Labs/Lab3_VertikaleFragmentierung/lab3_bea2.png}{8cm}
\bigpic[Bond Energy Algorithm]{Labs/Lab3_VertikaleFragmentierung/lab3_bea3.png}{7cm}
\bigpic[Hilfsmatrizen]{Labs/Lab3_VertikaleFragmentierung/lab3_help.png}{17cm}
\bigpic[Splitting und Fragmentierung]{Labs/Lab3_VertikaleFragmentierung/lab3_splitting_fragments.png}{18cm}
\bigpic[Korrektheit]{Labs/Lab3_VertikaleFragmentierung/lab3_correctness.png}{18cm}
\subsection{Lösungsbeschrieb}
\subsubsection{Weg zur globalen Affinität}
Als Grundlage für die Usage Matrix dienen die Anwendung/Queries. Bei jedem Attribut, auf die ein Query zugreift, wird das Flag[1=true,0=false] in der Matrix gesetzt. Beim Zusammentragen der Attribute muss man sich darauf achten, dass nicht nur der Select berücksichtigt wird, sondern auch andere Zugriffe, wie zum Beispiel im Where.\\
Die Zugriffshäufigkeitsmatrix wird durch das Zusammentragen der Informationen aus der Ausführungshäufigkeit der einzelnen Queries auf den einzelnen Systemen gebildet. Es ist ratsam für weitere Anwendung gleich eine Summe zu bilden, wie viel mal auf ein Query zugegriffen wird.\\
Die Attributs-Affinität Matrix bildet nun sämtliche Attribute ab. Diese Matrix kann wie folgt ausgefüllt werden:
\begin{itemize}
\item Parameter Zeile / Spalte => Name / Telefon
\item In der Usage Matrix prüfen welche Queries auf diese Attribute zugreifen, also eine 1 in beiden \item Feldern stehen haben => q2, q5
\item Zugriffshäufigkeit dieser Queries aufsummieren => 15 + 15 = 30
\item In der AA-Matrix unter Zeile / Spalte und Spalte / Zeile eintragen
\end{itemize}
Da die AA-Matrix symmetrisch an der Diagonale aufgebaut ist, können meistens Zeile und Spalte vertauscht werden, um das gleiche Resultat in der Matrix einzutragen.\\
\textbf{Achtung!} Der Bond Energy Algorithmus ist hier vereinfacht dargestellt, da mit Excel gearbeitet wurde. Beim korrekten Vorgehen muss der Beitrag einer Änderung berechnet werden. Dabei kann wie folgt überlegt werden:
\begin{itemize}
\item Die "'lokale Affinität"' zweier benachbarter Spalten kann berechnet werden indem Werte der benachbarten Spalten in einer Zeile multipliziert und anschliessend zeilenweise aufsummiert werden. Beispiel Name Telefon: 60*30 + 30*70 + 60*30 + 45*55 = 8175.
\item Wird bei BEA nun eine Änderung durch das Hinzufügen einer weiteren Spalte bewirkt, wird der Beitrag wie folgt berechnet:
\begin{itemize}
\item bestehende Nachbarschaften werden nicht aufgelöst: "'lokale Affinität"' berechnen; das ist der Beitrag; dies ist der Fall, wenn der neue Nachbar links oder rechts der Matrix hinzugefügt wird
\item bestehende Nachbarschaften werden aufgelöst: "'lokale Affinität"' der aufgelösten Nachbarn berechnen (z.B. Name-Telefon), dieser Wert wird vom Beitrag abgezogen; "'lokale Affinität"' des neuen Nachbarn jeweils mit der rechten und linken Spalte berechnen (z.B. Name-Typ, Typ-Telefon); diese Werte werden zum Beitrag addiert
\end{itemize}
\item Die Änderung mit dem grössten Beitrag wird durchgeführt und die nächste Iteration beginnt.
\item Sind alle Iterationen durch, können die Zeilen entsprechend den Spalten angeordnet werden.
\end{itemize}
Es wurde die Matrix berechnet, die die grösste globale Affinität besitzt. Nun muss noch berechnet werden, wie die Fragmente dieser Tabelle gebildet werden sollen. Dazu wird entlang der Diagonalen der Matrix der Trennpunkt verschoben. Es muss nun die Fragmentierung mit der grössten Trennqualität gefunden werden. Der Algorithmus acc berechnet sich wie folgt:
\begin{itemize}
\item ein Parameter: falls in einer Zeile/Query in der Usage Matrix in irgend einer Spalte des Parameters/des Fragments eine 1 steht und in allen Spalten, die nicht zum Fragments gehören, eine 0, dann wird die Zugriffshäufigkeit des Queries zum Resultat addiert
\item zwei Parameter: gleich wie bei einem Parameter, nur muss mindestens die Bedingung für Spalten des ersten UND des zweiten Fragments zutreffen
\end{itemize}
Für das Splitting wählt man ebenfalls die Variante mit dem höchsten Wert aus und drückt es anschliessend in algebraischer Form aus.
% part labs_ (end)
\end{document}
