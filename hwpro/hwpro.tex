\documentclass[a4paper, 11pt]{article}

%\usepackage[utf8]{input}

%Math
\usepackage{amsmath}
\usepackage{amsfonts}
\usepackage{amssymb}
\usepackage{amsthm}
\usepackage{ulem}
\usepackage{stmaryrd} %f\UTF{00FC}r Blitz!

%PageStyle
\usepackage[ngerman]{babel} % deutsche Silbentrennung
\usepackage[ansinew]{inputenc} % wegen deutschen Umlauten
\usepackage{fontenc}
\usepackage{fancyhdr, graphicx} %for header/footer
\usepackage{wasysym}
\usepackage{fullpage}
\usepackage{textcomp}

% Listings
\usepackage{color}
\usepackage{xcolor}
\usepackage{listings}
\usepackage{caption}

% Code listenings
\DeclareCaptionFont{white}{\color{white}}
\DeclareCaptionFormat{listing}{\colorbox{gray}{\parbox{\textwidth}{#1#2#3}}}
\captionsetup[lstlisting]{format=listing,labelfont=white,textfont=white}

\definecolor{DarkPurple}{rgb}{0.4,0.1,0.4}
\definecolor{DarkCyan}{rgb}{0.0,0.5,0.4}
\definecolor{LightLime}{rgb}{0.4,0.6,0.5}
\definecolor{Blue}{rgb}{0.0,0.0,1.0}
 
\lstdefinestyle{JavaStyle}{
 language=Java,
 basicstyle=\footnotesize\ttfamily, % Standardschrift
 numbers=left,               % Ort der Zeilennummern
 numberstyle=\tiny,          % Stil der Zeilennummern
 stepnumber=5,              % Abstand zwischen den Zeilennummern
 numbersep=5pt,              % Abstand der Nummern zum Text
 tabsize=2,                  % Groesse von Tabs
 extendedchars=true,         %
 breaklines=true,            % Zeilen werden Umgebrochen
 frame=b,         
 %commentstyle=\itshape\color{LightLime}, Was isch das? O_o
 %keywordstyle=\bfseries\color{DarkPurple}, und das O_o
 basicstyle=\footnotesize\ttfamily,
 stringstyle=\color[RGB]{42,0,255}\ttfamily, % Farbe der String
 keywordstyle=\color[RGB]{127,0,85}\ttfamily, % Farbe der Keywords
 commentstyle=\color[RGB]{63,127,95}\ttfamily, % Farbe des Kommentars
 showspaces=false,           % Leerzeichen anzeigen ?
 showtabs=false,             % Tabs anzeigen ?
 xleftmargin=17pt,
 framexleftmargin=17pt,
 framexrightmargin=3pt,
 framexbottommargin=2pt,
 showstringspaces=false      % Leerzeichen in Strings anzeigen ?        
}

%Config
\renewcommand{\headrulewidth}{0pt}
\setlength{\headheight}{15.2pt}
\pagestyle{plain}

\begin{document}

\section{Register}
\begin{itemize}
	\item \%esp = stack pointer
	\item \%ebp = base pointer
	\item \%eax = accumulator, return Werte von Funktionen werden hier abgelegt.
	\item \%ebx = base index (array manipulation)
	\item \%ecx = counter (array manipulation)
	\item \%edx = data / general register
	\item \%esi = source index (string manipulation)
	\item \%edi = destination index (string manipulation)
	\item \%eip = instruction pointer
\end{itemize}
Ausser \%eip und \%esp sind alles General Purpose Register, man kann auch \%ebx für eine Array-Manipulation verwenden.

\subsubsection{movl}
movl kann in drei Varianten verwendet werden:
\begin{itemize}
	\item movl "register", "register"
	\item movl "register, [Expression]
	\item movl [Expression], "register"
\end{itemize}

Generelle Funktion für Expressions: $D(Rb,Ri,S)= Mem[Reg[Rb]+S \cdot Reg[Ri]+D]$\\
\begin{itemize}
	\item D: Konstante in Byte(4 Byte für 64b)
	\item Rb: Base Register
	\item Ri: Index Register, können alle sein ausser \%esp und \%ebp
	\item S: Skalar in Zweierpotenz
\end{itemize}

Beispiele:
\begin{tabular}{|c|c|c|}
	\hline
	Ausdruck & Berechnung & Adresse im Hauptspeicher \\\hline\hline
	0x8(\%edx) & 0xf000 + 0x8 & 0xf008 \\\hline
	(\%edx,\%ecx,4)&0xf000 + 4*0x100&0xf400\\\hline
	0x80(,\%edx,2)&2*0xf000 + 0x80 & 0x1e080\\\hline
\end{tabular}

\section{Function Call}
\subsection{Stack Frame}
\%ebp zeigt immer auf die "Basis" des stacks, heisst alle Adressen kleiner als \%ebp gehören zur momentan ausgeführten Methode. Die Parameter dieser Methode sind dabei auf den Adressen grösser als \%ebp abgespeichert. Die Speicherstelle, auf die \%ebp hinzeigt, ist der \&ebp Wert der vorherigen Methode. 4(\%ebp) beinhaltet die Return-Adresse für diese Methode, alles höher als 4(\%ebp) sind Parameter der momentanen Methode.

\subsection{Function Call Setup}
Nachdem der Aufrufer die Parameter auf den Stack abgelegt und "Call Function" ausgeführt hat.
\begin{lstlisting}
	pushl \%ebp
	movl \%esp, \%ebp
\end{lstlisting}

Stack Frame
\begin{tabular}{|c|c}
	$\vdots$&\\\cline{0-0}
	yp\\\cline{0-0}
	xp\\\cline{0-0}
	ret addr\\\cline{0-0}
	\%ebp & $\leftarrow$ pushl \%ebp, movl \%esp, \%ebp\\\cline{0-0}
	\%ebx & $\leftarrow$ pushl \%ebx\\\cline{0-0}
\end{tabular}\\

\subsection{Function Call Teardown}
\begin{lstlisting}
	//allenfalls Returnwert in \%eax speichern
	movl \%ebp, \%esp
	pop \%ebp
	return
\end{lstlisting}

\section{Instruktionen}
\subsection{Arithmetische Operatoren}

\subsubsection{Binäre Operatoren}
Alle binären Operatoren lesen aus dem Source Register und den berechneten Wert in das Destination Register.

\begin{tabular}{|c|c|}
	\hline
	Befehl & Beschreibung\\\hline\hline
	addl & Dest += Source\\\hline
	subl & Dest -= Source \\\hline
	imull & Dest *= Source \\\hline
	sall & Dest $<<$ Source\\\hline
	sarl & Dest $>>$ Source, füllt mit 1 auf falls MSB = 1 \\\hline
	shrl & Dest $>>$ Source, füllt immer mit 0 auf\\\hline
	leal & siehe LEA Instruction. \\\hline
	xorl & \ldots \\\hline
	andl &\ldots  \\\hline
	orl & \ldots \\\hline
\end{tabular}

\subsubsection{Unäre Operatoren}
\begin{tabular}{|c|c|}
	\hline
	Befehl & Beschreibung\\\hline\hline
	incl & increment \\\hline
	decl & decrement \\\hline
	negl & negate \\\hline
	notl & not operator\\\hline
\end{tabular}

\subsubsection{LEA Instruction}
Vom Internet: LEA, the only instruction that performs memory addressing calculations but doesn't actually address memory. LEA accepts a standard memory addressing operand, but does nothing more than store the calculated memory offset in the specified register, which may be any general purpose register.

What does that give us? Two things that ADD doesn't provide:

the ability to perform addition with either two or three operands, and
the ability to store the result in any register; not just one of the source operands.

\subsection{Compare und Konditionen}
\begin{tabular}{|c|c|}
	\hline
	cmpl & compare\\\hline
	testl & (jmp \%eax \%eax) überprüft, ob \%eax grösser,kleiner oder = 0 ist.\\\hline
\end{tabular}

\begin{tabular}{|c|c|}
	\hline
	

\end{tabular}

\subsubsection{Flags}
\begin{tabular}{|c|c|c|}
	\hline
	Abkürzung & Name & wird gesetzt durch \\\hline 
	ZF & Zero Flag & wird von testl gesetzt.\\\hline
	SF & Signed Flag & wird von testl gesetzt.\\\hline
	OF & Overflow Flag & von arithmetischen Operationen gesetzt.\\\hline
	CF & Carry Flag & von arithmetischen Operationen gesetzt.\\\hline
\end{tabular}

\begin{tabular}{|c|c|c|}
	\hline
	Befehl & Ausdruck & Beschreibung\\\hline 
	sete & $ZF$ & Equal / Zero\\\hline 
	setne & $~ZF$ & Not Equal / Not Zero\\\hline 
	sets & $SF$ & Negative\\\hline 
	setns & $~SF$ & Nonnegative\\\hline 
 
\end{tabular}

\subsection{Control Statements}
\begin{tabular}{|c|c|}
	\hline
	jmp (label) & jump \\\hline
	je (label) & jump equal \\\hline
	jne (label) & jump not equal  \\\hline
	js (label) & jump negative \\\hline
	jns (label) & jump not negative  \\\hline
	jg (label) & jump greater \\\hline
	jge (label) & jump greater or equal \\\hline
	jl (label) & jump less  \\\hline
	jle (label) & jump less or equal \\\hline
	ja (label) & jump above (unsigned) \\\hline
	jb (label) & jump below (unsigned) \\\hline
	push Src &  \\\hline
	pop Dest & \\\hline
	call (label) &  \\\hline
	ret &  \\\hline
\end{tabular}

\section{Loops und If's}

\subsection{If Statement}
\subsubsection{Unter 32Bit}
C Code:
\begin{lstlisting}
	int absdiff(int x,int y)
	{
		int result;
		
		if(x > y)
			result = x-y;
		else
			result y-x;
			
		return result;
	}
\end{lstlisting}

Assembler:
\begin{lstlisting}
	absdiff:
		pushl %ebp
		movl %esp,%ebp
		
		movl 8(%ebp),%edx
		movl 12(%ebp),%eax
		cmpl %eax,%edx
		jle .L7			 
		movl %edx,%eax 
	.L8:
		movl %ebp,%esp
		popl %ebp
		ret
	.L7:
		subl %edx, %eax
		jmp .L8
\end{lstlisting}

\subsubsection{Unter 64Bit}
\begin{lstlisting}
	absdiff:
		pushl %ebp
		movl %esp,%ebp
		
		movl %edi, %eax	# v = x
		movl %esi, %edx	# ve = y
		subl %esi, %eax # v -= y
		subl %edi, %edx # ve -= x
		cmpl %esi, %edi # x:y
		cmovle %edxm %eax # v=ve if <=
		movl %ebp,%esp
		popl %ebp
		ret
\end{lstlisting}

\subsection{Loops}
\begin{lstlisting}
	
\end{lstlisting}

\section{Bitwise Magix}
\begin{lstlisting}
	int bitXor(int x, int y) {
		return ~x & y;
	}
	
	int isEqual(int x, int y) {
		return !(x ^ y);
	}
\end{lstlisting}


\end{document}
